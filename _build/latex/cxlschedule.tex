%% Generated by Sphinx.
\def\sphinxdocclass{report}
\documentclass[a4paper,10pt,english]{sphinxmanual}
\ifdefined\pdfpxdimen
   \let\sphinxpxdimen\pdfpxdimen\else\newdimen\sphinxpxdimen
\fi \sphinxpxdimen=.75bp\relax


\catcode`^^^^00a0\active\protected\def^^^^00a0{\leavevmode\nobreak\ }
\usepackage{cmap}
\usepackage{fontspec}
\usepackage{amsmath,amssymb,amstext}
\usepackage{polyglossia}
\setmainlanguage{english}

\usepackage[Bjarne]{fncychap}
\usepackage[dontkeepoldnames]{sphinx}

\usepackage{geometry}

% Include hyperref last.
\usepackage{hyperref}
% Fix anchor placement for figures with captions.
\usepackage{hypcap}% it must be loaded after hyperref.
% Set up styles of URL: it should be placed after hyperref.
\urlstyle{same}

\addto\captionsenglish{\renewcommand{\figurename}{Fig.}}
\addto\captionsenglish{\renewcommand{\tablename}{Table}}
\addto\captionsenglish{\renewcommand{\literalblockname}{Listing}}

\addto\captionsenglish{\renewcommand{\literalblockcontinuedname}{continued from previous page}}
\addto\captionsenglish{\renewcommand{\literalblockcontinuesname}{continues on next page}}

\def\pageautorefname{page}

\setcounter{tocdepth}{1}


        \usepackage{xeCJK}
    

\title{cxl schedule Documentation}
\date{Dec 15, 2017}
\release{1.0.0}
\author{cxl}
\newcommand{\sphinxlogo}{\vbox{}}
\renewcommand{\releasename}{Release}
\makeindex

\begin{document}

\maketitle
\sphinxtableofcontents
\phantomsection\label{\detokenize{index::doc}}



\chapter{introduce}
\label{\detokenize{intro::doc}}\label{\detokenize{intro:introduce}}\label{\detokenize{intro:welcome-to-cxl-schedule-s-documentation}}\begin{quote}

\sphinxstylestrong{good good study ,day day up !}

\sphinxstyleemphasis{good good study ,day day up !}

\sphinxcode{good good study ,day day up !}
\end{quote}
\phantomsection\label{\detokenize{intro:my-reference-label}}\begin{quote}

\sphinxstyleemphasis{good good study ,day day up !}$_{\text{oMygod}}$

\sphinxstyleemphasis{good good study ,day day up !}$^{\text{oMygod}}$

\sphinxtitleref{good good study ,day day up !}
\end{quote}
\begin{itemize}
\item {} 
one

\item {} 
two
three

\end{itemize}
\begin{itemize}
\item {} 
onechild

\item {} 
twochild

\end{itemize}
\begin{enumerate}
\item {} 
one

\item {} 
two我是一个ingnew

\end{enumerate}
\begin{quote}

\begin{DUlineblock}{0em}
\item[] one
\item[] two
\end{DUlineblock}
\end{quote}

\noindent\sphinxincludegraphics{{c0d}.png}


\begin{savenotes}\sphinxattablestart
\centering
\sphinxcapstartof{table}
\sphinxcaption{Frozen Delights!}\label{\detokenize{intro:id3}}
\sphinxaftercaption
\begin{tabular}[t]{|\X{15}{55}|\X{10}{55}|\X{30}{55}|}
\hline
\sphinxstylethead{\sphinxstyletheadfamily 
Treat
\unskip}\relax &\sphinxstylethead{\sphinxstyletheadfamily 
Quantity
\unskip}\relax &\sphinxstylethead{\sphinxstyletheadfamily 
Description
\unskip}\relax \\
\hline
Albatross
&
2.99
&
On a stick!
\\
\hline
Crunchy Frog
&
1.49
&
If we took the bones out, it wouldn’t be
crunchy, now would it?
\\
\hline
Gannet Ripple
&
1.99
&
On a stick!
\\
\hline
\end{tabular}
\par
\sphinxattableend\end{savenotes}


\section{read the docs}
\label{\detokenize{intro:read-the-docs}}

\subsection{refer}
\label{\detokenize{intro:refer}}

\sphinxstrong{See also:}


\begin{DUlineblock}{0em}
\item[] \sphinxhref{https://readthedocs.org/}{readthedocs官方网址}
\item[] \sphinxhref{http://www.sphinx-doc.org/en/stable/rest.html}{sphinx语法文档}
\item[] \sphinxhref{http://www.jianshu.com/p/78e9e1b8553a}{个人博客文档}
\item[] \sphinxhref{https://zh-sphinx-doc.readthedocs.io/en/latest/contents.html}{sphinx中文语法文档}
\end{DUlineblock}




\subsection{install}
\label{\detokenize{intro:install}}\begin{enumerate}
\item {} 
apt install build-essential

\item {} 
apt install python-dev

\item {} 
apt install python-pip

\item {} 
apt install python-setuptools

\item {} 
apt install libxml2-dev

\item {} 
apt install libxslt1-dev

\item {} 
apt install zlib1g-dev

\item {} 
apt install elasticsearch

\item {} 
apt install redis-server

\item {} 
pip install virtualenv

\item {} 
virtualenv rtd \&\& cd rtd

\item {} 
source bin/activate

\item {} 
apt install git

\item {} 
git clone \sphinxurl{https://github.com/rtfd/readthedocs.org.git}

\item {} 
cd readthedocs.org

\item {} 
pip install -r requirements.txt

\item {} 
python manage.py migrate

\item {} 
python manage.py collectstatic

\item {} 
python manage.py loaddata test\_data

\item {} 
python manage.py runserver 0.0.0.0:8000

\item {} 
apt install texlive-latex-recommended

\item {} 
apt install texlive-fonts-recommended

\item {} 
apt install texlive-latex-extra

\item {} 
apt install latexmk

\item {} 
make latexpdf LATEXMKOPTS=”-silent”

\item {} 
apt install texlive-xetex

\item {} 
apt install latex-cjk-all

\end{enumerate}


\subsection{usage}
\label{\detokenize{intro:usage}}\begin{itemize}
\item {} 
install Sphinx

\begin{sphinxVerbatim}[commandchars=\\\{\}]
\PYG{n}{apt} \PYG{n}{install} \PYG{n}{python}\PYG{o}{\PYGZhy{}}\PYG{n}{pip} \PYG{o}{\PYGZhy{}}\PYG{n}{y}
\PYG{n}{pip} \PYG{n}{install} \PYG{n}{sphinx} \PYG{n}{sphinx}\PYG{o}{\PYGZhy{}}\PYG{n}{autobuild}
\PYG{n}{pip} \PYG{n}{install} \PYG{o}{\PYGZhy{}}\PYG{o}{\PYGZhy{}}\PYG{n}{upgrade} \PYG{n}{pip} \PYG{n}{安装最新版pip}
\PYG{n}{mkdir} \PYG{n}{doc} \PYG{o}{\PYGZam{}}\PYG{o}{\PYGZam{}} \PYG{n}{cd} \PYG{n}{doc}
\PYG{n}{sphinx}\PYG{o}{\PYGZhy{}}\PYG{n}{quickstart}
\PYG{n}{make} \PYG{n}{html}
\end{sphinxVerbatim}

\item {} \begin{description}
\item[{conf.py}] \leavevmode
latex\_engine=xelatex 解决无法生成pdf问题

\end{description}

\end{itemize}


\chapter{LINUX}
\label{\detokenize{linux/index::doc}}\label{\detokenize{linux/index:linux}}

\section{point}
\label{\detokenize{linux/point::doc}}\label{\detokenize{linux/point:point}}

\subsection{bootloader}
\label{\detokenize{linux/point:bootloader}}

\subsubsection{refer}
\label{\detokenize{linux/point:refer}}
\sphinxurl{http://blog.csdn.net/darling757267/article/details/52356354}
\sphinxurl{http://www.cnblogs.com/unicode/archive/2010/06/12/1756755.html}


\subsubsection{ubuntu启动过程}
\label{\detokenize{linux/point:ubuntu}}
假如您使用的 Linux 发行版是 Ubuntu,很可能会发现在您的计算机上找不到/etc/inittab 文件了,这是因为 Ubuntu 使用了一种被称为 upstart 的新型 init 系统。


\subsubsection{开发 Upstart 的缘由}
\label{\detokenize{linux/point:upstart}}
大约在 2006 年或者更早的时候, Ubuntu 开发人员试图将 Linux 安装在笔记本电脑上。在这期间技术人员发现经典的 sysvinit 存在一些问题:它不适合笔记本环境。这促使程序员 Scott James Remnant 着手开发 upstart。
当 Linux 内核进入 2.6 时代时,内核功能有了很多新的更新。新特性使得 Linux 不仅是一款优秀的服务器操作系统,也可以被用于桌面系统,甚至嵌入式设备。桌面系统或便携式设备的一个特点是经常重启,而且要频繁地使用硬件热插拔技术。在现代计算机系统中,硬件繁多、接口有限,人们并非将所有设备都始终连接在计算机上,比如 U 盘平时并不连接电脑,使用时才插入 USB 插口。因此,当系统上电启动时,一些外设可能并没有连接。而是在启动后当需要的时候才连接这些设备。在 2.6 内核支持下,一旦新外设连接到系统,内核便可以自动实时地发现它们,并初始化这些设备,进而使用它们。这为便携式设备用户提供了很大的灵活性。
可是这些特性为 sysvinit 带来了一些挑战。当系统初始化时,需要被初始化的设备并没有连接到系统上;比如打印机。为了管理打印任务,系统需要启动 CUPS 等服务,而如果打印机没有接入系统的情况下,启动这些服务就是一种浪费。Sysvinit 没有办法处理这类需求,它必须一次性把所有可能用到的服务都启动起来,即使打印机并没有连接到系统,CUPS 服务也必须启动。
还有网络共享盘的挂载问题。在/etc/fstab 中,可以指定系统自动挂载一个网络盘,比如 NFS,或者 iSCSI 设备。在本文的第一部分 sysvinit 的简介中可以看到,sysvinit 分析/etc/fstab 挂载文件系统这个步骤是在网络启动之前。可是如果网络没有启动,NFS 或者 iSCSI 都不可访问,当然也无法进行挂载操作。Sysvinit 采用 netdev 的方式来解决这个问题,即/etc/fstab 发现 netdev 属性挂载点的时候,不尝试挂载它,在网络初始化并使能之后,还有一个专门的 netfs 服务来挂载所有这些网络盘。这是一个不得已的补救方法,给管理员带来不便。部分新手管理员甚至从来也没有听说过 netdev 选项,因此经常成为系统管理的一个陷阱。
针对以上种种情况,Ubuntu 开发人员在评估了当时的几个可选 init 系统之后,决定重新设计和开发一个全新的 init 系统,即 UpStart。UpStart 基于事件机制,比如 U 盘插入 USB 接口后,udev 得到内核通知,发现该设备,这就是一个新的事件。UpStart 在感知到该事件之后触发相应的等待任务,比如处理/etc/fstab 中存在的挂载点。采用这种事件驱动的模式,upstart 完美地解决了即插即用设备带来的新问题。
此外,采用事件驱动机制也带来了一些其它有益的变化,比如加快了系统启动时间。sysvinit 运行时是同步阻塞的。一个脚本运行的时候,后续脚本必须等待。这意味着所有的初始化步骤都是串行执行的,而实际上很多服务彼此并不相关,完全可以并行启动,从而减小系统的启动时间。在 Linux 大量应用于服务器的时代,系统启动时间也许还不那么重要;然而对于桌面系统和便携式设备,启动时间的长短对用户体验影响很大。此外云计算等新的 Server 端技术也往往需要单个设备可以更加快速地启动。
UpStart 满足了这些需求,目前不仅桌面系统 Ubuntu 采用了 UpStart,甚至企业级服务器级的 RHEL 也默认采用 UpStart 来替换 sysvinit 作为 init 系统。


\subsubsection{Upstart 的特点}
\label{\detokenize{linux/point:id1}}
UpStart 解决了之前提到的 sysvinit 的缺点。采用事件驱动模型,UpStart 可以:
更快地启动系统
当新硬件被发现时动态启动服务
硬件被拔除时动态停止服务
这些特点使得 UpStart 可以很好地应用在桌面或者便携式系统中,处理这些系统中的动态硬件插拔特性。


\subsubsection{Upstart 概念和术语}
\label{\detokenize{linux/point:id2}}
Upstart 的基本概念和设计清晰明确。UpStart 主要的概念是 job 和 event。Job 就是一个工作单元,用来完成一件工作,比如启动一个后台服务,或者运行一个配置命令。每个 Job 都等待一个或多个事件,一旦事件发生,upstart 就触发该 job 完成相应的工作。
\begin{itemize}
\item {} 
Job

\end{itemize}

Job 就是一个工作的单元,一个任务或者一个服务。可以理解为 sysvinit 中的一个服务脚本。有三种类型的工作:
task job;
service job;
abstract job;
task job 代表在一定时间内会执行完毕的任务,比如删除一个文件;
service job 代表后台服务进程,比如 apache httpd。这里进程一般不会退出,一旦开始运行就成为一个后台精灵进程,由 init 进程管理,如果这类进程退出,由 init 进程重新启动,它们只能由 init 进程发送信号停止。它们的停止一般也是由于所依赖的停止事件而触发的,不过 upstart 也提供命令行工具,让管理人员手动停止某个服务;
Abstract job 仅由 upstart 内部使用,仅对理解 upstart 内部机理有所帮助。我们不用关心它。
除了以上的分类之外,还有另一种工作(Job)分类方法。Upstart 不仅可以用来为整个系统的初始化服务,也可以为每个用户会话(session)的初始化服务。系统的初始化任务就叫做 system job,比如挂载文件系统的任务就是一个 system job;用户会话的初始化服务就叫做 session job。
\begin{itemize}
\item {} 
Job 生命周期

\end{itemize}

Upstart 为每个工作都维护一个生命周期。一般来说,工作有开始,运行和结束这几种状态。为了更精细地描述工作的变化,Upstart 还引入了一些其它的状态。比如开始就有开始之前(pre-start),即将开始(starting)和已经开始了(started)几种不同的状态,这样可以更加精确地描述工作的当前状态。
工作从某种初始状态开始,逐渐变化,或许要经历其它几种不同的状态,最终进入另外一种状态,形成一个状态机。在这个过程中,当工作的状态即将发生变化的时候,init 进程会发出相应的事件(event)。
表 1.Upstart 中 Job 的可能状态
状态名 含义
Waiting 初始状态
Starting    Job 即将开始
pre-start   执行 pre-start 段,即任务开始前应该完成的工作
Spawned 准备执行 script 或者 exec 段
post-start  执行 post-start 动作
Running interim state set after post-start section processed denoting job is running (But it may have no associated PID!)
pre-stop    执行 pre-stop 段
Stopping    interim state set after pre-stop section processed
Killed  任务即将被停止
post-stop   执行 post-stop 段
图 1 展示了 Job 的状态机。

其中有四个状态会引起 init 进程发送相应的事件,表明该工作的相应变化:
Starting
Started
Stopping
Stopped
而其它的状态变化不会发出事件。那么我们接下来就来看看事件的详细含义吧。
\begin{itemize}
\item {} 
事件 Event

\end{itemize}

顾名思义,Event 就是一个事件。事件在 upstart 中以通知消息的形式具体存在。一旦某个事件发生了,Upstart 就向整个系统发送一个消息。没有任何手段阻止事件消息被 upstart 的其它部分知晓,也就是说,事件一旦发生,整个 upstart 系统中所有工作和其它的事件都会得到通知。
Event 可以分为三类: signal,methods 或者 hooks。
Signals
Signal 事件是非阻塞的,异步的。发送一个信号之后控制权立即返回。
Methods
Methods 事件是阻塞的,同步的。
\begin{itemize}
\item {} 
Hooks

\end{itemize}

Hooks 事件是阻塞的,同步的。它介于 Signals 和 Methods 之间,调用发出 Hooks 事件的进程必须等待事件完成才可以得到控制权,但不检查事件是否成功。
事件是个非常抽象的概念,下面我罗列出一些常见的事件,希望可以帮助您进一步了解事件的含义:
系统上电启动,init 进程会发送”start”事件
根文件系统可写时,相应 job 会发送文件系统就绪的事件
一个块设备被发现并初始化完成,发送相应的事件
某个文件系统被挂载,发送相应的事件
类似 atd 和 cron,可以在某个时间点,或者周期的时间点发送事件
另外一个 job 开始或结束时,发送相应的事件
一个磁盘文件被修改时,可以发出相应的事件
一个网络设备被发现时,可以发出相应的事件
缺省路由被添加或删除时,可以发出相应的事件
不同的 Linux 发行版对 upstart 有不同的定制和实现,实现和支持的事件也有所不同,可以用man 7 upstart-events来查看事件列表。
\begin{itemize}
\item {} 
Job 和 Event 的相互协作

\end{itemize}

Upstart 就是由事件触发工作运行的一个系统,每一个程序的运行都由其依赖的事件发生而触发的。
系统初始化的过程是在工作和事件的相互协作下完成的,可以大致描述如下:系统初始化时,init 进程开始运行,init 进程自身会发出不同的事件,这些最初的事件会触发一些工作运行。每个工作运行过程中会释放不同的事件,这些事件又将触发新的工作运行。如此反复,直到整个系统正常运行起来。
究竟哪些事件会触发某个工作的运行?这是由工作配置文件定义的。
\begin{itemize}
\item {} 
工作配置文件

\end{itemize}

任何一个工作都是由一个工作配置文件(Job Configuration File)定义的。这个文件是一个文本文件,包含一个或者多个小节(stanza)。每个小节是一个完整的定义模块,定义了工作的一个方面,比如 author 小节定义了工作的作者。工作配置文件存放在/etc/init 下面,是以.conf 作为文件后缀的文件。
清单 1. 一个最简单的工作配置文件

\#This is a simple demo of Job Configure file
\#This line is comment, start with \#

\#Stanza 1, The author
author “Liu Ming”

\#Stanza 2, Description
description “This job only has author and description, so no use, just a demo”
上面的例子不会产生任何作用,一个真正的工作配置文件会包含很多小节,其中比较重要的小节有以下几个:
\begin{itemize}
\item {} 
“expect” Stanza

\end{itemize}

Upstart 除了负责系统的启动过程之外,和 SysVinit 一样,Upstart 还提供一系列的管理工具。当系统启动之后,管理员可能还需要进行维护和调整,比如启动或者停止某项系统服务。或者将系统切换到其它的工作状态,比如改变运行级别。本文后续将详细介绍 Upstart 的管理工具的使用。
为了启动,停止,重启和查询某个系统服务。Upstart 需要跟踪该服务所对应的进程。比如 httpd 服务的进程 PID 为 1000。当用户需要查询 httpd 服务是否正常运行时,Upstart 就可以利用 ps 命令查询进程 1000,假如它还在正常运行,则表明服务正常。当用户需要停止 httpd 服务时,Upstart 就使用 kill 命令终止该进程。为此,Upstart 必须跟踪服务进程的进程号。
部分服务进程为了将自己变成后台精灵进程(daemon),会采用两次派生(fork)的技术,另外一些服务则不会这样做。假如一个服务派生了两次,那么 UpStart 必须采用第二个派生出来的进程号作为服务的 PID。但是,UpStart 本身无法判断服务进程是否会派生两次,为此在定义该服务的工作配置文件中必须写明 expect 小节,告诉 UpStart 进程是否会派生两次。
Expect 有两种,”expect fork”表示进程只会 fork 一次;”expect daemonize”表示进程会 fork 两次。
\begin{itemize}
\item {} 
“exec” Stanza 和”script” Stanza

\end{itemize}

一个 UpStart 工作一定需要做些什么,可能是运行一条 shell 命令,或者运行一段脚本。用”exec”关键字配置工作需要运行的命令;用”script”关键字定义需要运行的脚本。
清单 2 显示了 exec 和 script 的用法:
清单 2.script 例子

\# mountall.conf
description “Mount filesystems on boot”
start on startup
stop on starting rcS
…
script

\begin{sphinxVerbatim}[commandchars=\\\{\}]
. /etc/default/rcS
[ \PYGZhy{}f /forcefsck ] \PYGZam{}\PYGZam{} force\PYGZus{}fsck=”\PYGZhy{}\PYGZhy{}force\PYGZhy{}fsck”
[ “\PYGZdl{}FSCKFIX”=”yes” ] \PYGZam{}\PYGZam{} fsck\PYGZus{}fix=”\PYGZhy{}\PYGZhy{}fsck\PYGZhy{}fix”

...

exec mountall \textendash{}daemon \PYGZdl{}force\PYGZus{}fsck \PYGZdl{}fsck\PYGZus{}fix
\end{sphinxVerbatim}

end script
…
这是 mountall 的例子,该工作在系统启动时运行,负责挂载所有的文件系统。该工作需要执行复杂的脚本,由”script”关键字定义;在脚本中,使用了 exec 来执行 mountall 命令。
\begin{itemize}
\item {} 
“start on” Stanza 和”stop on” Stanza

\end{itemize}

“start on”定义了触发工作的所有事件。”start on”的语法很简单,如下所示:
start on EVENT {[}{[}KEY={]}VALUE{]}… {[}and\textbar{}or…{]}
EVENT 表示事件的名字,可以在 start on 中指定多个事件,表示该工作的开始需要依赖多个事件发生。多个事件之间可以用 and 或者 or 组合,”表示全部都必须发生”或者”其中之一发生即可”等不同的依赖条件。除了事件发生之外,工作的启动还可以依赖特定的条件,因此在 start on 的 EVENT 之后,可以用 KEY=VALUE 来表示额外的条件,一般是某个环境变量(KEY)和特定值(VALUE)进行比较。如果只有一个变量,或者变量的顺序已知,则 KEY 可以省略。
“stop on”和”start on”非常类似,只不过是定义工作在什么情况下需要停止。
代码清单 3 是”start on”和”stop on”的一个例子。
清单 3. start on/ stop on 例子

\#dbus.conf
description     “D-Bus system message bus”

start on local-filesystems
stop on deconfiguring-networking
…
D-Bus 是一个系统消息服务,上面的配置文件表明当系统发出 local-filesystems 事件时启动 D-Bus;当系统发出 deconfiguring-networking 事件时,停止 D-Bus 服务。
\begin{itemize}
\item {} 
Session Init

\end{itemize}

UpStart 还可以用于管理用户会话的初始化。在我写这篇文章的今天,多数 Linux 发行版还没有使用 UpStart 管理会话。只有在 Ubuntu Raring 版本中,使用 UpStart 管理用户会话的初始化过程。
首先让我们了解一下 Session 的概念。Session 就是一个用户会话,即用户从远程或者本地登入系统开始工作,直到用户退出。这整个过程就构成一个会话。
每个用户的使用习惯和使用方法都不相同,因此用户往往需要为自己的会话做一个定制,比如添加特定的命令别名,启动特殊的应用程序或者服务,等等。这些工作都属于对特定会话的初始化操作,因此可以被称为 Session Init。
用户使用 Linux 可以有两种模式:字符模式和图形界面。在字符模式下,会话初始化相对简单。用户登录后只能启动一个 Shell,通过 shell 命令使用系统。各种 shell 程序都支持一个自动运行的启动脚本,比如\textasciitilde{}/.bashrc。用户在这些脚本中加入需要运行的定制化命令。字符会话需求简单,因此这种现有的机制工作的很好。
在图形界面下,事情就变得复杂一些。用户登录后看到的并不是一个 shell 提示符,而是一个桌面。一个完整的桌面环境由很多组件组成。
一个桌面环境包括 window manager,panel 以及其它一些定义在/usr/share/gnome-session/sessions/下面的基本组件;此外还有一些辅助的应用程序,共同帮助构成一个完整的方便的桌面,比如 system monitors,panel applets,NetworkManager,Bluetooth,printers 等。当用户登录之后,这些组件都需要被初始化,这个过程比字符界面要复杂的多。目前启动各种图形组件和应用的工作由 gnome-session 完成。过程如下:
以 Ubuntu 为例,当用户登录 Ubuntu 图形界面后,显示管理器(Display Manager)lightDM 启动 Xsession。Xsession 接着启动 gnome-session,gnome-session 负责其它的初始化工作,然后就开始了一个 desktop session。
图 2.传统 desktop session 启动过程

\begin{sphinxVerbatim}[commandchars=\\\{\}]
init
 \textbar{}\PYGZhy{} lightdm
 \textbar{}   \textbar{}\PYGZhy{} Xorg
 \textbar{}   \textbar{}\PYGZhy{} lightdm \PYGZhy{}\PYGZhy{}\PYGZhy{}session\PYGZhy{}child
 \textbar{}        \textbar{}\PYGZhy{} gnome\PYGZhy{}session \PYGZhy{}\PYGZhy{}session=ubuntu
 \textbar{}             \textbar{}\PYGZhy{} compiz
 \textbar{}             \textbar{}\PYGZhy{} gwibber
 \textbar{}             \textbar{}\PYGZhy{} nautilus
 \textbar{}             \textbar{}\PYGZhy{} nm\PYGZhy{}applet
 \textbar{}             :
 \textbar{}             :
 \textbar{}
 \textbar{}\PYGZhy{} dbus\PYGZhy{}daemon \PYGZhy{}\PYGZhy{}session
 \textbar{}
 :
 :
这个过程有一些缺点(和 sysVInit 类似)。一些应用和组件其实并不需要在会话初始化过程中启动,更好的选择是在需要它们的时候才启动。比如 update\PYGZhy{}notifier 服务,该服务不停地监测几个文件系统路径,一旦这些路径上发现可以更新的软件包,就提醒用户。这些文件系统路径包括新插入的 DVD 盘等。Update\PYGZhy{}notifier 由 gnome\PYGZhy{}session 启动并一直运行着,在多数情况下,用户并不会插入新的 DVD,此时 update\PYGZhy{}notifier 服务一直在后台运行并消耗系统资源。更好的模式是当用户插入 DVD 的时候再运行 update\PYGZhy{}notifier。这样可以加快启动时间,减小系统运行过程中的内存等系统资源的开销。对于移动,嵌入式等设备等这还意味着省电。除了 Update\PYGZhy{}notifier 服务之外,还有其它一些类似的服务。比如 Network Manager,一天之内用户很少切换网络设备,所以大部分时间 Network Manager 服务仅仅是在浪费系统资源;再比如 backup manager 等其它常驻内存,后台不间断运行却很少真正被使用的服务。
用 UpStart 的基于事件的按需启动的模式就可以很好地解决这些问题,比如用户插入网线的时候才启动 Network Manager,因为用户插入网线表明需要使用网络,这可以被称为按需启动。
下图描述了采用 UpStart 之后的会话初始化过程。
图 3.采用 Upstart 的 Desktop session init 过程

init
 \textbar{}\PYGZhy{} lightdm
 \textbar{}   \textbar{}\PYGZhy{} Xorg
 \textbar{}   \textbar{}\PYGZhy{} lightdm \PYGZhy{}\PYGZhy{}\PYGZhy{}session\PYGZhy{}child
 \textbar{}        \textbar{}\PYGZhy{} session\PYGZhy{}init \PYGZsh{} \PYGZlt{}\PYGZhy{}\PYGZhy{} upstart running as normal user
 \textbar{}             \textbar{}\PYGZhy{} dbus\PYGZhy{}daemon \PYGZhy{}\PYGZhy{}session
 \textbar{}             \textbar{}\PYGZhy{} gnome\PYGZhy{}session \PYGZhy{}\PYGZhy{}session=ubuntu
 \textbar{}             \textbar{}\PYGZhy{} compiz
 \textbar{}             \textbar{}\PYGZhy{} gwibber
 \textbar{}             \textbar{}\PYGZhy{} nautilus
 \textbar{}             \textbar{}\PYGZhy{} nm\PYGZhy{}applet
 \textbar{}             :
 \textbar{}             :
 :
 :
\end{sphinxVerbatim}


\subsubsection{UpStart 使用}
\label{\detokenize{linux/point:id3}}
有两种人员需要了解 Upstart 的使用。第一类是系统开发人员,比如 MySQL 的开发人员。它们需要了解如何编写工作配置文件,以便用 UpStart 来管理服务。比如启动,停止 MySQL 服务。
另外一种情况是系统管理员,它们需要掌握 Upstart 的管理命令以便配置和管理系统的初始化,管理系统服务。
系统开发人员需要了解的 UpStart 知识

系统开发人员不仅需要掌握工作配置文件的写法,还需要了解一些针对服务进程编程上的要求。本文仅列出了少数工作配置文件的语法。要全面掌握工作配置文件的写法,需要详细阅读 Upstart 的手册。这里让我们来分析一下如何用 Upstart 来实现传统的运行级别,进而了解如何灵活使用工作配置文件。
Upstart 系统中的运行级别
Upstart 的运作完全是基于工作和事件的。工作的状态变化和运行会引起事件,进而触发其它工作和事件。
而传统的 Linux 系统初始化是基于运行级别的,即 SysVInit。因为历史的原因,Linux 上的多数软件还是采用传统的 SysVInit 脚本启动方式,并没有为 UpStart 开发新的启动脚本,因此即便在 Debian 和 Ubuntu 系统上,还是必须模拟老的 SysVInit 的运行级别模式,以便和多数现有软件兼容。
虽然 Upstart 本身并没有运行级别的概念,但完全可以用 UpStart 的工作模拟出来。让我们完整地考察一下 UpStart 机制下的系统启动过程。
系统启动过程
下图描述了 UpStart 的启动过程。

系统上电后运行 GRUB 载入内核。内核执行硬件初始化和内核自身初始化。在内核初始化的最后,内核将启动 pid 为 1 的 init 进程,即 UpStart 进程。
Upstart 进程在执行了一些自身的初始化工作后,立即发出”startup”事件。上图中用红色方框加红色箭头表示事件,可以在左上方看到”startup”事件。
所有依赖于”startup”事件的工作被触发,其中最重要的是 mountall。mountall 任务负责挂载系统中需要使用的文件系统,完成相应工作后,mountall 任务会发出以下事件:local-filesystem,virtual-filesystem,all-swaps,
其中 virtual-filesystem 事件触发 udev 任务开始工作。任务 udev 触发 upstart-udev-bridge 的工作。Upstart-udev-bridge 会发出 net-device-up IFACE=lo 事件,表示本地回环 IP 网络已经准备就绪。同时,任务 mountall 继续执行,最终会发出 filesystem 事件。
此时,任务 rc-sysinit 会被触发,因为 rc-sysinit 的 start on 条件如下:
start on filesystem and net-device-up IFACE=lo
任务 rc-sysinit 调用 telinit。Telinit 任务会发出 runlevel 事件,触发执行/etc/init/rc.conf。
rc.conf 执行/etc/rc\$.d/目录下的所有脚本,和 SysVInit 非常类似,读者可以参考本文第一部分的描述。

程序开发时需要注意的事项
作为程序开发人员,在编写系统服务时,需要了解 UpStart 的一些特殊要求。只有符合这些要求的软件才可以被 UpStart 管理。
规则一,派生次数需声明。
很多 Linux 后台服务都通过派生两次的技巧将自己变成后台服务程序。如果您编写的服务也采用了这个技术,就必须通过文档或其它的某种方式明确地让 UpStart 的维护人员知道这一点,这将影响 UpStart 的 expect stanza,我们在前面已经详细介绍过这个 stanza 的含义。
规则二,派生后即可用。
后台程序在完成第二次派生的时候,必须保证服务已经可用。因为 UpStart 通过派生计数来决定服务是否处于就绪状态。
规则三,遵守 SIGHUP 的要求。
UpStart 会给精灵进程发送 SIGHUP 信号,此时,UpStart 希望该精灵进程做以下这些响应工作:
•完成所有必要的重新初始化工作,比如重新读取配置文件。这是因为 UpStart 的命令”initctl reload”被设计为可以让服务在不重启的情况下更新配置。
•精灵进程必须继续使用现有的 PID,即收到 SIGHUP 时不能调用 fork。如果服务必须在这里调用 fork,则等同于派生两次,参考上面的规则一的处理。这个规则保证了 UpStart 可以继续使用 PID 管理本服务。
规则四,收到 SIGTEM 即 shutdown。
•当收到 SIGTERM 信号后,UpStart 希望精灵进程进程立即干净地退出,释放所有资源。如果一个进程在收到 SIGTERM 信号后不退出,Upstart 将对其发送 SIGKILL 信号。
\begin{itemize}
\item {} 
系统管理员需要了解的 Upstart 命令

\end{itemize}

作为系统管理员,一个重要的职责就是管理系统服务。比如系统服务的监控,启动,停止和配置。UpStart 提供了一系列的命令来完成这些工作。其中的核心是initctl,这是一个带子命令风格的命令行工具。
比如可以用 initctl list 来查看所有工作的概况:
\$initctl list
alsa-mixer-save stop/waiting
avahi-daemon start/running, process 690
mountall-net stop/waiting
rc stop/waiting
rsyslog start/running, process 482
screen-cleanup stop/waiting
tty4 start/running, process 859
udev start/running, process 334
upstart-udev-bridge start/running, process 304
ureadahead-other stop/waiting
这是在 Ubuntu10.10 系统上的输出,其它的 Linux 发行版上的输出会有所不同。第一列是工作名,比如 rsyslog。第二列是工作的目标;第三列是工作的状态。
此外还可以用 initctl stop 停止一个正在运行的工作;用 initctl start 开始一个工作;还可以用 initctl status 来查看一个工作的状态;initctl restart 重启一个工作;initctl reload 可以让一个正在运行的服务重新载入配置文件。这些命令和传统的 service 命令十分相似。
表 2.service 命令和 initctl 命令对照表

Service 命令  UpStart initctl 命令
service start   initctl start
service stop    initctl stop
service restart initctl restart
service reload  initctl reload
很多情况下管理员并不喜欢子命令风格,因为需要手动键入的字符太多。UpStart 还提供了一些快捷命令来简化 initctl,实际上这些命令只是在内部调用相应的 initctl 命令。比如 reload,restart,start,stop 等等。启动一个服务可以简单地调用
start \textless{}job\textgreater{}
这和执行 initctl start \textless{}job\textgreater{}是一样的效果。
一些命令是为了兼容其它系统(主要是 sysvinit),比如显示 runlevel 用/sbin/runlevel 命令:
\$runlevel
N 2
这个输出说明当前系统的运行级别为 2。而且系统没有之前的运行级别,也就是说在系统上电启动进入预定运行级别之后没有再修改过运行级别。
那么如何修改系统上电之后的默认运行级别呢?
在 Upstart 系统中,需要修改/etc/init/rc-sysinti.conf 中的 DEFAULT\_RUNLEVEL 这个参数,以便修改默认启动运行级别。这一点和 sysvinit 的习惯有所不同,大家需要格外留意。
还有一些随 UpStart 发布的小工具,用来帮助开发 UpStart 或者诊断 UpStart 的问题。比如 init-checkconf 和 upstart-monitor
还可以使用 initctl 的 emit 命令从命令行发送一个事件。
\#initctl emit \textless{}event\textgreater{}
这一般是用于 UpStart 本身的排错。


\subsubsection{Upstart 小结}
\label{\detokenize{linux/point:id4}}
可以看到,UpStart 的设计比 SysVInit 更加先进。多数 Linux 发行版上已经不再使用 SysVInit,一部分发行版采用了 UpStart,比如 Ubuntu;而另外一些比如 Fedora,采用了一种被称为 systemd 的 init 系统。Systemd 出现的比 UpStart 更晚,但发展迅速,虽然 UpStart 也还在积极开发并被越来越多地应用,但 systemd 似乎发展更快,我将在下一篇文章中再介绍 systemd。


\subsection{defconf-utils}
\label{\detokenize{linux/point:defconf-utils}}\begin{itemize}
\item {} 
install

\begin{sphinxVerbatim}[commandchars=\\\{\}]
\PYG{n}{apt} \PYG{n}{install} \PYG{n}{debconf}\PYG{o}{\PYGZhy{}}\PYG{n}{utils}
\end{sphinxVerbatim}

\end{itemize}


\subsection{network}
\label{\detokenize{linux/point:network}}

\subsubsection{配置固定ip}
\label{\detokenize{linux/point:ip}}
\# vi /etc/network/interfaces
\# 配置

\begin{sphinxVerbatim}[commandchars=\\\{\}]
\PYG{c+c1}{\PYGZsh{} This file describes the network interfaces available on your system}
\PYG{c+c1}{\PYGZsh{} and how to activate them. For more information, see interfaces(5).}
\PYG{n}{source} \PYG{o}{/}\PYG{n}{etc}\PYG{o}{/}\PYG{n}{network}\PYG{o}{/}\PYG{n}{interfaces}\PYG{o}{.}\PYG{n}{d}\PYG{o}{/}\PYG{o}{*}
\PYG{c+c1}{\PYGZsh{} The loopback network interface}
\PYG{n}{auto} \PYG{n}{lo}
\PYG{n}{iface} \PYG{n}{lo} \PYG{n}{inet} \PYG{n}{loopback}
\PYG{c+c1}{\PYGZsh{} The primary network interface}
\PYG{n}{auto} \PYG{n}{enp0s3}  \PYG{c+c1}{\PYGZsh{} 网卡端口}
\PYG{c+c1}{\PYGZsh{}iface enp0s3 inet dhcp  \PYGZsh{} dhcp自动获取ip}
\PYG{n}{iface} \PYG{n}{enp0s3}  \PYG{n}{inet} \PYG{n}{static}  \PYG{c+c1}{\PYGZsh{} 静态ip}
\PYG{n}{address} \PYG{l+m+mf}{10.0}\PYG{o}{.}\PYG{l+m+mf}{0.42}          \PYG{c+c1}{\PYGZsh{} ip地址}
\PYG{n}{netmask} \PYG{l+m+mf}{255.255}\PYG{o}{.}\PYG{l+m+mf}{255.0}      \PYG{c+c1}{\PYGZsh{} 掩码}
\PYG{n}{gateway} \PYG{l+m+mf}{10.0}\PYG{o}{.}\PYG{l+m+mf}{0.254}                  \PYG{c+c1}{\PYGZsh{} 网关}
\PYG{n}{dns}\PYG{o}{\PYGZhy{}}\PYG{n}{nameservers} \PYG{l+m+mf}{202.106}\PYG{o}{.}\PYG{l+m+mf}{0.20} \PYG{l+m+mf}{8.8}\PYG{o}{.}\PYG{l+m+mf}{8.8} \PYG{c+c1}{\PYGZsh{} dns服务器地址}
\end{sphinxVerbatim}

\# 重启network

\begin{sphinxVerbatim}[commandchars=\\\{\}]
\PYG{o}{/}\PYG{n}{etc}\PYG{o}{/}\PYG{n}{init}\PYG{o}{.}\PYG{n}{d}\PYG{o}{/}\PYG{n}{networking} \PYG{n}{restart}
\end{sphinxVerbatim}


\section{apt}
\label{\detokenize{linux/apt::doc}}\label{\detokenize{linux/apt:apt}}

\subsection{refer}
\label{\detokenize{linux/apt:refer}}
none refer


\subsection{command example}
\label{\detokenize{linux/apt:command-example}}

\subsubsection{base method}
\label{\detokenize{linux/apt:base-method}}\begin{itemize}
\item {} 
install package

\begin{sphinxVerbatim}[commandchars=\\\{\}]
\PYGZdl{} apt install package\PYGZus{}name  \PYGZhy{}y
\end{sphinxVerbatim}

\item {} 
remove package

\begin{sphinxVerbatim}[commandchars=\\\{\}]
\PYGZdl{} apt autoremove \PYGZhy{}\PYGZhy{}purge package\PYGZus{}name \PYGZhy{}y
\PYGZdl{} apt remove \PYGZhy{}\PYGZhy{}purge package\PYGZus{}name \PYGZhy{}y
\end{sphinxVerbatim}

\item {} 
check package

\begin{sphinxVerbatim}[commandchars=\\\{\}]
\PYGZdl{} dpkg \PYGZhy{}l \textbar{} grep package\PYGZus{}name
\end{sphinxVerbatim}

\end{itemize}


\subsubsection{install\&remove Nginx}
\label{\detokenize{linux/apt:install-remove-nginx}}\begin{quote}
\begin{itemize}
\item {} 
install nginx

\begin{sphinxVerbatim}[commandchars=\\\{\}]
\PYG{n}{apt} \PYG{n}{install} \PYG{n}{nginx} \PYG{o}{\PYGZhy{}}\PYG{n}{y}
\PYG{n}{find} \PYG{o}{/}\PYG{n}{etc} \PYG{o}{\PYGZhy{}}\PYG{n}{name} \PYG{l+s+s2}{\PYGZdq{}}\PYG{l+s+s2}{*nginx*}\PYG{l+s+s2}{\PYGZdq{}}
    \PYG{o}{/}\PYG{n}{etc}\PYG{o}{/}\PYG{n}{logrotate}\PYG{o}{.}\PYG{n}{d}\PYG{o}{/}\PYG{n}{nginx}
    \PYG{o}{/}\PYG{n}{etc}\PYG{o}{/}\PYG{n}{rc1}\PYG{o}{.}\PYG{n}{d}\PYG{o}{/}\PYG{n}{K01nginx}
    \PYG{o}{/}\PYG{n}{etc}\PYG{o}{/}\PYG{n}{rc5}\PYG{o}{.}\PYG{n}{d}\PYG{o}{/}\PYG{n}{S02nginx}
    \PYG{o}{/}\PYG{n}{etc}\PYG{o}{/}\PYG{n}{nginx}
    \PYG{o}{/}\PYG{n}{etc}\PYG{o}{/}\PYG{n}{nginx}\PYG{o}{/}\PYG{n}{nginx}\PYG{o}{.}\PYG{n}{conf}
    \PYG{o}{/}\PYG{n}{etc}\PYG{o}{/}\PYG{n}{init}\PYG{o}{.}\PYG{n}{d}\PYG{o}{/}\PYG{n}{nginx}
    \PYG{o}{/}\PYG{n}{etc}\PYG{o}{/}\PYG{n}{rc3}\PYG{o}{.}\PYG{n}{d}\PYG{o}{/}\PYG{n}{S02nginx}
    \PYG{o}{/}\PYG{n}{etc}\PYG{o}{/}\PYG{n}{ufw}\PYG{o}{/}\PYG{n}{applications}\PYG{o}{.}\PYG{n}{d}\PYG{o}{/}\PYG{n}{nginx}
    \PYG{o}{/}\PYG{n}{etc}\PYG{o}{/}\PYG{n}{rc2}\PYG{o}{.}\PYG{n}{d}\PYG{o}{/}\PYG{n}{S02nginx}
    \PYG{o}{/}\PYG{n}{etc}\PYG{o}{/}\PYG{n}{rc0}\PYG{o}{.}\PYG{n}{d}\PYG{o}{/}\PYG{n}{K01nginx}
    \PYG{o}{/}\PYG{n}{etc}\PYG{o}{/}\PYG{n}{init}\PYG{o}{/}\PYG{n}{nginx}\PYG{o}{.}\PYG{n}{conf}
    \PYG{o}{/}\PYG{n}{etc}\PYG{o}{/}\PYG{n}{default}\PYG{o}{/}\PYG{n}{nginx}
    \PYG{o}{/}\PYG{n}{etc}\PYG{o}{/}\PYG{n}{systemd}\PYG{o}{/}\PYG{n}{system}\PYG{o}{/}\PYG{n}{multi}\PYG{o}{\PYGZhy{}}\PYG{n}{user}\PYG{o}{.}\PYG{n}{target}\PYG{o}{.}\PYG{n}{wants}\PYG{o}{/}\PYG{n}{nginx}\PYG{o}{.}\PYG{n}{service}
    \PYG{o}{/}\PYG{n}{etc}\PYG{o}{/}\PYG{n}{rc4}\PYG{o}{.}\PYG{n}{d}\PYG{o}{/}\PYG{n}{S02nginx}
    \PYG{o}{/}\PYG{n}{etc}\PYG{o}{/}\PYG{n}{rc6}\PYG{o}{.}\PYG{n}{d}\PYG{o}{/}\PYG{n}{K01nginx}
\PYG{n}{find} \PYG{o}{/}\PYG{n}{usr} \PYG{o}{\PYGZhy{}}\PYG{n}{name} \PYG{l+s+s2}{\PYGZdq{}}\PYG{l+s+s2}{*nginx*}\PYG{l+s+s2}{\PYGZdq{}}
    \PYG{o}{/}\PYG{n}{usr}\PYG{o}{/}\PYG{n}{sbin}\PYG{o}{/}\PYG{n}{nginx}
    \PYG{o}{/}\PYG{n}{usr}\PYG{o}{/}\PYG{n}{share}\PYG{o}{/}\PYG{n}{doc}\PYG{o}{/}\PYG{n}{nginx}
    \PYG{o}{/}\PYG{n}{usr}\PYG{o}{/}\PYG{n}{share}\PYG{o}{/}\PYG{n}{doc}\PYG{o}{/}\PYG{n}{nginx}\PYG{o}{\PYGZhy{}}\PYG{n}{common}
    \PYG{o}{/}\PYG{n}{usr}\PYG{o}{/}\PYG{n}{share}\PYG{o}{/}\PYG{n}{doc}\PYG{o}{/}\PYG{n}{nginx}\PYG{o}{\PYGZhy{}}\PYG{n}{core}
    \PYG{o}{/}\PYG{n}{usr}\PYG{o}{/}\PYG{n}{share}\PYG{o}{/}\PYG{n}{nginx}
    \PYG{o}{/}\PYG{n}{usr}\PYG{o}{/}\PYG{n}{share}\PYG{o}{/}\PYG{n}{vim}\PYG{o}{/}\PYG{n}{registry}\PYG{o}{/}\PYG{n}{nginx}\PYG{o}{.}\PYG{n}{yaml}
    \PYG{o}{/}\PYG{n}{usr}\PYG{o}{/}\PYG{n}{share}\PYG{o}{/}\PYG{n}{vim}\PYG{o}{/}\PYG{n}{addons}\PYG{o}{/}\PYG{n}{indent}\PYG{o}{/}\PYG{n}{nginx}\PYG{o}{.}\PYG{n}{vim}
    \PYG{o}{/}\PYG{n}{usr}\PYG{o}{/}\PYG{n}{share}\PYG{o}{/}\PYG{n}{vim}\PYG{o}{/}\PYG{n}{addons}\PYG{o}{/}\PYG{n}{syntax}\PYG{o}{/}\PYG{n}{nginx}\PYG{o}{.}\PYG{n}{vim}
    \PYG{o}{/}\PYG{n}{usr}\PYG{o}{/}\PYG{n}{share}\PYG{o}{/}\PYG{n}{vim}\PYG{o}{/}\PYG{n}{addons}\PYG{o}{/}\PYG{n}{ftdetect}\PYG{o}{/}\PYG{n}{nginx}\PYG{o}{.}\PYG{n}{vim}
    \PYG{o}{/}\PYG{n}{usr}\PYG{o}{/}\PYG{n}{share}\PYG{o}{/}\PYG{n}{lintian}\PYG{o}{/}\PYG{n}{overrides}\PYG{o}{/}\PYG{n}{nginx}\PYG{o}{\PYGZhy{}}\PYG{n}{common}
    \PYG{o}{/}\PYG{n}{usr}\PYG{o}{/}\PYG{n}{share}\PYG{o}{/}\PYG{n}{lintian}\PYG{o}{/}\PYG{n}{overrides}\PYG{o}{/}\PYG{n}{nginx}\PYG{o}{\PYGZhy{}}\PYG{n}{core}
    \PYG{o}{/}\PYG{n}{usr}\PYG{o}{/}\PYG{n}{share}\PYG{o}{/}\PYG{n}{apport}\PYG{o}{/}\PYG{n}{package}\PYG{o}{\PYGZhy{}}\PYG{n}{hooks}\PYG{o}{/}\PYG{n}{source\PYGZus{}nginx}\PYG{o}{.}\PYG{n}{py}
\PYG{n}{find} \PYG{o}{/}\PYG{n}{lib} \PYG{o}{\PYGZhy{}}\PYG{n}{name} \PYG{l+s+s2}{\PYGZdq{}}\PYG{l+s+s2}{*nginx*}\PYG{l+s+s2}{\PYGZdq{}}
    \PYG{o}{/}\PYG{n}{lib}\PYG{o}{/}\PYG{n}{systemd}\PYG{o}{/}\PYG{n}{system}\PYG{o}{/}\PYG{n}{nginx}\PYG{o}{.}\PYG{n}{service}
\PYG{n}{dpkg} \PYG{o}{\PYGZhy{}}\PYG{n}{l} \PYG{o}{\textbar{}} \PYG{n}{grep} \PYG{n}{nginx}
    \PYG{n}{ii}  \PYG{n}{nginx}           \PYG{l+m+mf}{1.10}\PYG{o}{.}\PYG{l+m+mi}{3}\PYG{o}{\PYGZhy{}}\PYG{l+m+mi}{0}\PYG{n}{ubuntu0}\PYG{o}{.}\PYG{l+m+mf}{16.04}\PYG{o}{.}\PYG{l+m+mi}{2}     \PYG{n+nb}{all}          \PYG{n}{small}\PYG{p}{,} \PYG{n}{powerful}\PYG{p}{,} \PYG{n}{scalable} \PYG{n}{web}\PYG{o}{/}\PYG{n}{proxy} \PYG{n}{server}
    \PYG{n}{ii}  \PYG{n}{nginx}\PYG{o}{\PYGZhy{}}\PYG{n}{common}    \PYG{l+m+mf}{1.10}\PYG{o}{.}\PYG{l+m+mi}{3}\PYG{o}{\PYGZhy{}}\PYG{l+m+mi}{0}\PYG{n}{ubuntu0}\PYG{o}{.}\PYG{l+m+mf}{16.04}\PYG{o}{.}\PYG{l+m+mi}{2}     \PYG{n+nb}{all}          \PYG{n}{small}\PYG{p}{,} \PYG{n}{powerful}\PYG{p}{,} \PYG{n}{scalable} \PYG{n}{web}\PYG{o}{/}\PYG{n}{proxy} \PYG{n}{server} \PYG{o}{\PYGZhy{}} \PYG{n}{common} \PYG{n}{files}
    \PYG{n}{ii}  \PYG{n}{nginx}\PYG{o}{\PYGZhy{}}\PYG{n}{core}      \PYG{l+m+mf}{1.10}\PYG{o}{.}\PYG{l+m+mi}{3}\PYG{o}{\PYGZhy{}}\PYG{l+m+mi}{0}\PYG{n}{ubuntu0}\PYG{o}{.}\PYG{l+m+mf}{16.04}\PYG{o}{.}\PYG{l+m+mi}{2}     \PYG{n}{amd64}        \PYG{n}{nginx} \PYG{n}{web}\PYG{o}{/}\PYG{n}{proxy} \PYG{n}{server} \PYG{p}{(}\PYG{n}{core} \PYG{n}{version}\PYG{p}{)}
\end{sphinxVerbatim}

\item {} 
remove nginx

\begin{sphinxVerbatim}[commandchars=\\\{\}]
\PYG{n}{apt} \PYG{n}{autoremove} \PYG{o}{\PYGZhy{}}\PYG{o}{\PYGZhy{}}\PYG{n}{purge} \PYG{n}{nginx}\PYG{o}{\PYGZhy{}}\PYG{n}{common} \PYG{o}{\PYGZhy{}}\PYG{n}{y} \PYG{n}{直接删除nginx}\PYG{o}{\PYGZhy{}}\PYG{n}{common}
\end{sphinxVerbatim}

\end{itemize}

会自动卸载依赖nginx及配置文件,执行以上命令都为空
\end{quote}


\subsubsection{install\&remove mysql}
\label{\detokenize{linux/apt:install-remove-mysql}}\begin{itemize}
\item {} 
install mysql

\begin{sphinxVerbatim}[commandchars=\\\{\}]
\PYG{n}{apt} \PYG{n}{install} \PYG{n}{mysql}\PYG{o}{\PYGZhy{}}\PYG{n}{server} \PYG{o}{\PYGZhy{}}\PYG{n}{y}
\PYG{n}{find} \PYG{o}{/}\PYG{n}{etc} \PYG{o}{\PYGZhy{}}\PYG{n}{name} \PYG{l+s+s2}{\PYGZdq{}}\PYG{l+s+s2}{*mysql*}\PYG{l+s+s2}{\PYGZdq{}}
    \PYG{o}{/}\PYG{n}{etc}\PYG{o}{/}\PYG{n}{logrotate}\PYG{o}{.}\PYG{n}{d}\PYG{o}{/}\PYG{n}{mysql}\PYG{o}{\PYGZhy{}}\PYG{n}{server}
    \PYG{o}{/}\PYG{n}{etc}\PYG{o}{/}\PYG{n}{logcheck}\PYG{o}{/}\PYG{n}{ignore}\PYG{o}{.}\PYG{n}{d}\PYG{o}{.}\PYG{n}{server}\PYG{o}{/}\PYG{n}{mysql}\PYG{o}{\PYGZhy{}}\PYG{n}{server}\PYG{o}{\PYGZhy{}}\PYG{l+m+mi}{5}\PYG{n}{\PYGZus{}7}
    \PYG{o}{/}\PYG{n}{etc}\PYG{o}{/}\PYG{n}{logcheck}\PYG{o}{/}\PYG{n}{ignore}\PYG{o}{.}\PYG{n}{d}\PYG{o}{.}\PYG{n}{workstation}\PYG{o}{/}\PYG{n}{mysql}\PYG{o}{\PYGZhy{}}\PYG{n}{server}\PYG{o}{\PYGZhy{}}\PYG{l+m+mi}{5}\PYG{n}{\PYGZus{}7}
    \PYG{o}{/}\PYG{n}{etc}\PYG{o}{/}\PYG{n}{logcheck}\PYG{o}{/}\PYG{n}{ignore}\PYG{o}{.}\PYG{n}{d}\PYG{o}{.}\PYG{n}{paranoid}\PYG{o}{/}\PYG{n}{mysql}\PYG{o}{\PYGZhy{}}\PYG{n}{server}\PYG{o}{\PYGZhy{}}\PYG{l+m+mi}{5}\PYG{n}{\PYGZus{}7}
    \PYG{o}{/}\PYG{n}{etc}\PYG{o}{/}\PYG{n}{rc1}\PYG{o}{.}\PYG{n}{d}\PYG{o}{/}\PYG{n}{K02mysql}
    \PYG{o}{/}\PYG{n}{etc}\PYG{o}{/}\PYG{n}{rc5}\PYG{o}{.}\PYG{n}{d}\PYG{o}{/}\PYG{n}{S02mysql}
    \PYG{o}{/}\PYG{n}{etc}\PYG{o}{/}\PYG{n}{init}\PYG{o}{.}\PYG{n}{d}\PYG{o}{/}\PYG{n}{mysql}
    \PYG{o}{/}\PYG{n}{etc}\PYG{o}{/}\PYG{n}{rc3}\PYG{o}{.}\PYG{n}{d}\PYG{o}{/}\PYG{n}{S02mysql}
    \PYG{o}{/}\PYG{n}{etc}\PYG{o}{/}\PYG{n}{rc2}\PYG{o}{.}\PYG{n}{d}\PYG{o}{/}\PYG{n}{S02mysql}
    \PYG{o}{/}\PYG{n}{etc}\PYG{o}{/}\PYG{n}{rc0}\PYG{o}{.}\PYG{n}{d}\PYG{o}{/}\PYG{n}{K02mysql}
    \PYG{o}{/}\PYG{n}{etc}\PYG{o}{/}\PYG{n}{init}\PYG{o}{/}\PYG{n}{mysql}\PYG{o}{.}\PYG{n}{conf}
    \PYG{o}{/}\PYG{n}{etc}\PYG{o}{/}\PYG{n}{apparmor}\PYG{o}{.}\PYG{n}{d}\PYG{o}{/}\PYG{n}{usr}\PYG{o}{.}\PYG{n}{sbin}\PYG{o}{.}\PYG{n}{mysqld}
    \PYG{o}{/}\PYG{n}{etc}\PYG{o}{/}\PYG{n}{apparmor}\PYG{o}{.}\PYG{n}{d}\PYG{o}{/}\PYG{n}{local}\PYG{o}{/}\PYG{n}{usr}\PYG{o}{.}\PYG{n}{sbin}\PYG{o}{.}\PYG{n}{mysqld}
    \PYG{o}{/}\PYG{n}{etc}\PYG{o}{/}\PYG{n}{systemd}\PYG{o}{/}\PYG{n}{system}\PYG{o}{/}\PYG{n}{multi}\PYG{o}{\PYGZhy{}}\PYG{n}{user}\PYG{o}{.}\PYG{n}{target}\PYG{o}{.}\PYG{n}{wants}\PYG{o}{/}\PYG{n}{mysql}\PYG{o}{.}\PYG{n}{service}
    \PYG{o}{/}\PYG{n}{etc}\PYG{o}{/}\PYG{n}{mysql}
    \PYG{o}{/}\PYG{n}{etc}\PYG{o}{/}\PYG{n}{mysql}\PYG{o}{/}\PYG{n}{conf}\PYG{o}{.}\PYG{n}{d}\PYG{o}{/}\PYG{n}{mysql}\PYG{o}{.}\PYG{n}{cnf}
    \PYG{o}{/}\PYG{n}{etc}\PYG{o}{/}\PYG{n}{mysql}\PYG{o}{/}\PYG{n}{conf}\PYG{o}{.}\PYG{n}{d}\PYG{o}{/}\PYG{n}{mysqldump}\PYG{o}{.}\PYG{n}{cnf}
    \PYG{o}{/}\PYG{n}{etc}\PYG{o}{/}\PYG{n}{mysql}\PYG{o}{/}\PYG{n}{mysql}\PYG{o}{.}\PYG{n}{cnf}
    \PYG{o}{/}\PYG{n}{etc}\PYG{o}{/}\PYG{n}{mysql}\PYG{o}{/}\PYG{n}{mysql}\PYG{o}{.}\PYG{n}{conf}\PYG{o}{.}\PYG{n}{d}
    \PYG{o}{/}\PYG{n}{etc}\PYG{o}{/}\PYG{n}{mysql}\PYG{o}{/}\PYG{n}{mysql}\PYG{o}{.}\PYG{n}{conf}\PYG{o}{.}\PYG{n}{d}\PYG{o}{/}\PYG{n}{mysqld\PYGZus{}safe\PYGZus{}syslog}\PYG{o}{.}\PYG{n}{cnf}
    \PYG{o}{/}\PYG{n}{etc}\PYG{o}{/}\PYG{n}{mysql}\PYG{o}{/}\PYG{n}{mysql}\PYG{o}{.}\PYG{n}{conf}\PYG{o}{.}\PYG{n}{d}\PYG{o}{/}\PYG{n}{mysqld}\PYG{o}{.}\PYG{n}{cnf}
    \PYG{o}{/}\PYG{n}{etc}\PYG{o}{/}\PYG{n}{rc4}\PYG{o}{.}\PYG{n}{d}\PYG{o}{/}\PYG{n}{S02mysql}
    \PYG{o}{/}\PYG{n}{etc}\PYG{o}{/}\PYG{n}{rc6}\PYG{o}{.}\PYG{n}{d}\PYG{o}{/}\PYG{n}{K02mysql}
\PYG{n}{find} \PYG{o}{/}\PYG{n}{usr} \PYG{o}{\PYGZhy{}}\PYG{n}{name} \PYG{l+s+s2}{\PYGZdq{}}\PYG{l+s+s2}{*mysql*}\PYG{l+s+s2}{\PYGZdq{}}
    \PYG{o}{/}\PYG{n}{usr}\PYG{o}{/}\PYG{n}{lib}\PYG{o}{/}\PYG{n}{mysql}
    \PYG{o}{/}\PYG{n}{usr}\PYG{o}{/}\PYG{n}{lib}\PYG{o}{/}\PYG{n}{mysql}\PYG{o}{/}\PYG{n}{plugin}\PYG{o}{/}\PYG{n}{mysql\PYGZus{}no\PYGZus{}login}\PYG{o}{.}\PYG{n}{so}
    \PYG{o}{/}\PYG{n}{usr}\PYG{o}{/}\PYG{n}{sbin}\PYG{o}{/}\PYG{n}{mysqld}
    \PYG{o}{/}\PYG{n}{usr}\PYG{o}{/}\PYG{n+nb}{bin}\PYG{o}{/}\PYG{n}{mysql\PYGZus{}upgrade}
    \PYG{o}{/}\PYG{n}{usr}\PYG{o}{/}\PYG{n+nb}{bin}\PYG{o}{/}\PYG{n}{mysql\PYGZus{}embedded}
    \PYG{o}{/}\PYG{n}{usr}\PYG{o}{/}\PYG{n+nb}{bin}\PYG{o}{/}\PYG{n}{mysqlpump}
    \PYG{o}{/}\PYG{n}{usr}\PYG{o}{/}\PYG{n+nb}{bin}\PYG{o}{/}\PYG{n}{mysqldumpslow}
    \PYG{o}{/}\PYG{n}{usr}\PYG{o}{/}\PYG{n+nb}{bin}\PYG{o}{/}\PYG{n}{mysql\PYGZus{}secure\PYGZus{}installation}
    \PYG{o}{/}\PYG{n}{usr}\PYG{o}{/}\PYG{n+nb}{bin}\PYG{o}{/}\PYG{n}{mysqladmin}
    \PYG{o}{/}\PYG{n}{usr}\PYG{o}{/}\PYG{n+nb}{bin}\PYG{o}{/}\PYG{n}{mysql\PYGZus{}config\PYGZus{}editor}
    \PYG{o}{/}\PYG{n}{usr}\PYG{o}{/}\PYG{n+nb}{bin}\PYG{o}{/}\PYG{n}{mysqlcheck}
    \PYG{o}{/}\PYG{n}{usr}\PYG{o}{/}\PYG{n+nb}{bin}\PYG{o}{/}\PYG{n}{mysqld\PYGZus{}multi}
    \PYG{o}{/}\PYG{n}{usr}\PYG{o}{/}\PYG{n+nb}{bin}\PYG{o}{/}\PYG{n}{mysqldump}
    \PYG{o}{/}\PYG{n}{usr}\PYG{o}{/}\PYG{n+nb}{bin}\PYG{o}{/}\PYG{n}{mysqloptimize}
    \PYG{o}{/}\PYG{n}{usr}\PYG{o}{/}\PYG{n+nb}{bin}\PYG{o}{/}\PYG{n}{mysql\PYGZus{}tzinfo\PYGZus{}to\PYGZus{}sql}
    \PYG{o}{/}\PYG{n}{usr}\PYG{o}{/}\PYG{n+nb}{bin}\PYG{o}{/}\PYG{n}{mysqlshow}
    \PYG{o}{/}\PYG{n}{usr}\PYG{o}{/}\PYG{n+nb}{bin}\PYG{o}{/}\PYG{n}{mysqlimport}
    \PYG{o}{/}\PYG{n}{usr}\PYG{o}{/}\PYG{n+nb}{bin}\PYG{o}{/}\PYG{n}{mysqlrepair}
    \PYG{o}{/}\PYG{n}{usr}\PYG{o}{/}\PYG{n+nb}{bin}\PYG{o}{/}\PYG{n}{mysql\PYGZus{}plugin}
    \PYG{o}{/}\PYG{n}{usr}\PYG{o}{/}\PYG{n+nb}{bin}\PYG{o}{/}\PYG{n}{mysqlslap}
    \PYG{o}{/}\PYG{n}{usr}\PYG{o}{/}\PYG{n+nb}{bin}\PYG{o}{/}\PYG{n}{mysqlreport}
    \PYG{o}{/}\PYG{n}{usr}\PYG{o}{/}\PYG{n+nb}{bin}\PYG{o}{/}\PYG{n}{mysqld\PYGZus{}safe}
    \PYG{o}{/}\PYG{n}{usr}\PYG{o}{/}\PYG{n+nb}{bin}\PYG{o}{/}\PYG{n}{mysql}
    \PYG{o}{/}\PYG{n}{usr}\PYG{o}{/}\PYG{n+nb}{bin}\PYG{o}{/}\PYG{n}{mysqlanalyze}
    \PYG{o}{/}\PYG{n}{usr}\PYG{o}{/}\PYG{n+nb}{bin}\PYG{o}{/}\PYG{n}{mysqlbinlog}
    \PYG{o}{/}\PYG{n}{usr}\PYG{o}{/}\PYG{n+nb}{bin}\PYG{o}{/}\PYG{n}{mysql\PYGZus{}install\PYGZus{}db}
    \PYG{o}{/}\PYG{n}{usr}\PYG{o}{/}\PYG{n+nb}{bin}\PYG{o}{/}\PYG{n}{mysql\PYGZus{}ssl\PYGZus{}rsa\PYGZus{}setup}
    \PYG{o}{/}\PYG{n}{usr}\PYG{o}{/}\PYG{n}{share}\PYG{o}{/}\PYG{n}{doc}\PYG{o}{/}\PYG{n}{mysql}\PYG{o}{\PYGZhy{}}\PYG{n}{server}\PYG{o}{\PYGZhy{}}\PYG{l+m+mf}{5.7}
    \PYG{o}{/}\PYG{n}{usr}\PYG{o}{/}\PYG{n}{share}\PYG{o}{/}\PYG{n}{doc}\PYG{o}{/}\PYG{n}{mysql}\PYG{o}{\PYGZhy{}}\PYG{n}{server}\PYG{o}{\PYGZhy{}}\PYG{l+m+mf}{5.7}\PYG{o}{/}\PYG{n}{mysqld}\PYG{o}{.}\PYG{n}{sym}\PYG{o}{.}\PYG{n}{gz}
    \PYG{o}{/}\PYG{n}{usr}\PYG{o}{/}\PYG{n}{share}\PYG{o}{/}\PYG{n}{doc}\PYG{o}{/}\PYG{n}{mysql}\PYG{o}{\PYGZhy{}}\PYG{n}{server}
    \PYG{o}{/}\PYG{n}{usr}\PYG{o}{/}\PYG{n}{share}\PYG{o}{/}\PYG{n}{doc}\PYG{o}{/}\PYG{n}{mysql}\PYG{o}{\PYGZhy{}}\PYG{n}{server}\PYG{o}{\PYGZhy{}}\PYG{n}{core}\PYG{o}{\PYGZhy{}}\PYG{l+m+mf}{5.7}
    \PYG{o}{/}\PYG{n}{usr}\PYG{o}{/}\PYG{n}{share}\PYG{o}{/}\PYG{n}{doc}\PYG{o}{/}\PYG{n}{mysql}\PYG{o}{\PYGZhy{}}\PYG{n}{client}\PYG{o}{\PYGZhy{}}\PYG{l+m+mf}{5.7}
    \PYG{o}{/}\PYG{n}{usr}\PYG{o}{/}\PYG{n}{share}\PYG{o}{/}\PYG{n}{doc}\PYG{o}{/}\PYG{n}{kamailio}\PYG{o}{/}\PYG{n}{examples}\PYG{o}{/}\PYG{n}{icscf}\PYG{o}{/}\PYG{n}{icscf}\PYG{o}{.}\PYG{n}{mysql}\PYG{o}{.}\PYG{n}{sql}\PYG{o}{.}\PYG{n}{gz}
    \PYG{o}{/}\PYG{n}{usr}\PYG{o}{/}\PYG{n}{share}\PYG{o}{/}\PYG{n}{doc}\PYG{o}{/}\PYG{n}{kamailio}\PYG{o}{/}\PYG{n}{examples}\PYG{o}{/}\PYG{n}{kamailio}\PYG{o}{/}\PYG{n}{acc}\PYG{o}{\PYGZhy{}}\PYG{n}{mysql}\PYG{o}{.}\PYG{n}{cfg}\PYG{o}{.}\PYG{n}{gz}
    \PYG{o}{/}\PYG{n}{usr}\PYG{o}{/}\PYG{n}{share}\PYG{o}{/}\PYG{n}{doc}\PYG{o}{/}\PYG{n}{mysql}\PYG{o}{\PYGZhy{}}\PYG{n}{common}
    \PYG{o}{/}\PYG{n}{usr}\PYG{o}{/}\PYG{n}{share}\PYG{o}{/}\PYG{n}{doc}\PYG{o}{/}\PYG{n}{mysql}\PYG{o}{\PYGZhy{}}\PYG{n}{client}\PYG{o}{\PYGZhy{}}\PYG{n}{core}\PYG{o}{\PYGZhy{}}\PYG{l+m+mf}{5.7}
    \PYG{o}{/}\PYG{n}{usr}\PYG{o}{/}\PYG{n}{share}\PYG{o}{/}\PYG{n}{vim}\PYG{o}{/}\PYG{n}{vim74}\PYG{o}{/}\PYG{n}{syntax}\PYG{o}{/}\PYG{n}{mysql}\PYG{o}{.}\PYG{n}{vim}
    \PYG{o}{/}\PYG{n}{usr}\PYG{o}{/}\PYG{n}{share}\PYG{o}{/}\PYG{n}{mysql}
    \PYG{o}{/}\PYG{n}{usr}\PYG{o}{/}\PYG{n}{share}\PYG{o}{/}\PYG{n}{mysql}\PYG{o}{/}\PYG{n}{mysql}\PYG{o}{\PYGZhy{}}\PYG{n}{log}\PYG{o}{\PYGZhy{}}\PYG{n}{rotate}
    \PYG{o}{/}\PYG{n}{usr}\PYG{o}{/}\PYG{n}{share}\PYG{o}{/}\PYG{n}{mysql}\PYG{o}{/}\PYG{n}{mysql\PYGZus{}security\PYGZus{}commands}\PYG{o}{.}\PYG{n}{sql}
    \PYG{o}{/}\PYG{n}{usr}\PYG{o}{/}\PYG{n}{share}\PYG{o}{/}\PYG{n}{mysql}\PYG{o}{/}\PYG{n}{mysql\PYGZus{}sys\PYGZus{}schema}\PYG{o}{.}\PYG{n}{sql}
    \PYG{o}{/}\PYG{n}{usr}\PYG{o}{/}\PYG{n}{share}\PYG{o}{/}\PYG{n}{mysql}\PYG{o}{/}\PYG{n}{mysql\PYGZus{}test\PYGZus{}data\PYGZus{}timezone}\PYG{o}{.}\PYG{n}{sql}
    \PYG{o}{/}\PYG{n}{usr}\PYG{o}{/}\PYG{n}{share}\PYG{o}{/}\PYG{n}{mysql}\PYG{o}{/}\PYG{n}{mysql}\PYG{o}{\PYGZhy{}}\PYG{n}{systemd}\PYG{o}{\PYGZhy{}}\PYG{n}{start}
    \PYG{o}{/}\PYG{n}{usr}\PYG{o}{/}\PYG{n}{share}\PYG{o}{/}\PYG{n}{mysql}\PYG{o}{/}\PYG{n}{mysqld\PYGZus{}multi}\PYG{o}{.}\PYG{n}{server}
    \PYG{o}{/}\PYG{n}{usr}\PYG{o}{/}\PYG{n}{share}\PYG{o}{/}\PYG{n}{mysql}\PYG{o}{/}\PYG{n}{mysql\PYGZus{}system\PYGZus{}tables}\PYG{o}{.}\PYG{n}{sql}
    \PYG{o}{/}\PYG{n}{usr}\PYG{o}{/}\PYG{n}{share}\PYG{o}{/}\PYG{n}{mysql}\PYG{o}{/}\PYG{n}{mysql\PYGZus{}system\PYGZus{}tables\PYGZus{}data}\PYG{o}{.}\PYG{n}{sql}
    \PYG{o}{/}\PYG{n}{usr}\PYG{o}{/}\PYG{n}{share}\PYG{o}{/}\PYG{n}{lintian}\PYG{o}{/}\PYG{n}{overrides}\PYG{o}{/}\PYG{n}{mysql}\PYG{o}{\PYGZhy{}}\PYG{n}{server}\PYG{o}{\PYGZhy{}}\PYG{l+m+mf}{5.7}
    \PYG{o}{/}\PYG{n}{usr}\PYG{o}{/}\PYG{n}{share}\PYG{o}{/}\PYG{n}{lintian}\PYG{o}{/}\PYG{n}{overrides}\PYG{o}{/}\PYG{n}{mysql}\PYG{o}{\PYGZhy{}}\PYG{n}{client}\PYG{o}{\PYGZhy{}}\PYG{l+m+mf}{5.7}
    \PYG{o}{/}\PYG{n}{usr}\PYG{o}{/}\PYG{n}{share}\PYG{o}{/}\PYG{n}{lintian}\PYG{o}{/}\PYG{n}{overrides}\PYG{o}{/}\PYG{n}{mysql}\PYG{o}{\PYGZhy{}}\PYG{n}{common}
    \PYG{o}{/}\PYG{n}{usr}\PYG{o}{/}\PYG{n}{share}\PYG{o}{/}\PYG{n}{man}\PYG{o}{/}\PYG{n}{man1}\PYG{o}{/}\PYG{n}{mysqldumpslow}\PYG{o}{.}\PYG{l+m+mf}{1.}\PYG{n}{gz}
    \PYG{o}{/}\PYG{n}{usr}\PYG{o}{/}\PYG{n}{share}\PYG{o}{/}\PYG{n}{man}\PYG{o}{/}\PYG{n}{man1}\PYG{o}{/}\PYG{n}{mysqloptimize}\PYG{o}{.}\PYG{l+m+mf}{1.}\PYG{n}{gz}
    \PYG{o}{/}\PYG{n}{usr}\PYG{o}{/}\PYG{n}{share}\PYG{o}{/}\PYG{n}{man}\PYG{o}{/}\PYG{n}{man1}\PYG{o}{/}\PYG{n}{mysqlpump}\PYG{o}{.}\PYG{l+m+mf}{1.}\PYG{n}{gz}
    \PYG{o}{/}\PYG{n}{usr}\PYG{o}{/}\PYG{n}{share}\PYG{o}{/}\PYG{n}{man}\PYG{o}{/}\PYG{n}{man1}\PYG{o}{/}\PYG{n}{mysqlshow}\PYG{o}{.}\PYG{l+m+mf}{1.}\PYG{n}{gz}
    \PYG{o}{/}\PYG{n}{usr}\PYG{o}{/}\PYG{n}{share}\PYG{o}{/}\PYG{n}{man}\PYG{o}{/}\PYG{n}{man1}\PYG{o}{/}\PYG{n}{mysql\PYGZus{}upgrade}\PYG{o}{.}\PYG{l+m+mf}{1.}\PYG{n}{gz}
    \PYG{o}{/}\PYG{n}{usr}\PYG{o}{/}\PYG{n}{share}\PYG{o}{/}\PYG{n}{man}\PYG{o}{/}\PYG{n}{man1}\PYG{o}{/}\PYG{n}{mysql\PYGZus{}config\PYGZus{}editor}\PYG{o}{.}\PYG{l+m+mf}{1.}\PYG{n}{gz}
    \PYG{o}{/}\PYG{n}{usr}\PYG{o}{/}\PYG{n}{share}\PYG{o}{/}\PYG{n}{man}\PYG{o}{/}\PYG{n}{man1}\PYG{o}{/}\PYG{n}{mysqlbinlog}\PYG{o}{.}\PYG{l+m+mf}{1.}\PYG{n}{gz}
    \PYG{o}{/}\PYG{n}{usr}\PYG{o}{/}\PYG{n}{share}\PYG{o}{/}\PYG{n}{man}\PYG{o}{/}\PYG{n}{man1}\PYG{o}{/}\PYG{n}{mysqld\PYGZus{}safe}\PYG{o}{.}\PYG{l+m+mf}{1.}\PYG{n}{gz}
    \PYG{o}{/}\PYG{n}{usr}\PYG{o}{/}\PYG{n}{share}\PYG{o}{/}\PYG{n}{man}\PYG{o}{/}\PYG{n}{man1}\PYG{o}{/}\PYG{n}{mysqlrepair}\PYG{o}{.}\PYG{l+m+mf}{1.}\PYG{n}{gz}
    \PYG{o}{/}\PYG{n}{usr}\PYG{o}{/}\PYG{n}{share}\PYG{o}{/}\PYG{n}{man}\PYG{o}{/}\PYG{n}{man1}\PYG{o}{/}\PYG{n}{mysqldump}\PYG{o}{.}\PYG{l+m+mf}{1.}\PYG{n}{gz}
    \PYG{o}{/}\PYG{n}{usr}\PYG{o}{/}\PYG{n}{share}\PYG{o}{/}\PYG{n}{man}\PYG{o}{/}\PYG{n}{man1}\PYG{o}{/}\PYG{n}{mysqlanalyze}\PYG{o}{.}\PYG{l+m+mf}{1.}\PYG{n}{gz}
    \PYG{o}{/}\PYG{n}{usr}\PYG{o}{/}\PYG{n}{share}\PYG{o}{/}\PYG{n}{man}\PYG{o}{/}\PYG{n}{man1}\PYG{o}{/}\PYG{n}{mysqlslap}\PYG{o}{.}\PYG{l+m+mf}{1.}\PYG{n}{gz}
    \PYG{o}{/}\PYG{n}{usr}\PYG{o}{/}\PYG{n}{share}\PYG{o}{/}\PYG{n}{man}\PYG{o}{/}\PYG{n}{man1}\PYG{o}{/}\PYG{n}{mysqld\PYGZus{}multi}\PYG{o}{.}\PYG{l+m+mf}{1.}\PYG{n}{gz}
    \PYG{o}{/}\PYG{n}{usr}\PYG{o}{/}\PYG{n}{share}\PYG{o}{/}\PYG{n}{man}\PYG{o}{/}\PYG{n}{man1}\PYG{o}{/}\PYG{n}{mysql\PYGZus{}secure\PYGZus{}installation}\PYG{o}{.}\PYG{l+m+mf}{1.}\PYG{n}{gz}
    \PYG{o}{/}\PYG{n}{usr}\PYG{o}{/}\PYG{n}{share}\PYG{o}{/}\PYG{n}{man}\PYG{o}{/}\PYG{n}{man1}\PYG{o}{/}\PYG{n}{mysqlreport}\PYG{o}{.}\PYG{l+m+mf}{1.}\PYG{n}{gz}
    \PYG{o}{/}\PYG{n}{usr}\PYG{o}{/}\PYG{n}{share}\PYG{o}{/}\PYG{n}{man}\PYG{o}{/}\PYG{n}{man1}\PYG{o}{/}\PYG{n}{mysqlimport}\PYG{o}{.}\PYG{l+m+mf}{1.}\PYG{n}{gz}
    \PYG{o}{/}\PYG{n}{usr}\PYG{o}{/}\PYG{n}{share}\PYG{o}{/}\PYG{n}{man}\PYG{o}{/}\PYG{n}{man1}\PYG{o}{/}\PYG{n}{mysqlcheck}\PYG{o}{.}\PYG{l+m+mf}{1.}\PYG{n}{gz}
    \PYG{o}{/}\PYG{n}{usr}\PYG{o}{/}\PYG{n}{share}\PYG{o}{/}\PYG{n}{man}\PYG{o}{/}\PYG{n}{man1}\PYG{o}{/}\PYG{n}{mysql\PYGZus{}plugin}\PYG{o}{.}\PYG{l+m+mf}{1.}\PYG{n}{gz}
    \PYG{o}{/}\PYG{n}{usr}\PYG{o}{/}\PYG{n}{share}\PYG{o}{/}\PYG{n}{man}\PYG{o}{/}\PYG{n}{man1}\PYG{o}{/}\PYG{n}{mysqladmin}\PYG{o}{.}\PYG{l+m+mf}{1.}\PYG{n}{gz}
    \PYG{o}{/}\PYG{n}{usr}\PYG{o}{/}\PYG{n}{share}\PYG{o}{/}\PYG{n}{man}\PYG{o}{/}\PYG{n}{man1}\PYG{o}{/}\PYG{n}{mysql}\PYG{o}{.}\PYG{l+m+mf}{1.}\PYG{n}{gz}
    \PYG{o}{/}\PYG{n}{usr}\PYG{o}{/}\PYG{n}{share}\PYG{o}{/}\PYG{n}{man}\PYG{o}{/}\PYG{n}{man1}\PYG{o}{/}\PYG{n}{mysql\PYGZus{}embedded}\PYG{o}{.}\PYG{l+m+mf}{1.}\PYG{n}{gz}
    \PYG{o}{/}\PYG{n}{usr}\PYG{o}{/}\PYG{n}{share}\PYG{o}{/}\PYG{n}{man}\PYG{o}{/}\PYG{n}{man1}\PYG{o}{/}\PYG{n}{mysql\PYGZus{}install\PYGZus{}db}\PYG{o}{.}\PYG{l+m+mf}{1.}\PYG{n}{gz}
    \PYG{o}{/}\PYG{n}{usr}\PYG{o}{/}\PYG{n}{share}\PYG{o}{/}\PYG{n}{man}\PYG{o}{/}\PYG{n}{man1}\PYG{o}{/}\PYG{n}{mysql\PYGZus{}ssl\PYGZus{}rsa\PYGZus{}setup}\PYG{o}{.}\PYG{l+m+mf}{1.}\PYG{n}{gz}
    \PYG{o}{/}\PYG{n}{usr}\PYG{o}{/}\PYG{n}{share}\PYG{o}{/}\PYG{n}{man}\PYG{o}{/}\PYG{n}{man1}\PYG{o}{/}\PYG{n}{mysqlman}\PYG{o}{.}\PYG{l+m+mf}{1.}\PYG{n}{gz}
    \PYG{o}{/}\PYG{n}{usr}\PYG{o}{/}\PYG{n}{share}\PYG{o}{/}\PYG{n}{man}\PYG{o}{/}\PYG{n}{man1}\PYG{o}{/}\PYG{n}{mysql\PYGZus{}tzinfo\PYGZus{}to\PYGZus{}sql}\PYG{o}{.}\PYG{l+m+mf}{1.}\PYG{n}{gz}
    \PYG{o}{/}\PYG{n}{usr}\PYG{o}{/}\PYG{n}{share}\PYG{o}{/}\PYG{n}{man}\PYG{o}{/}\PYG{n}{man8}\PYG{o}{/}\PYG{n}{mysqld}\PYG{o}{.}\PYG{l+m+mf}{8.}\PYG{n}{gz}
    \PYG{o}{/}\PYG{n}{usr}\PYG{o}{/}\PYG{n}{share}\PYG{o}{/}\PYG{n}{sosreport}\PYG{o}{/}\PYG{n}{sos}\PYG{o}{/}\PYG{n}{plugins}\PYG{o}{/}\PYG{n}{mysql}\PYG{o}{.}\PYG{n}{py}
    \PYG{o}{/}\PYG{n}{usr}\PYG{o}{/}\PYG{n}{share}\PYG{o}{/}\PYG{n}{sosreport}\PYG{o}{/}\PYG{n}{sos}\PYG{o}{/}\PYG{n}{plugins}\PYG{o}{/}\PYG{n}{\PYGZus{}\PYGZus{}pycache\PYGZus{}\PYGZus{}}\PYG{o}{/}\PYG{n}{mysql}\PYG{o}{.}\PYG{n}{cpython}\PYG{o}{\PYGZhy{}}\PYG{l+m+mf}{35.}\PYG{n}{pyc}
    \PYG{o}{/}\PYG{n}{usr}\PYG{o}{/}\PYG{n}{share}\PYG{o}{/}\PYG{n}{apport}\PYG{o}{/}\PYG{n}{package}\PYG{o}{\PYGZhy{}}\PYG{n}{hooks}\PYG{o}{/}\PYG{n}{source\PYGZus{}mysql}\PYG{o}{\PYGZhy{}}\PYG{l+m+mf}{5.7}\PYG{o}{.}\PYG{n}{py}
    \PYG{o}{/}\PYG{n}{usr}\PYG{o}{/}\PYG{n}{share}\PYG{o}{/}\PYG{n}{bash}\PYG{o}{\PYGZhy{}}\PYG{n}{completion}\PYG{o}{/}\PYG{n}{completions}\PYG{o}{/}\PYG{n}{mysqladmin}
    \PYG{o}{/}\PYG{n}{usr}\PYG{o}{/}\PYG{n}{share}\PYG{o}{/}\PYG{n}{bash}\PYG{o}{\PYGZhy{}}\PYG{n}{completion}\PYG{o}{/}\PYG{n}{completions}\PYG{o}{/}\PYG{n}{mysql}
    \PYG{o}{/}\PYG{n}{usr}\PYG{o}{/}\PYG{n}{share}\PYG{o}{/}\PYG{n}{mysql}\PYG{o}{\PYGZhy{}}\PYG{n}{common}
\PYG{n}{find} \PYG{o}{/}\PYG{n}{lib} \PYG{o}{\PYGZhy{}}\PYG{n}{name} \PYG{l+s+s2}{\PYGZdq{}}\PYG{l+s+s2}{*mysql*}\PYG{l+s+s2}{\PYGZdq{}}
    \PYG{o}{/}\PYG{n}{lib}\PYG{o}{/}\PYG{n}{systemd}\PYG{o}{/}\PYG{n}{system}\PYG{o}{/}\PYG{n}{mysql}\PYG{o}{.}\PYG{n}{service}
\PYG{n}{dpkg} \PYG{o}{\PYGZhy{}}\PYG{n}{l} \PYG{o}{\textbar{}} \PYG{n}{grep} \PYG{n}{mysql}
    \PYG{n}{ii}  \PYG{n}{mysql}\PYG{o}{\PYGZhy{}}\PYG{n}{client}\PYG{o}{\PYGZhy{}}\PYG{l+m+mf}{5.7}                   \PYG{l+m+mf}{5.7}\PYG{o}{.}\PYG{l+m+mi}{19}\PYG{o}{\PYGZhy{}}\PYG{l+m+mi}{0}\PYG{n}{ubuntu0}\PYG{o}{.}\PYG{l+m+mf}{16.04}\PYG{o}{.}\PYG{l+m+mi}{1}                    \PYG{n}{amd64}        \PYG{n}{MySQL} \PYG{n}{database} \PYG{n}{client} \PYG{n}{binaries}
    \PYG{n}{ii}  \PYG{n}{mysql}\PYG{o}{\PYGZhy{}}\PYG{n}{client}\PYG{o}{\PYGZhy{}}\PYG{n}{core}\PYG{o}{\PYGZhy{}}\PYG{l+m+mf}{5.7}              \PYG{l+m+mf}{5.7}\PYG{o}{.}\PYG{l+m+mi}{19}\PYG{o}{\PYGZhy{}}\PYG{l+m+mi}{0}\PYG{n}{ubuntu0}\PYG{o}{.}\PYG{l+m+mf}{16.04}\PYG{o}{.}\PYG{l+m+mi}{1}                    \PYG{n}{amd64}        \PYG{n}{MySQL} \PYG{n}{database} \PYG{n}{core} \PYG{n}{client} \PYG{n}{binaries}
    \PYG{n}{ii}  \PYG{n}{mysql}\PYG{o}{\PYGZhy{}}\PYG{n}{common}                       \PYG{l+m+mf}{5.7}\PYG{o}{.}\PYG{l+m+mi}{19}\PYG{o}{\PYGZhy{}}\PYG{l+m+mi}{0}\PYG{n}{ubuntu0}\PYG{o}{.}\PYG{l+m+mf}{16.04}\PYG{o}{.}\PYG{l+m+mi}{1}                    \PYG{n+nb}{all}          \PYG{n}{MySQL} \PYG{n}{database} \PYG{n}{common} \PYG{n}{files}\PYG{p}{,} \PYG{n}{e}\PYG{o}{.}\PYG{n}{g}\PYG{o}{.} \PYG{o}{/}\PYG{n}{etc}\PYG{o}{/}\PYG{n}{mysql}\PYG{o}{/}\PYG{n}{my}\PYG{o}{.}\PYG{n}{cnf}
    \PYG{n}{ii}  \PYG{n}{mysql}\PYG{o}{\PYGZhy{}}\PYG{n}{server}                       \PYG{l+m+mf}{5.7}\PYG{o}{.}\PYG{l+m+mi}{19}\PYG{o}{\PYGZhy{}}\PYG{l+m+mi}{0}\PYG{n}{ubuntu0}\PYG{o}{.}\PYG{l+m+mf}{16.04}\PYG{o}{.}\PYG{l+m+mi}{1}                    \PYG{n+nb}{all}          \PYG{n}{MySQL} \PYG{n}{database} \PYG{n}{server} \PYG{p}{(}\PYG{n}{metapackage} \PYG{n}{depending} \PYG{n}{on} \PYG{n}{the} \PYG{n}{latest} \PYG{n}{version}\PYG{p}{)}
    \PYG{n}{ii}  \PYG{n}{mysql}\PYG{o}{\PYGZhy{}}\PYG{n}{server}\PYG{o}{\PYGZhy{}}\PYG{l+m+mf}{5.7}                   \PYG{l+m+mf}{5.7}\PYG{o}{.}\PYG{l+m+mi}{19}\PYG{o}{\PYGZhy{}}\PYG{l+m+mi}{0}\PYG{n}{ubuntu0}\PYG{o}{.}\PYG{l+m+mf}{16.04}\PYG{o}{.}\PYG{l+m+mi}{1}                    \PYG{n}{amd64}        \PYG{n}{MySQL} \PYG{n}{database} \PYG{n}{server} \PYG{n}{binaries} \PYG{o+ow}{and} \PYG{n}{system} \PYG{n}{database} \PYG{n}{setup}
    \PYG{n}{ii}  \PYG{n}{mysql}\PYG{o}{\PYGZhy{}}\PYG{n}{server}\PYG{o}{\PYGZhy{}}\PYG{n}{core}\PYG{o}{\PYGZhy{}}\PYG{l+m+mf}{5.7}              \PYG{l+m+mf}{5.7}\PYG{o}{.}\PYG{l+m+mi}{19}\PYG{o}{\PYGZhy{}}\PYG{l+m+mi}{0}\PYG{n}{ubuntu0}\PYG{o}{.}\PYG{l+m+mf}{16.04}\PYG{o}{.}\PYG{l+m+mi}{1}                    \PYG{n}{amd64}        \PYG{n}{MySQL} \PYG{n}{database} \PYG{n}{server} \PYG{n}{binaries}
\end{sphinxVerbatim}

\item {} 
remove mysql

\begin{sphinxVerbatim}[commandchars=\\\{\}]
apt autoremove \PYGZhy{}\PYGZhy{}purge mysql\PYGZhy{}common \PYGZhy{}y
find /etc \PYGZhy{}name \PYGZdq{}*mysql*\PYGZdq{}
    /etc/mysql
find /usr \PYGZhy{}name \PYGZdq{}*mysql*\PYGZdq{}
    /usr/share/doc/kamailio/examples/icscf/icscf.mysql.sql.gz
    /usr/share/doc/kamailio/examples/kamailio/acc\PYGZhy{}mysql.cfg.gz
    /usr/share/vim/vim74/syntax/mysql.vim
    /usr/share/sosreport/sos/plugins/mysql.py
    /usr/share/sosreport/sos/plugins/\PYGZus{}\PYGZus{}pycache\PYGZus{}\PYGZus{}/mysql.cpython\PYGZhy{}35.pyc
    /usr/share/bash\PYGZhy{}completion/completions/mysqladmin
    /usr/share/bash\PYGZhy{}completion/completions/mysql
这个配置目录没有删除,手动删除
rm /etc/mysql \PYGZhy{}rf
rm /var/lib/mysql \PYGZhy{}rf
\end{sphinxVerbatim}

\end{itemize}


\subsubsection{install\&remove redis}
\label{\detokenize{linux/apt:install-remove-redis}}\begin{itemize}
\item {} 
install redis

\begin{sphinxVerbatim}[commandchars=\\\{\}]
\PYG{n}{apt} \PYG{n}{install} \PYG{n}{redis}\PYG{o}{\PYGZhy{}}\PYG{n}{server} \PYG{o}{\PYGZhy{}}\PYG{n}{y}
\PYG{n}{find} \PYG{o}{/}\PYG{n}{etc} \PYG{o}{\PYGZhy{}}\PYG{n}{name} \PYG{l+s+s2}{\PYGZdq{}}\PYG{l+s+s2}{*redis*}\PYG{l+s+s2}{\PYGZdq{}}
    \PYG{o}{/}\PYG{n}{etc}\PYG{o}{/}\PYG{n}{logrotate}\PYG{o}{.}\PYG{n}{d}\PYG{o}{/}\PYG{n}{redis}\PYG{o}{\PYGZhy{}}\PYG{n}{server}
    \PYG{o}{/}\PYG{n}{etc}\PYG{o}{/}\PYG{n}{rc1}\PYG{o}{.}\PYG{n}{d}\PYG{o}{/}\PYG{n}{K01redis}\PYG{o}{\PYGZhy{}}\PYG{n}{server}
    \PYG{o}{/}\PYG{n}{etc}\PYG{o}{/}\PYG{n}{rc5}\PYG{o}{.}\PYG{n}{d}\PYG{o}{/}\PYG{n}{S02redis}\PYG{o}{\PYGZhy{}}\PYG{n}{server}
    \PYG{o}{/}\PYG{n}{etc}\PYG{o}{/}\PYG{n}{redis}
    \PYG{o}{/}\PYG{n}{etc}\PYG{o}{/}\PYG{n}{redis}\PYG{o}{/}\PYG{n}{redis}\PYG{o}{.}\PYG{n}{conf}
    \PYG{o}{/}\PYG{n}{etc}\PYG{o}{/}\PYG{n}{redis}\PYG{o}{/}\PYG{n}{redis}\PYG{o}{\PYGZhy{}}\PYG{n}{server}\PYG{o}{.}\PYG{n}{pre}\PYG{o}{\PYGZhy{}}\PYG{n}{up}\PYG{o}{.}\PYG{n}{d}
    \PYG{o}{/}\PYG{n}{etc}\PYG{o}{/}\PYG{n}{redis}\PYG{o}{/}\PYG{n}{redis}\PYG{o}{\PYGZhy{}}\PYG{n}{server}\PYG{o}{.}\PYG{n}{post}\PYG{o}{\PYGZhy{}}\PYG{n}{down}\PYG{o}{.}\PYG{n}{d}
    \PYG{o}{/}\PYG{n}{etc}\PYG{o}{/}\PYG{n}{redis}\PYG{o}{/}\PYG{n}{redis}\PYG{o}{\PYGZhy{}}\PYG{n}{server}\PYG{o}{.}\PYG{n}{post}\PYG{o}{\PYGZhy{}}\PYG{n}{up}\PYG{o}{.}\PYG{n}{d}
    \PYG{o}{/}\PYG{n}{etc}\PYG{o}{/}\PYG{n}{redis}\PYG{o}{/}\PYG{n}{redis}\PYG{o}{\PYGZhy{}}\PYG{n}{server}\PYG{o}{.}\PYG{n}{pre}\PYG{o}{\PYGZhy{}}\PYG{n}{down}\PYG{o}{.}\PYG{n}{d}
    \PYG{o}{/}\PYG{n}{etc}\PYG{o}{/}\PYG{n}{init}\PYG{o}{.}\PYG{n}{d}\PYG{o}{/}\PYG{n}{redis}\PYG{o}{\PYGZhy{}}\PYG{n}{server}
    \PYG{o}{/}\PYG{n}{etc}\PYG{o}{/}\PYG{n}{rc3}\PYG{o}{.}\PYG{n}{d}\PYG{o}{/}\PYG{n}{S02redis}\PYG{o}{\PYGZhy{}}\PYG{n}{server}
    \PYG{o}{/}\PYG{n}{etc}\PYG{o}{/}\PYG{n}{rc2}\PYG{o}{.}\PYG{n}{d}\PYG{o}{/}\PYG{n}{S02redis}\PYG{o}{\PYGZhy{}}\PYG{n}{server}
    \PYG{o}{/}\PYG{n}{etc}\PYG{o}{/}\PYG{n}{rc0}\PYG{o}{.}\PYG{n}{d}\PYG{o}{/}\PYG{n}{K01redis}\PYG{o}{\PYGZhy{}}\PYG{n}{server}
    \PYG{o}{/}\PYG{n}{etc}\PYG{o}{/}\PYG{n}{default}\PYG{o}{/}\PYG{n}{redis}\PYG{o}{\PYGZhy{}}\PYG{n}{server}
    \PYG{o}{/}\PYG{n}{etc}\PYG{o}{/}\PYG{n}{systemd}\PYG{o}{/}\PYG{n}{system}\PYG{o}{/}\PYG{n}{multi}\PYG{o}{\PYGZhy{}}\PYG{n}{user}\PYG{o}{.}\PYG{n}{target}\PYG{o}{.}\PYG{n}{wants}\PYG{o}{/}\PYG{n}{redis}\PYG{o}{\PYGZhy{}}\PYG{n}{server}\PYG{o}{.}\PYG{n}{service}
    \PYG{o}{/}\PYG{n}{etc}\PYG{o}{/}\PYG{n}{systemd}\PYG{o}{/}\PYG{n}{system}\PYG{o}{/}\PYG{n}{redis}\PYG{o}{.}\PYG{n}{service}
    \PYG{o}{/}\PYG{n}{etc}\PYG{o}{/}\PYG{n}{rc4}\PYG{o}{.}\PYG{n}{d}\PYG{o}{/}\PYG{n}{S02redis}\PYG{o}{\PYGZhy{}}\PYG{n}{server}
    \PYG{o}{/}\PYG{n}{etc}\PYG{o}{/}\PYG{n}{rc6}\PYG{o}{.}\PYG{n}{d}\PYG{o}{/}\PYG{n}{K01redis}\PYG{o}{\PYGZhy{}}\PYG{n}{server}
\PYG{n}{find} \PYG{o}{/}\PYG{n}{usr} \PYG{o}{\PYGZhy{}}\PYG{n}{name} \PYG{l+s+s2}{\PYGZdq{}}\PYG{l+s+s2}{*redis*}\PYG{l+s+s2}{\PYGZdq{}}
    \PYG{o}{/}\PYG{n}{usr}\PYG{o}{/}\PYG{n}{lib}\PYG{o}{/}\PYG{n}{tmpfiles}\PYG{o}{.}\PYG{n}{d}\PYG{o}{/}\PYG{n}{redis}\PYG{o}{\PYGZhy{}}\PYG{n}{server}\PYG{o}{.}\PYG{n}{conf}
    \PYG{o}{/}\PYG{n}{usr}\PYG{o}{/}\PYG{n+nb}{bin}\PYG{o}{/}\PYG{n}{redis}\PYG{o}{\PYGZhy{}}\PYG{n}{check}\PYG{o}{\PYGZhy{}}\PYG{n}{dump}
    \PYG{o}{/}\PYG{n}{usr}\PYG{o}{/}\PYG{n+nb}{bin}\PYG{o}{/}\PYG{n}{redis}\PYG{o}{\PYGZhy{}}\PYG{n}{benchmark}
    \PYG{o}{/}\PYG{n}{usr}\PYG{o}{/}\PYG{n+nb}{bin}\PYG{o}{/}\PYG{n}{redis}\PYG{o}{\PYGZhy{}}\PYG{n}{server}
    \PYG{o}{/}\PYG{n}{usr}\PYG{o}{/}\PYG{n+nb}{bin}\PYG{o}{/}\PYG{n}{redis}\PYG{o}{\PYGZhy{}}\PYG{n}{cli}
    \PYG{o}{/}\PYG{n}{usr}\PYG{o}{/}\PYG{n+nb}{bin}\PYG{o}{/}\PYG{n}{redis}\PYG{o}{\PYGZhy{}}\PYG{n}{check}\PYG{o}{\PYGZhy{}}\PYG{n}{aof}
    \PYG{o}{/}\PYG{n}{usr}\PYG{o}{/}\PYG{n}{share}\PYG{o}{/}\PYG{n}{doc}\PYG{o}{/}\PYG{n}{redis}\PYG{o}{\PYGZhy{}}\PYG{n}{server}
    \PYG{o}{/}\PYG{n}{usr}\PYG{o}{/}\PYG{n}{share}\PYG{o}{/}\PYG{n}{doc}\PYG{o}{/}\PYG{n}{redis}\PYG{o}{\PYGZhy{}}\PYG{n}{tools}
    \PYG{o}{/}\PYG{n}{usr}\PYG{o}{/}\PYG{n}{share}\PYG{o}{/}\PYG{n}{doc}\PYG{o}{/}\PYG{n}{redis}\PYG{o}{\PYGZhy{}}\PYG{n}{tools}\PYG{o}{/}\PYG{n}{examples}\PYG{o}{/}\PYG{n}{redis}\PYG{o}{\PYGZhy{}}\PYG{n}{trib}\PYG{o}{.}\PYG{n}{rb}
    \PYG{o}{/}\PYG{n}{usr}\PYG{o}{/}\PYG{n}{share}\PYG{o}{/}\PYG{n}{man}\PYG{o}{/}\PYG{n}{man1}\PYG{o}{/}\PYG{n}{redis}\PYG{o}{\PYGZhy{}}\PYG{n}{server}\PYG{o}{.}\PYG{l+m+mf}{1.}\PYG{n}{gz}
    \PYG{o}{/}\PYG{n}{usr}\PYG{o}{/}\PYG{n}{share}\PYG{o}{/}\PYG{n}{man}\PYG{o}{/}\PYG{n}{man1}\PYG{o}{/}\PYG{n}{redis}\PYG{o}{\PYGZhy{}}\PYG{n}{benchmark}\PYG{o}{.}\PYG{l+m+mf}{1.}\PYG{n}{gz}
    \PYG{o}{/}\PYG{n}{usr}\PYG{o}{/}\PYG{n}{share}\PYG{o}{/}\PYG{n}{man}\PYG{o}{/}\PYG{n}{man1}\PYG{o}{/}\PYG{n}{redis}\PYG{o}{\PYGZhy{}}\PYG{n}{cli}\PYG{o}{.}\PYG{l+m+mf}{1.}\PYG{n}{gz}
    \PYG{o}{/}\PYG{n}{usr}\PYG{o}{/}\PYG{n}{share}\PYG{o}{/}\PYG{n}{bash}\PYG{o}{\PYGZhy{}}\PYG{n}{completion}\PYG{o}{/}\PYG{n}{completions}\PYG{o}{/}\PYG{n}{bash\PYGZus{}completion}\PYG{o}{.}\PYG{n}{d}\PYG{o}{/}\PYG{n}{redis}\PYG{o}{\PYGZhy{}}\PYG{n}{cli}
\PYG{n}{find} \PYG{o}{/}\PYG{n}{lib} \PYG{o}{\PYGZhy{}}\PYG{n}{name} \PYG{l+s+s2}{\PYGZdq{}}\PYG{l+s+s2}{*redis*}\PYG{l+s+s2}{\PYGZdq{}}
    \PYG{o}{/}\PYG{n}{lib}\PYG{o}{/}\PYG{n}{systemd}\PYG{o}{/}\PYG{n}{system}\PYG{o}{/}\PYG{n}{redis}\PYG{o}{\PYGZhy{}}\PYG{n}{server}\PYG{o}{.}\PYG{n}{service}
\PYG{n}{dpkg} \PYG{o}{\PYGZhy{}}\PYG{n}{l} \PYG{o}{\textbar{}} \PYG{n}{grep} \PYG{n}{redis}
    \PYG{n}{ii}  \PYG{n}{redis}\PYG{o}{\PYGZhy{}}\PYG{n}{server}  \PYG{l+m+mi}{2}\PYG{p}{:}\PYG{l+m+mf}{3.0}\PYG{o}{.}\PYG{l+m+mi}{6}\PYG{o}{\PYGZhy{}}\PYG{l+m+mi}{1}      \PYG{n}{amd64}        \PYG{n}{Persistent} \PYG{n}{key}\PYG{o}{\PYGZhy{}}\PYG{n}{value} \PYG{n}{database} \PYG{k}{with} \PYG{n}{network} \PYG{n}{interface}
    \PYG{n}{ii}  \PYG{n}{redis}\PYG{o}{\PYGZhy{}}\PYG{n}{tools}   \PYG{l+m+mi}{2}\PYG{p}{:}\PYG{l+m+mf}{3.0}\PYG{o}{.}\PYG{l+m+mi}{6}\PYG{o}{\PYGZhy{}}\PYG{l+m+mi}{1}      \PYG{n}{amd64}        \PYG{n}{Persistent} \PYG{n}{key}\PYG{o}{\PYGZhy{}}\PYG{n}{value} \PYG{n}{database} \PYG{k}{with} \PYG{n}{network} \PYG{n}{interface} \PYG{p}{(}\PYG{n}{client}\PYG{p}{)}
\end{sphinxVerbatim}

\item {} 
remove redis

\begin{sphinxVerbatim}[commandchars=\\\{\}]
\PYG{n}{所有内容都返回空}
\PYG{n}{apt} \PYG{n}{autoremove} \PYG{o}{\PYGZhy{}}\PYG{o}{\PYGZhy{}}\PYG{n}{purge} \PYG{n}{redis}\PYG{o}{\PYGZhy{}}\PYG{n}{server} \PYG{o}{\PYGZhy{}}\PYG{n}{y}
\end{sphinxVerbatim}

\end{itemize}


\subsubsection{install\&remove apache}
\label{\detokenize{linux/apt:install-remove-apache}}\begin{itemize}
\item {} 
install apache

\begin{sphinxVerbatim}[commandchars=\\\{\}]
\PYG{n}{apt} \PYG{n}{install} \PYG{n}{apache2} \PYG{o}{\PYGZhy{}}\PYG{n}{y}
\PYG{n}{find} \PYG{o}{/}\PYG{n}{etc} \PYG{o}{\PYGZhy{}}\PYG{n}{name} \PYG{l+s+s2}{\PYGZdq{}}\PYG{l+s+s2}{*apache*}\PYG{l+s+s2}{\PYGZdq{}}
    \PYG{o}{/}\PYG{n}{etc}\PYG{o}{/}\PYG{n}{logrotate}\PYG{o}{.}\PYG{n}{d}\PYG{o}{/}\PYG{n}{apache2}
    \PYG{o}{/}\PYG{n}{etc}\PYG{o}{/}\PYG{n}{rc1}\PYG{o}{.}\PYG{n}{d}\PYG{o}{/}\PYG{n}{K01apache}\PYG{o}{\PYGZhy{}}\PYG{n}{htcacheclean}
    \PYG{o}{/}\PYG{n}{etc}\PYG{o}{/}\PYG{n}{rc1}\PYG{o}{.}\PYG{n}{d}\PYG{o}{/}\PYG{n}{K01apache2}
    \PYG{o}{/}\PYG{n}{etc}\PYG{o}{/}\PYG{n}{rc5}\PYG{o}{.}\PYG{n}{d}\PYG{o}{/}\PYG{n}{K01apache}\PYG{o}{\PYGZhy{}}\PYG{n}{htcacheclean}
    \PYG{o}{/}\PYG{n}{etc}\PYG{o}{/}\PYG{n}{rc5}\PYG{o}{.}\PYG{n}{d}\PYG{o}{/}\PYG{n}{S02apache2}
    \PYG{o}{/}\PYG{n}{etc}\PYG{o}{/}\PYG{n}{init}\PYG{o}{.}\PYG{n}{d}\PYG{o}{/}\PYG{n}{apache2}
    \PYG{o}{/}\PYG{n}{etc}\PYG{o}{/}\PYG{n}{init}\PYG{o}{.}\PYG{n}{d}\PYG{o}{/}\PYG{n}{apache}\PYG{o}{\PYGZhy{}}\PYG{n}{htcacheclean}
    \PYG{o}{/}\PYG{n}{etc}\PYG{o}{/}\PYG{n}{rc3}\PYG{o}{.}\PYG{n}{d}\PYG{o}{/}\PYG{n}{K01apache}\PYG{o}{\PYGZhy{}}\PYG{n}{htcacheclean}
    \PYG{o}{/}\PYG{n}{etc}\PYG{o}{/}\PYG{n}{rc3}\PYG{o}{.}\PYG{n}{d}\PYG{o}{/}\PYG{n}{S02apache2}
    \PYG{o}{/}\PYG{n}{etc}\PYG{o}{/}\PYG{n}{ufw}\PYG{o}{/}\PYG{n}{applications}\PYG{o}{.}\PYG{n}{d}\PYG{o}{/}\PYG{n}{apache2}
    \PYG{o}{/}\PYG{n}{etc}\PYG{o}{/}\PYG{n}{ufw}\PYG{o}{/}\PYG{n}{applications}\PYG{o}{.}\PYG{n}{d}\PYG{o}{/}\PYG{n}{apache2}\PYG{o}{\PYGZhy{}}\PYG{n}{utils}\PYG{o}{.}\PYG{n}{ufw}\PYG{o}{.}\PYG{n}{profile}
    \PYG{o}{/}\PYG{n}{etc}\PYG{o}{/}\PYG{n}{apache2}
    \PYG{o}{/}\PYG{n}{etc}\PYG{o}{/}\PYG{n}{apache2}\PYG{o}{/}\PYG{n}{apache2}\PYG{o}{.}\PYG{n}{conf}
    \PYG{o}{/}\PYG{n}{etc}\PYG{o}{/}\PYG{n}{rc2}\PYG{o}{.}\PYG{n}{d}\PYG{o}{/}\PYG{n}{K01apache}\PYG{o}{\PYGZhy{}}\PYG{n}{htcacheclean}
    \PYG{o}{/}\PYG{n}{etc}\PYG{o}{/}\PYG{n}{rc2}\PYG{o}{.}\PYG{n}{d}\PYG{o}{/}\PYG{n}{S02apache2}
    \PYG{o}{/}\PYG{n}{etc}\PYG{o}{/}\PYG{n}{rc0}\PYG{o}{.}\PYG{n}{d}\PYG{o}{/}\PYG{n}{K01apache}\PYG{o}{\PYGZhy{}}\PYG{n}{htcacheclean}
    \PYG{o}{/}\PYG{n}{etc}\PYG{o}{/}\PYG{n}{rc0}\PYG{o}{.}\PYG{n}{d}\PYG{o}{/}\PYG{n}{K01apache2}
    \PYG{o}{/}\PYG{n}{etc}\PYG{o}{/}\PYG{n}{default}\PYG{o}{/}\PYG{n}{apache}\PYG{o}{\PYGZhy{}}\PYG{n}{htcacheclean}
    \PYG{o}{/}\PYG{n}{etc}\PYG{o}{/}\PYG{n}{cron}\PYG{o}{.}\PYG{n}{daily}\PYG{o}{/}\PYG{n}{apache2}
    \PYG{o}{/}\PYG{n}{etc}\PYG{o}{/}\PYG{n}{rc4}\PYG{o}{.}\PYG{n}{d}\PYG{o}{/}\PYG{n}{K01apache}\PYG{o}{\PYGZhy{}}\PYG{n}{htcacheclean}
    \PYG{o}{/}\PYG{n}{etc}\PYG{o}{/}\PYG{n}{rc4}\PYG{o}{.}\PYG{n}{d}\PYG{o}{/}\PYG{n}{S02apache2}
    \PYG{o}{/}\PYG{n}{etc}\PYG{o}{/}\PYG{n}{rc6}\PYG{o}{.}\PYG{n}{d}\PYG{o}{/}\PYG{n}{K01apache}\PYG{o}{\PYGZhy{}}\PYG{n}{htcacheclean}
    \PYG{o}{/}\PYG{n}{etc}\PYG{o}{/}\PYG{n}{rc6}\PYG{o}{.}\PYG{n}{d}\PYG{o}{/}\PYG{n}{K01apache2}
\PYG{n}{find} \PYG{o}{/}\PYG{n}{usr} \PYG{o}{\PYGZhy{}}\PYG{n}{name} \PYG{l+s+s2}{\PYGZdq{}}\PYG{l+s+s2}{*apache*}\PYG{l+s+s2}{\PYGZdq{}}
    \PYG{o}{/}\PYG{n}{usr}\PYG{o}{/}\PYG{n}{lib}\PYG{o}{/}\PYG{n}{apache2}
    \PYG{o}{/}\PYG{n}{usr}\PYG{o}{/}\PYG{n}{sbin}\PYG{o}{/}\PYG{n}{apachectl}
    \PYG{o}{/}\PYG{n}{usr}\PYG{o}{/}\PYG{n}{sbin}\PYG{o}{/}\PYG{n}{apache2ctl}
    \PYG{o}{/}\PYG{n}{usr}\PYG{o}{/}\PYG{n}{sbin}\PYG{o}{/}\PYG{n}{apache2}
    \PYG{o}{/}\PYG{n}{usr}\PYG{o}{/}\PYG{n}{share}\PYG{o}{/}\PYG{n}{doc}\PYG{o}{/}\PYG{n}{apache2}\PYG{o}{\PYGZhy{}}\PYG{n+nb}{bin}
    \PYG{o}{/}\PYG{n}{usr}\PYG{o}{/}\PYG{n}{share}\PYG{o}{/}\PYG{n}{doc}\PYG{o}{/}\PYG{n}{apache2}
    \PYG{o}{/}\PYG{n}{usr}\PYG{o}{/}\PYG{n}{share}\PYG{o}{/}\PYG{n}{doc}\PYG{o}{/}\PYG{n}{apache2}\PYG{o}{/}\PYG{n}{examples}\PYG{o}{/}\PYG{n}{apache2}\PYG{o}{.}\PYG{n}{monit}
    \PYG{o}{/}\PYG{n}{usr}\PYG{o}{/}\PYG{n}{share}\PYG{o}{/}\PYG{n}{doc}\PYG{o}{/}\PYG{n}{apache2}\PYG{o}{\PYGZhy{}}\PYG{n}{data}
    \PYG{o}{/}\PYG{n}{usr}\PYG{o}{/}\PYG{n}{share}\PYG{o}{/}\PYG{n}{doc}\PYG{o}{/}\PYG{n}{apache2}\PYG{o}{\PYGZhy{}}\PYG{n}{utils}
    \PYG{o}{/}\PYG{n}{usr}\PYG{o}{/}\PYG{n}{share}\PYG{o}{/}\PYG{n}{apache2}
    \PYG{o}{/}\PYG{n}{usr}\PYG{o}{/}\PYG{n}{share}\PYG{o}{/}\PYG{n}{apache2}\PYG{o}{/}\PYG{n}{apache2}\PYG{o}{\PYGZhy{}}\PYG{n}{maintscript}\PYG{o}{\PYGZhy{}}\PYG{n}{helper}
    \PYG{o}{/}\PYG{n}{usr}\PYG{o}{/}\PYG{n}{share}\PYG{o}{/}\PYG{n}{apache2}\PYG{o}{/}\PYG{n}{icons}\PYG{o}{/}\PYG{n}{apache\PYGZus{}pb}\PYG{o}{.}\PYG{n}{png}
    \PYG{o}{/}\PYG{n}{usr}\PYG{o}{/}\PYG{n}{share}\PYG{o}{/}\PYG{n}{apache2}\PYG{o}{/}\PYG{n}{icons}\PYG{o}{/}\PYG{n}{apache\PYGZus{}pb}\PYG{o}{.}\PYG{n}{gif}
    \PYG{o}{/}\PYG{n}{usr}\PYG{o}{/}\PYG{n}{share}\PYG{o}{/}\PYG{n}{apache2}\PYG{o}{/}\PYG{n}{icons}\PYG{o}{/}\PYG{n}{apache\PYGZus{}pb}\PYG{o}{.}\PYG{n}{svg}
    \PYG{o}{/}\PYG{n}{usr}\PYG{o}{/}\PYG{n}{share}\PYG{o}{/}\PYG{n}{apache2}\PYG{o}{/}\PYG{n}{icons}\PYG{o}{/}\PYG{n}{apache\PYGZus{}pb2}\PYG{o}{.}\PYG{n}{gif}
    \PYG{o}{/}\PYG{n}{usr}\PYG{o}{/}\PYG{n}{share}\PYG{o}{/}\PYG{n}{apache2}\PYG{o}{/}\PYG{n}{icons}\PYG{o}{/}\PYG{n}{apache\PYGZus{}pb2}\PYG{o}{.}\PYG{n}{png}
    \PYG{o}{/}\PYG{n}{usr}\PYG{o}{/}\PYG{n}{share}\PYG{o}{/}\PYG{n}{vim}\PYG{o}{/}\PYG{n}{vim74}\PYG{o}{/}\PYG{n}{syntax}\PYG{o}{/}\PYG{n}{apachestyle}\PYG{o}{.}\PYG{n}{vim}
    \PYG{o}{/}\PYG{n}{usr}\PYG{o}{/}\PYG{n}{share}\PYG{o}{/}\PYG{n}{vim}\PYG{o}{/}\PYG{n}{vim74}\PYG{o}{/}\PYG{n}{syntax}\PYG{o}{/}\PYG{n}{apache}\PYG{o}{.}\PYG{n}{vim}
    \PYG{o}{/}\PYG{n}{usr}\PYG{o}{/}\PYG{n}{share}\PYG{o}{/}\PYG{n}{bug}\PYG{o}{/}\PYG{n}{apache2}\PYG{o}{\PYGZhy{}}\PYG{n+nb}{bin}
    \PYG{o}{/}\PYG{n}{usr}\PYG{o}{/}\PYG{n}{share}\PYG{o}{/}\PYG{n}{bug}\PYG{o}{/}\PYG{n}{apache2}
    \PYG{o}{/}\PYG{n}{usr}\PYG{o}{/}\PYG{n}{share}\PYG{o}{/}\PYG{n}{lintian}\PYG{o}{/}\PYG{n}{overrides}\PYG{o}{/}\PYG{n}{apache2}\PYG{o}{\PYGZhy{}}\PYG{n+nb}{bin}
    \PYG{o}{/}\PYG{n}{usr}\PYG{o}{/}\PYG{n}{share}\PYG{o}{/}\PYG{n}{lintian}\PYG{o}{/}\PYG{n}{overrides}\PYG{o}{/}\PYG{n}{apache2}
    \PYG{o}{/}\PYG{n}{usr}\PYG{o}{/}\PYG{n}{share}\PYG{o}{/}\PYG{n}{lintian}\PYG{o}{/}\PYG{n}{overrides}\PYG{o}{/}\PYG{n}{apache2}\PYG{o}{\PYGZhy{}}\PYG{n}{data}
    \PYG{o}{/}\PYG{n}{usr}\PYG{o}{/}\PYG{n}{share}\PYG{o}{/}\PYG{n}{man}\PYG{o}{/}\PYG{n}{man8}\PYG{o}{/}\PYG{n}{apache2}\PYG{o}{.}\PYG{l+m+mf}{8.}\PYG{n}{gz}
    \PYG{o}{/}\PYG{n}{usr}\PYG{o}{/}\PYG{n}{share}\PYG{o}{/}\PYG{n}{man}\PYG{o}{/}\PYG{n}{man8}\PYG{o}{/}\PYG{n}{apachectl}\PYG{o}{.}\PYG{l+m+mf}{8.}\PYG{n}{gz}
    \PYG{o}{/}\PYG{n}{usr}\PYG{o}{/}\PYG{n}{share}\PYG{o}{/}\PYG{n}{man}\PYG{o}{/}\PYG{n}{man8}\PYG{o}{/}\PYG{n}{apache2ctl}\PYG{o}{.}\PYG{l+m+mf}{8.}\PYG{n}{gz}
    \PYG{o}{/}\PYG{n}{usr}\PYG{o}{/}\PYG{n}{share}\PYG{o}{/}\PYG{n}{sosreport}\PYG{o}{/}\PYG{n}{sos}\PYG{o}{/}\PYG{n}{plugins}\PYG{o}{/}\PYG{n}{\PYGZus{}\PYGZus{}pycache\PYGZus{}\PYGZus{}}\PYG{o}{/}\PYG{n}{apache}\PYG{o}{.}\PYG{n}{cpython}\PYG{o}{\PYGZhy{}}\PYG{l+m+mf}{35.}\PYG{n}{pyc}
    \PYG{o}{/}\PYG{n}{usr}\PYG{o}{/}\PYG{n}{share}\PYG{o}{/}\PYG{n}{sosreport}\PYG{o}{/}\PYG{n}{sos}\PYG{o}{/}\PYG{n}{plugins}\PYG{o}{/}\PYG{n}{apache}\PYG{o}{.}\PYG{n}{py}
    \PYG{o}{/}\PYG{n}{usr}\PYG{o}{/}\PYG{n}{share}\PYG{o}{/}\PYG{n}{apport}\PYG{o}{/}\PYG{n}{package}\PYG{o}{\PYGZhy{}}\PYG{n}{hooks}\PYG{o}{/}\PYG{n}{apache2}\PYG{o}{.}\PYG{n}{py}
    \PYG{o}{/}\PYG{n}{usr}\PYG{o}{/}\PYG{n}{share}\PYG{o}{/}\PYG{n}{bash}\PYG{o}{\PYGZhy{}}\PYG{n}{completion}\PYG{o}{/}\PYG{n}{completions}\PYG{o}{/}\PYG{n}{apache2ctl}
\PYG{n}{find} \PYG{o}{/}\PYG{n}{lib} \PYG{o}{\PYGZhy{}}\PYG{n}{name} \PYG{l+s+s2}{\PYGZdq{}}\PYG{l+s+s2}{*apache*}\PYG{l+s+s2}{\PYGZdq{}}
    \PYG{o}{/}\PYG{n}{lib}\PYG{o}{/}\PYG{n}{systemd}\PYG{o}{/}\PYG{n}{system}\PYG{o}{/}\PYG{n}{apache2}\PYG{o}{.}\PYG{n}{service}\PYG{o}{.}\PYG{n}{d}
    \PYG{o}{/}\PYG{n}{lib}\PYG{o}{/}\PYG{n}{systemd}\PYG{o}{/}\PYG{n}{system}\PYG{o}{/}\PYG{n}{apache2}\PYG{o}{.}\PYG{n}{service}\PYG{o}{.}\PYG{n}{d}\PYG{o}{/}\PYG{n}{apache2}\PYG{o}{\PYGZhy{}}\PYG{n}{systemd}\PYG{o}{.}\PYG{n}{conf}
\PYG{n}{dpkg} \PYG{o}{\PYGZhy{}}\PYG{n}{l} \PYG{o}{\textbar{}} \PYG{n}{grep} \PYG{n}{apache}
    \PYG{n}{ii}  \PYG{n}{apache2}          \PYG{l+m+mf}{2.4}\PYG{o}{.}\PYG{l+m+mi}{18}\PYG{o}{\PYGZhy{}}\PYG{l+m+mi}{2}\PYG{n}{ubuntu3}\PYG{o}{.}\PYG{l+m+mi}{4}    \PYG{n}{amd64}        \PYG{n}{Apache} \PYG{n}{HTTP} \PYG{n}{Server}
    \PYG{n}{ii}  \PYG{n}{apache2}\PYG{o}{\PYGZhy{}}\PYG{n+nb}{bin}      \PYG{l+m+mf}{2.4}\PYG{o}{.}\PYG{l+m+mi}{18}\PYG{o}{\PYGZhy{}}\PYG{l+m+mi}{2}\PYG{n}{ubuntu3}\PYG{o}{.}\PYG{l+m+mi}{4}    \PYG{n}{amd64}        \PYG{n}{Apache} \PYG{n}{HTTP} \PYG{n}{Server} \PYG{p}{(}\PYG{n}{modules} \PYG{o+ow}{and} \PYG{n}{other} \PYG{n}{binary} \PYG{n}{files}\PYG{p}{)}
    \PYG{n}{ii}  \PYG{n}{apache2}\PYG{o}{\PYGZhy{}}\PYG{n}{data}     \PYG{l+m+mf}{2.4}\PYG{o}{.}\PYG{l+m+mi}{18}\PYG{o}{\PYGZhy{}}\PYG{l+m+mi}{2}\PYG{n}{ubuntu3}\PYG{o}{.}\PYG{l+m+mi}{4}    \PYG{n+nb}{all}          \PYG{n}{Apache} \PYG{n}{HTTP} \PYG{n}{Server} \PYG{p}{(}\PYG{n}{common} \PYG{n}{files}\PYG{p}{)}
    \PYG{n}{ii}  \PYG{n}{apache2}\PYG{o}{\PYGZhy{}}\PYG{n}{utils}    \PYG{l+m+mf}{2.4}\PYG{o}{.}\PYG{l+m+mi}{18}\PYG{o}{\PYGZhy{}}\PYG{l+m+mi}{2}\PYG{n}{ubuntu3}\PYG{o}{.}\PYG{l+m+mi}{4}    \PYG{n}{amd64}        \PYG{n}{Apache} \PYG{n}{HTTP} \PYG{n}{Server} \PYG{p}{(}\PYG{n}{utility} \PYG{n}{programs} \PYG{k}{for} \PYG{n}{web} \PYG{n}{servers}\PYG{p}{)}
\end{sphinxVerbatim}

\item {} 
remove apache

\begin{sphinxVerbatim}[commandchars=\\\{\}]
\PYG{n}{apt} \PYG{n}{autoremove} \PYG{o}{\PYGZhy{}}\PYG{o}{\PYGZhy{}}\PYG{n}{purge} \PYG{n}{apache2} \PYG{o}{\PYGZhy{}}\PYG{n}{y} \PYG{n}{满足需求}
\end{sphinxVerbatim}

\end{itemize}


\subsubsection{install\&remove php}
\label{\detokenize{linux/apt:install-remove-php}}\begin{itemize}
\item {} 
install php

\begin{sphinxVerbatim}[commandchars=\\\{\}]
\PYG{n}{apt} \PYG{n}{install} \PYG{n}{apache2} \PYG{o}{\PYGZhy{}}\PYG{n}{y}                  \PYG{n}{安装apache2}
\PYG{n}{apt} \PYG{n}{install} \PYG{n}{mysql}\PYG{o}{\PYGZhy{}}\PYG{n}{server} \PYG{o}{\PYGZhy{}}\PYG{n}{y}             \PYG{n}{安装mysql}
\PYG{n}{apt} \PYG{n}{install} \PYG{n}{php7}\PYG{o}{.}\PYG{l+m+mi}{0} \PYG{o}{\PYGZhy{}}\PYG{n}{y}
\PYG{n}{find} \PYG{o}{/}\PYG{n}{etc} \PYG{o}{\PYGZhy{}}\PYG{n}{name} \PYG{l+s+s2}{\PYGZdq{}}\PYG{l+s+s2}{*php*}\PYG{l+s+s2}{\PYGZdq{}}
    \PYG{o}{/}\PYG{n}{etc}\PYG{o}{/}\PYG{n}{logrotate}\PYG{o}{.}\PYG{n}{d}\PYG{o}{/}\PYG{n}{php7}\PYG{o}{.}\PYG{l+m+mi}{0}\PYG{o}{\PYGZhy{}}\PYG{n}{fpm}
    \PYG{o}{/}\PYG{n}{etc}\PYG{o}{/}\PYG{n}{rc1}\PYG{o}{.}\PYG{n}{d}\PYG{o}{/}\PYG{n}{K01php7}\PYG{o}{.}\PYG{l+m+mi}{0}\PYG{o}{\PYGZhy{}}\PYG{n}{fpm}
    \PYG{o}{/}\PYG{n}{etc}\PYG{o}{/}\PYG{n}{rc5}\PYG{o}{.}\PYG{n}{d}\PYG{o}{/}\PYG{n}{S01php7}\PYG{o}{.}\PYG{l+m+mi}{0}\PYG{o}{\PYGZhy{}}\PYG{n}{fpm}
    \PYG{o}{/}\PYG{n}{etc}\PYG{o}{/}\PYG{n}{alternatives}\PYG{o}{/}\PYG{n}{php}\PYG{o}{.}\PYG{l+m+mf}{1.}\PYG{n}{gz}
    \PYG{o}{/}\PYG{n}{etc}\PYG{o}{/}\PYG{n}{alternatives}\PYG{o}{/}\PYG{n}{php}
    \PYG{o}{/}\PYG{n}{etc}\PYG{o}{/}\PYG{n}{init}\PYG{o}{.}\PYG{n}{d}\PYG{o}{/}\PYG{n}{php7}\PYG{o}{.}\PYG{l+m+mi}{0}\PYG{o}{\PYGZhy{}}\PYG{n}{fpm}
    \PYG{o}{/}\PYG{n}{etc}\PYG{o}{/}\PYG{n}{rc3}\PYG{o}{.}\PYG{n}{d}\PYG{o}{/}\PYG{n}{S01php7}\PYG{o}{.}\PYG{l+m+mi}{0}\PYG{o}{\PYGZhy{}}\PYG{n}{fpm}
    \PYG{o}{/}\PYG{n}{etc}\PYG{o}{/}\PYG{n}{apache2}\PYG{o}{/}\PYG{n}{conf}\PYG{o}{\PYGZhy{}}\PYG{n}{available}\PYG{o}{/}\PYG{n}{php7}\PYG{o}{.}\PYG{l+m+mi}{0}\PYG{o}{\PYGZhy{}}\PYG{n}{fpm}\PYG{o}{.}\PYG{n}{conf}
    \PYG{o}{/}\PYG{n}{etc}\PYG{o}{/}\PYG{n}{rc2}\PYG{o}{.}\PYG{n}{d}\PYG{o}{/}\PYG{n}{S01php7}\PYG{o}{.}\PYG{l+m+mi}{0}\PYG{o}{\PYGZhy{}}\PYG{n}{fpm}
    \PYG{o}{/}\PYG{n}{etc}\PYG{o}{/}\PYG{n}{rc0}\PYG{o}{.}\PYG{n}{d}\PYG{o}{/}\PYG{n}{K01php7}\PYG{o}{.}\PYG{l+m+mi}{0}\PYG{o}{\PYGZhy{}}\PYG{n}{fpm}
    \PYG{o}{/}\PYG{n}{etc}\PYG{o}{/}\PYG{n}{php}
    \PYG{o}{/}\PYG{n}{etc}\PYG{o}{/}\PYG{n}{php}\PYG{o}{/}\PYG{l+m+mf}{7.0}\PYG{o}{/}\PYG{n}{fpm}\PYG{o}{/}\PYG{n}{php}\PYG{o}{\PYGZhy{}}\PYG{n}{fpm}\PYG{o}{.}\PYG{n}{conf}
    \PYG{o}{/}\PYG{n}{etc}\PYG{o}{/}\PYG{n}{php}\PYG{o}{/}\PYG{l+m+mf}{7.0}\PYG{o}{/}\PYG{n}{fpm}\PYG{o}{/}\PYG{n}{php}\PYG{o}{.}\PYG{n}{ini}
    \PYG{o}{/}\PYG{n}{etc}\PYG{o}{/}\PYG{n}{php}\PYG{o}{/}\PYG{l+m+mf}{7.0}\PYG{o}{/}\PYG{n}{cli}\PYG{o}{/}\PYG{n}{php}\PYG{o}{.}\PYG{n}{ini}
    \PYG{o}{/}\PYG{n}{etc}\PYG{o}{/}\PYG{n}{init}\PYG{o}{/}\PYG{n}{php7}\PYG{o}{.}\PYG{l+m+mi}{0}\PYG{o}{\PYGZhy{}}\PYG{n}{fpm}\PYG{o}{.}\PYG{n}{conf}
    \PYG{o}{/}\PYG{n}{etc}\PYG{o}{/}\PYG{n}{apparmor}\PYG{o}{.}\PYG{n}{d}\PYG{o}{/}\PYG{n}{abstractions}\PYG{o}{/}\PYG{n}{php5}
    \PYG{o}{/}\PYG{n}{etc}\PYG{o}{/}\PYG{n}{systemd}\PYG{o}{/}\PYG{n}{system}\PYG{o}{/}\PYG{n}{multi}\PYG{o}{\PYGZhy{}}\PYG{n}{user}\PYG{o}{.}\PYG{n}{target}\PYG{o}{.}\PYG{n}{wants}\PYG{o}{/}\PYG{n}{php7}\PYG{o}{.}\PYG{l+m+mi}{0}\PYG{o}{\PYGZhy{}}\PYG{n}{fpm}\PYG{o}{.}\PYG{n}{service}
    \PYG{o}{/}\PYG{n}{etc}\PYG{o}{/}\PYG{n}{rc4}\PYG{o}{.}\PYG{n}{d}\PYG{o}{/}\PYG{n}{S01php7}\PYG{o}{.}\PYG{l+m+mi}{0}\PYG{o}{\PYGZhy{}}\PYG{n}{fpm}
    \PYG{o}{/}\PYG{n}{etc}\PYG{o}{/}\PYG{n}{cron}\PYG{o}{.}\PYG{n}{d}\PYG{o}{/}\PYG{n}{php}
    \PYG{o}{/}\PYG{n}{etc}\PYG{o}{/}\PYG{n}{rc6}\PYG{o}{.}\PYG{n}{d}\PYG{o}{/}\PYG{n}{K01php7}\PYG{o}{.}\PYG{l+m+mi}{0}\PYG{o}{\PYGZhy{}}\PYG{n}{fpm}
\PYG{n}{find} \PYG{o}{/}\PYG{n}{usr} \PYG{o}{\PYGZhy{}}\PYG{n}{name} \PYG{l+s+s2}{\PYGZdq{}}\PYG{l+s+s2}{*php*}\PYG{l+s+s2}{\PYGZdq{}}
    \PYG{o}{/}\PYG{n}{usr}\PYG{o}{/}\PYG{n}{lib}\PYG{o}{/}\PYG{n}{tmpfiles}\PYG{o}{.}\PYG{n}{d}\PYG{o}{/}\PYG{n}{php7}\PYG{o}{.}\PYG{l+m+mi}{0}\PYG{o}{\PYGZhy{}}\PYG{n}{fpm}\PYG{o}{.}\PYG{n}{conf}
    \PYG{o}{/}\PYG{n}{usr}\PYG{o}{/}\PYG{n}{lib}\PYG{o}{/}\PYG{n}{php}
    \PYG{o}{/}\PYG{n}{usr}\PYG{o}{/}\PYG{n}{lib}\PYG{o}{/}\PYG{n}{php}\PYG{o}{/}\PYG{n}{php7}\PYG{o}{.}\PYG{l+m+mi}{0}\PYG{o}{\PYGZhy{}}\PYG{n}{fpm}\PYG{o}{\PYGZhy{}}\PYG{n}{reopenlogs}
    \PYG{o}{/}\PYG{n}{usr}\PYG{o}{/}\PYG{n}{lib}\PYG{o}{/}\PYG{n}{php}\PYG{o}{/}\PYG{n}{php}\PYG{o}{\PYGZhy{}}\PYG{n}{maintscript}\PYG{o}{\PYGZhy{}}\PYG{n}{helper}
    \PYG{o}{/}\PYG{n}{usr}\PYG{o}{/}\PYG{n}{lib}\PYG{o}{/}\PYG{n}{php}\PYG{o}{/}\PYG{n}{php7}\PYG{o}{.}\PYG{l+m+mi}{0}\PYG{o}{\PYGZhy{}}\PYG{n}{fpm}\PYG{o}{\PYGZhy{}}\PYG{n}{checkconf}
    \PYG{o}{/}\PYG{n}{usr}\PYG{o}{/}\PYG{n}{lib}\PYG{o}{/}\PYG{n}{php}\PYG{o}{/}\PYG{n}{php}\PYG{o}{\PYGZhy{}}\PYG{n}{helper}
    \PYG{o}{/}\PYG{n}{usr}\PYG{o}{/}\PYG{n}{lib}\PYG{o}{/}\PYG{n}{php}\PYG{o}{/}\PYG{l+m+mf}{7.0}\PYG{o}{/}\PYG{n}{php}\PYG{o}{.}\PYG{n}{ini}\PYG{o}{\PYGZhy{}}\PYG{n}{production}
    \PYG{o}{/}\PYG{n}{usr}\PYG{o}{/}\PYG{n}{lib}\PYG{o}{/}\PYG{n}{php}\PYG{o}{/}\PYG{l+m+mf}{7.0}\PYG{o}{/}\PYG{n}{php}\PYG{o}{.}\PYG{n}{ini}\PYG{o}{\PYGZhy{}}\PYG{n}{production}\PYG{o}{.}\PYG{n}{cli}
    \PYG{o}{/}\PYG{n}{usr}\PYG{o}{/}\PYG{n}{lib}\PYG{o}{/}\PYG{n}{php}\PYG{o}{/}\PYG{l+m+mf}{7.0}\PYG{o}{/}\PYG{n}{php}\PYG{o}{.}\PYG{n}{ini}\PYG{o}{\PYGZhy{}}\PYG{n}{development}
    \PYG{o}{/}\PYG{n}{usr}\PYG{o}{/}\PYG{n}{sbin}\PYG{o}{/}\PYG{n}{phpenmod}
    \PYG{o}{/}\PYG{n}{usr}\PYG{o}{/}\PYG{n}{sbin}\PYG{o}{/}\PYG{n}{phpdismod}
    \PYG{o}{/}\PYG{n}{usr}\PYG{o}{/}\PYG{n}{sbin}\PYG{o}{/}\PYG{n}{phpquery}
    \PYG{o}{/}\PYG{n}{usr}\PYG{o}{/}\PYG{n}{sbin}\PYG{o}{/}\PYG{n}{php}\PYG{o}{\PYGZhy{}}\PYG{n}{fpm7}\PYG{o}{.}\PYG{l+m+mi}{0}
    \PYG{o}{/}\PYG{n}{usr}\PYG{o}{/}\PYG{n+nb}{bin}\PYG{o}{/}\PYG{n}{php7}\PYG{o}{.}\PYG{l+m+mi}{0}
    \PYG{o}{/}\PYG{n}{usr}\PYG{o}{/}\PYG{n+nb}{bin}\PYG{o}{/}\PYG{n}{php}
    \PYG{o}{/}\PYG{n}{usr}\PYG{o}{/}\PYG{n}{share}\PYG{o}{/}\PYG{n}{doc}\PYG{o}{/}\PYG{n}{php7}\PYG{o}{.}\PYG{l+m+mi}{0}
    \PYG{o}{/}\PYG{n}{usr}\PYG{o}{/}\PYG{n}{share}\PYG{o}{/}\PYG{n}{doc}\PYG{o}{/}\PYG{n}{php7}\PYG{o}{.}\PYG{l+m+mi}{0}\PYG{o}{\PYGZhy{}}\PYG{n}{readline}
    \PYG{o}{/}\PYG{n}{usr}\PYG{o}{/}\PYG{n}{share}\PYG{o}{/}\PYG{n}{doc}\PYG{o}{/}\PYG{n}{php}\PYG{o}{\PYGZhy{}}\PYG{n}{common}
    \PYG{o}{/}\PYG{n}{usr}\PYG{o}{/}\PYG{n}{share}\PYG{o}{/}\PYG{n}{doc}\PYG{o}{/}\PYG{n}{php7}\PYG{o}{.}\PYG{l+m+mi}{0}\PYG{o}{\PYGZhy{}}\PYG{n}{json}
    \PYG{o}{/}\PYG{n}{usr}\PYG{o}{/}\PYG{n}{share}\PYG{o}{/}\PYG{n}{doc}\PYG{o}{/}\PYG{n}{php7}\PYG{o}{.}\PYG{l+m+mi}{0}\PYG{o}{\PYGZhy{}}\PYG{n}{cli}
    \PYG{o}{/}\PYG{n}{usr}\PYG{o}{/}\PYG{n}{share}\PYG{o}{/}\PYG{n}{doc}\PYG{o}{/}\PYG{n}{php7}\PYG{o}{.}\PYG{l+m+mi}{0}\PYG{o}{\PYGZhy{}}\PYG{n}{fpm}
    \PYG{o}{/}\PYG{n}{usr}\PYG{o}{/}\PYG{n}{share}\PYG{o}{/}\PYG{n}{doc}\PYG{o}{/}\PYG{n}{php7}\PYG{o}{.}\PYG{l+m+mi}{0}\PYG{o}{\PYGZhy{}}\PYG{n}{common}
    \PYG{o}{/}\PYG{n}{usr}\PYG{o}{/}\PYG{n}{share}\PYG{o}{/}\PYG{n}{doc}\PYG{o}{/}\PYG{n}{kamailio}\PYG{o}{/}\PYG{n}{examples}\PYG{o}{/}\PYG{n}{web\PYGZus{}im}\PYG{o}{/}\PYG{n}{click\PYGZus{}to\PYGZus{}dial}\PYG{o}{.}\PYG{n}{php}
    \PYG{o}{/}\PYG{n}{usr}\PYG{o}{/}\PYG{n}{share}\PYG{o}{/}\PYG{n}{doc}\PYG{o}{/}\PYG{n}{kamailio}\PYG{o}{/}\PYG{n}{examples}\PYG{o}{/}\PYG{n}{web\PYGZus{}im}\PYG{o}{/}\PYG{n}{send\PYGZus{}im}\PYG{o}{.}\PYG{n}{php}
    \PYG{o}{/}\PYG{n}{usr}\PYG{o}{/}\PYG{n}{share}\PYG{o}{/}\PYG{n}{doc}\PYG{o}{/}\PYG{n}{kamailio}\PYG{o}{/}\PYG{n}{examples}\PYG{o}{/}\PYG{n}{kamailio}\PYG{o}{/}\PYG{n}{web\PYGZus{}im}\PYG{o}{/}\PYG{n}{click\PYGZus{}to\PYGZus{}dial}\PYG{o}{.}\PYG{n}{php}
    \PYG{o}{/}\PYG{n}{usr}\PYG{o}{/}\PYG{n}{share}\PYG{o}{/}\PYG{n}{doc}\PYG{o}{/}\PYG{n}{kamailio}\PYG{o}{/}\PYG{n}{examples}\PYG{o}{/}\PYG{n}{kamailio}\PYG{o}{/}\PYG{n}{web\PYGZus{}im}\PYG{o}{/}\PYG{n}{send\PYGZus{}im}\PYG{o}{.}\PYG{n}{php}
    \PYG{o}{/}\PYG{n}{usr}\PYG{o}{/}\PYG{n}{share}\PYG{o}{/}\PYG{n}{doc}\PYG{o}{/}\PYG{n}{php7}\PYG{o}{.}\PYG{l+m+mi}{0}\PYG{o}{\PYGZhy{}}\PYG{n}{opcache}
    \PYG{o}{/}\PYG{n}{usr}\PYG{o}{/}\PYG{n}{share}\PYG{o}{/}\PYG{n}{php7}\PYG{o}{.}\PYG{l+m+mi}{0}\PYG{o}{\PYGZhy{}}\PYG{n}{readline}
    \PYG{o}{/}\PYG{n}{usr}\PYG{o}{/}\PYG{n}{share}\PYG{o}{/}\PYG{n}{vim}\PYG{o}{/}\PYG{n}{vim74}\PYG{o}{/}\PYG{n}{autoload}\PYG{o}{/}\PYG{n}{phpcomplete}\PYG{o}{.}\PYG{n}{vim}
    \PYG{o}{/}\PYG{n}{usr}\PYG{o}{/}\PYG{n}{share}\PYG{o}{/}\PYG{n}{vim}\PYG{o}{/}\PYG{n}{vim74}\PYG{o}{/}\PYG{n}{indent}\PYG{o}{/}\PYG{n}{php}\PYG{o}{.}\PYG{n}{vim}
    \PYG{o}{/}\PYG{n}{usr}\PYG{o}{/}\PYG{n}{share}\PYG{o}{/}\PYG{n}{vim}\PYG{o}{/}\PYG{n}{vim74}\PYG{o}{/}\PYG{n}{syntax}\PYG{o}{/}\PYG{n}{php}\PYG{o}{.}\PYG{n}{vim}
    \PYG{o}{/}\PYG{n}{usr}\PYG{o}{/}\PYG{n}{share}\PYG{o}{/}\PYG{n}{vim}\PYG{o}{/}\PYG{n}{vim74}\PYG{o}{/}\PYG{n}{ftplugin}\PYG{o}{/}\PYG{n}{php}\PYG{o}{.}\PYG{n}{vim}
    \PYG{o}{/}\PYG{n}{usr}\PYG{o}{/}\PYG{n}{share}\PYG{o}{/}\PYG{n}{vim}\PYG{o}{/}\PYG{n}{vim74}\PYG{o}{/}\PYG{n}{compiler}\PYG{o}{/}\PYG{n}{php}\PYG{o}{.}\PYG{n}{vim}
    \PYG{o}{/}\PYG{n}{usr}\PYG{o}{/}\PYG{n}{share}\PYG{o}{/}\PYG{n}{php7}\PYG{o}{.}\PYG{l+m+mi}{0}\PYG{o}{\PYGZhy{}}\PYG{n}{json}
    \PYG{o}{/}\PYG{n}{usr}\PYG{o}{/}\PYG{n}{share}\PYG{o}{/}\PYG{n}{nano}\PYG{o}{/}\PYG{n}{php}\PYG{o}{.}\PYG{n}{nanorc}
    \PYG{o}{/}\PYG{n}{usr}\PYG{o}{/}\PYG{n}{share}\PYG{o}{/}\PYG{n}{bug}\PYG{o}{/}\PYG{n}{php7}\PYG{o}{.}\PYG{l+m+mi}{0}
    \PYG{o}{/}\PYG{n}{usr}\PYG{o}{/}\PYG{n}{share}\PYG{o}{/}\PYG{n}{bug}\PYG{o}{/}\PYG{n}{php7}\PYG{o}{.}\PYG{l+m+mi}{0}\PYG{o}{\PYGZhy{}}\PYG{n}{readline}
    \PYG{o}{/}\PYG{n}{usr}\PYG{o}{/}\PYG{n}{share}\PYG{o}{/}\PYG{n}{bug}\PYG{o}{/}\PYG{n}{php7}\PYG{o}{.}\PYG{l+m+mi}{0}\PYG{o}{\PYGZhy{}}\PYG{n}{json}
    \PYG{o}{/}\PYG{n}{usr}\PYG{o}{/}\PYG{n}{share}\PYG{o}{/}\PYG{n}{bug}\PYG{o}{/}\PYG{n}{php7}\PYG{o}{.}\PYG{l+m+mi}{0}\PYG{o}{\PYGZhy{}}\PYG{n}{cli}
    \PYG{o}{/}\PYG{n}{usr}\PYG{o}{/}\PYG{n}{share}\PYG{o}{/}\PYG{n}{bug}\PYG{o}{/}\PYG{n}{php7}\PYG{o}{.}\PYG{l+m+mi}{0}\PYG{o}{\PYGZhy{}}\PYG{n}{fpm}
    \PYG{o}{/}\PYG{n}{usr}\PYG{o}{/}\PYG{n}{share}\PYG{o}{/}\PYG{n}{bug}\PYG{o}{/}\PYG{n}{php7}\PYG{o}{.}\PYG{l+m+mi}{0}\PYG{o}{\PYGZhy{}}\PYG{n}{common}
    \PYG{o}{/}\PYG{n}{usr}\PYG{o}{/}\PYG{n}{share}\PYG{o}{/}\PYG{n}{bug}\PYG{o}{/}\PYG{n}{php7}\PYG{o}{.}\PYG{l+m+mi}{0}\PYG{o}{\PYGZhy{}}\PYG{n}{opcache}
    \PYG{o}{/}\PYG{n}{usr}\PYG{o}{/}\PYG{n}{share}\PYG{o}{/}\PYG{n}{mime}\PYG{o}{/}\PYG{n}{application}\PYG{o}{/}\PYG{n}{x}\PYG{o}{\PYGZhy{}}\PYG{n}{php}\PYG{o}{.}\PYG{n}{xml}
    \PYG{o}{/}\PYG{n}{usr}\PYG{o}{/}\PYG{n}{share}\PYG{o}{/}\PYG{n}{php}
    \PYG{o}{/}\PYG{n}{usr}\PYG{o}{/}\PYG{n}{share}\PYG{o}{/}\PYG{n}{lintian}\PYG{o}{/}\PYG{n}{overrides}\PYG{o}{/}\PYG{n}{php7}\PYG{o}{.}\PYG{l+m+mi}{0}\PYG{o}{\PYGZhy{}}\PYG{n}{readline}
    \PYG{o}{/}\PYG{n}{usr}\PYG{o}{/}\PYG{n}{share}\PYG{o}{/}\PYG{n}{lintian}\PYG{o}{/}\PYG{n}{overrides}\PYG{o}{/}\PYG{n}{php7}\PYG{o}{.}\PYG{l+m+mi}{0}\PYG{o}{\PYGZhy{}}\PYG{n}{json}
    \PYG{o}{/}\PYG{n}{usr}\PYG{o}{/}\PYG{n}{share}\PYG{o}{/}\PYG{n}{lintian}\PYG{o}{/}\PYG{n}{overrides}\PYG{o}{/}\PYG{n}{php7}\PYG{o}{.}\PYG{l+m+mi}{0}\PYG{o}{\PYGZhy{}}\PYG{n}{cli}
    \PYG{o}{/}\PYG{n}{usr}\PYG{o}{/}\PYG{n}{share}\PYG{o}{/}\PYG{n}{lintian}\PYG{o}{/}\PYG{n}{overrides}\PYG{o}{/}\PYG{n}{php7}\PYG{o}{.}\PYG{l+m+mi}{0}\PYG{o}{\PYGZhy{}}\PYG{n}{fpm}
    \PYG{o}{/}\PYG{n}{usr}\PYG{o}{/}\PYG{n}{share}\PYG{o}{/}\PYG{n}{lintian}\PYG{o}{/}\PYG{n}{overrides}\PYG{o}{/}\PYG{n}{php7}\PYG{o}{.}\PYG{l+m+mi}{0}\PYG{o}{\PYGZhy{}}\PYG{n}{common}
    \PYG{o}{/}\PYG{n}{usr}\PYG{o}{/}\PYG{n}{share}\PYG{o}{/}\PYG{n}{lintian}\PYG{o}{/}\PYG{n}{overrides}\PYG{o}{/}\PYG{n}{php7}\PYG{o}{.}\PYG{l+m+mi}{0}\PYG{o}{\PYGZhy{}}\PYG{n}{opcache}
    \PYG{o}{/}\PYG{n}{usr}\PYG{o}{/}\PYG{n}{share}\PYG{o}{/}\PYG{n}{php7}\PYG{o}{.}\PYG{l+m+mi}{0}\PYG{o}{\PYGZhy{}}\PYG{n}{common}
    \PYG{o}{/}\PYG{n}{usr}\PYG{o}{/}\PYG{n}{share}\PYG{o}{/}\PYG{n}{man}\PYG{o}{/}\PYG{n}{man1}\PYG{o}{/}\PYG{n}{php}\PYG{o}{.}\PYG{l+m+mf}{1.}\PYG{n}{gz}
    \PYG{o}{/}\PYG{n}{usr}\PYG{o}{/}\PYG{n}{share}\PYG{o}{/}\PYG{n}{man}\PYG{o}{/}\PYG{n}{man1}\PYG{o}{/}\PYG{n}{php7}\PYG{o}{.}\PYG{l+m+mf}{0.1}\PYG{o}{.}\PYG{n}{gz}
    \PYG{o}{/}\PYG{n}{usr}\PYG{o}{/}\PYG{n}{share}\PYG{o}{/}\PYG{n}{man}\PYG{o}{/}\PYG{n}{man8}\PYG{o}{/}\PYG{n}{php}\PYG{o}{\PYGZhy{}}\PYG{n}{fpm7}\PYG{o}{.}\PYG{l+m+mf}{0.8}\PYG{o}{.}\PYG{n}{gz}
    \PYG{o}{/}\PYG{n}{usr}\PYG{o}{/}\PYG{n}{share}\PYG{o}{/}\PYG{n}{php7}\PYG{o}{.}\PYG{l+m+mi}{0}\PYG{o}{\PYGZhy{}}\PYG{n}{opcache}
\PYG{n}{find} \PYG{o}{/}\PYG{n}{lib} \PYG{o}{\PYGZhy{}}\PYG{n}{name} \PYG{l+s+s2}{\PYGZdq{}}\PYG{l+s+s2}{*php*}\PYG{l+s+s2}{\PYGZdq{}}
    \PYG{o}{/}\PYG{n}{lib}\PYG{o}{/}\PYG{n}{modules}\PYG{o}{/}\PYG{l+m+mf}{4.4}\PYG{o}{.}\PYG{l+m+mi}{0}\PYG{o}{\PYGZhy{}}\PYG{l+m+mi}{92}\PYG{o}{\PYGZhy{}}\PYG{n}{generic}\PYG{o}{/}\PYG{n}{kernel}\PYG{o}{/}\PYG{n}{drivers}\PYG{o}{/}\PYG{n}{pci}\PYG{o}{/}\PYG{n}{hotplug}\PYG{o}{/}\PYG{n}{acpiphp\PYGZus{}ibm}\PYG{o}{.}\PYG{n}{ko}
    \PYG{o}{/}\PYG{n}{lib}\PYG{o}{/}\PYG{n}{modules}\PYG{o}{/}\PYG{l+m+mf}{4.4}\PYG{o}{.}\PYG{l+m+mi}{0}\PYG{o}{\PYGZhy{}}\PYG{l+m+mi}{91}\PYG{o}{\PYGZhy{}}\PYG{n}{generic}\PYG{o}{/}\PYG{n}{kernel}\PYG{o}{/}\PYG{n}{drivers}\PYG{o}{/}\PYG{n}{pci}\PYG{o}{/}\PYG{n}{hotplug}\PYG{o}{/}\PYG{n}{acpiphp\PYGZus{}ibm}\PYG{o}{.}\PYG{n}{ko}
    \PYG{o}{/}\PYG{n}{lib}\PYG{o}{/}\PYG{n}{modules}\PYG{o}{/}\PYG{l+m+mf}{4.4}\PYG{o}{.}\PYG{l+m+mi}{0}\PYG{o}{\PYGZhy{}}\PYG{l+m+mi}{93}\PYG{o}{\PYGZhy{}}\PYG{n}{generic}\PYG{o}{/}\PYG{n}{kernel}\PYG{o}{/}\PYG{n}{drivers}\PYG{o}{/}\PYG{n}{pci}\PYG{o}{/}\PYG{n}{hotplug}\PYG{o}{/}\PYG{n}{acpiphp\PYGZus{}ibm}\PYG{o}{.}\PYG{n}{ko}
    \PYG{o}{/}\PYG{n}{lib}\PYG{o}{/}\PYG{n}{systemd}\PYG{o}{/}\PYG{n}{system}\PYG{o}{/}\PYG{n}{php7}\PYG{o}{.}\PYG{l+m+mi}{0}\PYG{o}{\PYGZhy{}}\PYG{n}{fpm}\PYG{o}{.}\PYG{n}{service}
\PYG{n}{dpkg} \PYG{o}{\PYGZhy{}}\PYG{n}{l} \PYG{o}{\textbar{}} \PYG{n}{grep} \PYG{n}{php}
    \PYG{n}{ii}  \PYG{n}{php}\PYG{o}{\PYGZhy{}}\PYG{n}{common}                         \PYG{l+m+mi}{1}\PYG{p}{:}\PYG{l+m+mi}{35}\PYG{n}{ubuntu6}                                \PYG{n+nb}{all}          \PYG{n}{Common} \PYG{n}{files} \PYG{k}{for} \PYG{n}{PHP} \PYG{n}{packages}
    \PYG{n}{ii}  \PYG{n}{php7}\PYG{o}{.}\PYG{l+m+mi}{0}                             \PYG{l+m+mf}{7.0}\PYG{o}{.}\PYG{l+m+mi}{22}\PYG{o}{\PYGZhy{}}\PYG{l+m+mi}{0}\PYG{n}{ubuntu0}\PYG{o}{.}\PYG{l+m+mf}{16.04}\PYG{o}{.}\PYG{l+m+mi}{1}                    \PYG{n+nb}{all}          \PYG{n}{server}\PYG{o}{\PYGZhy{}}\PYG{n}{side}\PYG{p}{,} \PYG{n}{HTML}\PYG{o}{\PYGZhy{}}\PYG{n}{embedded} \PYG{n}{scripting} \PYG{n}{language} \PYG{p}{(}\PYG{n}{metapackage}\PYG{p}{)}
    \PYG{n}{ii}  \PYG{n}{php7}\PYG{o}{.}\PYG{l+m+mi}{0}\PYG{o}{\PYGZhy{}}\PYG{n}{cli}                         \PYG{l+m+mf}{7.0}\PYG{o}{.}\PYG{l+m+mi}{22}\PYG{o}{\PYGZhy{}}\PYG{l+m+mi}{0}\PYG{n}{ubuntu0}\PYG{o}{.}\PYG{l+m+mf}{16.04}\PYG{o}{.}\PYG{l+m+mi}{1}                    \PYG{n}{amd64}        \PYG{n}{command}\PYG{o}{\PYGZhy{}}\PYG{n}{line} \PYG{n}{interpreter} \PYG{k}{for} \PYG{n}{the} \PYG{n}{PHP} \PYG{n}{scripting} \PYG{n}{language}
    \PYG{n}{ii}  \PYG{n}{php7}\PYG{o}{.}\PYG{l+m+mi}{0}\PYG{o}{\PYGZhy{}}\PYG{n}{common}                      \PYG{l+m+mf}{7.0}\PYG{o}{.}\PYG{l+m+mi}{22}\PYG{o}{\PYGZhy{}}\PYG{l+m+mi}{0}\PYG{n}{ubuntu0}\PYG{o}{.}\PYG{l+m+mf}{16.04}\PYG{o}{.}\PYG{l+m+mi}{1}                    \PYG{n}{amd64}        \PYG{n}{documentation}\PYG{p}{,} \PYG{n}{examples} \PYG{o+ow}{and} \PYG{n}{common} \PYG{n}{module} \PYG{k}{for} \PYG{n}{PHP}
    \PYG{n}{ii}  \PYG{n}{php7}\PYG{o}{.}\PYG{l+m+mi}{0}\PYG{o}{\PYGZhy{}}\PYG{n}{fpm}                         \PYG{l+m+mf}{7.0}\PYG{o}{.}\PYG{l+m+mi}{22}\PYG{o}{\PYGZhy{}}\PYG{l+m+mi}{0}\PYG{n}{ubuntu0}\PYG{o}{.}\PYG{l+m+mf}{16.04}\PYG{o}{.}\PYG{l+m+mi}{1}                    \PYG{n}{amd64}        \PYG{n}{server}\PYG{o}{\PYGZhy{}}\PYG{n}{side}\PYG{p}{,} \PYG{n}{HTML}\PYG{o}{\PYGZhy{}}\PYG{n}{embedded} \PYG{n}{scripting} \PYG{n}{language} \PYG{p}{(}\PYG{n}{FPM}\PYG{o}{\PYGZhy{}}\PYG{n}{CGI} \PYG{n}{binary}\PYG{p}{)}
    \PYG{n}{ii}  \PYG{n}{php7}\PYG{o}{.}\PYG{l+m+mi}{0}\PYG{o}{\PYGZhy{}}\PYG{n}{json}                        \PYG{l+m+mf}{7.0}\PYG{o}{.}\PYG{l+m+mi}{22}\PYG{o}{\PYGZhy{}}\PYG{l+m+mi}{0}\PYG{n}{ubuntu0}\PYG{o}{.}\PYG{l+m+mf}{16.04}\PYG{o}{.}\PYG{l+m+mi}{1}                    \PYG{n}{amd64}        \PYG{n}{JSON} \PYG{n}{module} \PYG{k}{for} \PYG{n}{PHP}
    \PYG{n}{ii}  \PYG{n}{php7}\PYG{o}{.}\PYG{l+m+mi}{0}\PYG{o}{\PYGZhy{}}\PYG{n}{opcache}                     \PYG{l+m+mf}{7.0}\PYG{o}{.}\PYG{l+m+mi}{22}\PYG{o}{\PYGZhy{}}\PYG{l+m+mi}{0}\PYG{n}{ubuntu0}\PYG{o}{.}\PYG{l+m+mf}{16.04}\PYG{o}{.}\PYG{l+m+mi}{1}                    \PYG{n}{amd64}        \PYG{n}{Zend} \PYG{n}{OpCache} \PYG{n}{module} \PYG{k}{for} \PYG{n}{PHP}
    \PYG{n}{ii}  \PYG{n}{php7}\PYG{o}{.}\PYG{l+m+mi}{0}\PYG{o}{\PYGZhy{}}\PYG{n}{readline}                    \PYG{l+m+mf}{7.0}\PYG{o}{.}\PYG{l+m+mi}{22}\PYG{o}{\PYGZhy{}}\PYG{l+m+mi}{0}\PYG{n}{ubuntu0}\PYG{o}{.}\PYG{l+m+mf}{16.04}\PYG{o}{.}\PYG{l+m+mi}{1}                    \PYG{n}{amd64}        \PYG{n}{readline} \PYG{n}{module} \PYG{k}{for} \PYG{n}{PHP}
\end{sphinxVerbatim}

\item {} 
remove php

\begin{sphinxVerbatim}[commandchars=\\\{\}]
\PYG{n}{apt} \PYG{n}{autoremove} \PYG{o}{\PYGZhy{}}\PYG{o}{\PYGZhy{}}\PYG{n}{purge} \PYG{n}{php}\PYG{o}{\PYGZhy{}}\PYG{n}{common} \PYG{o}{\PYGZhy{}}\PYG{n}{y}
\PYG{n}{find} \PYG{o}{/}\PYG{n}{etc} \PYG{o}{\PYGZhy{}}\PYG{n}{name} \PYG{l+s+s2}{\PYGZdq{}}\PYG{l+s+s2}{*php*}\PYG{l+s+s2}{\PYGZdq{}}
    \PYG{o}{/}\PYG{n}{etc}\PYG{o}{/}\PYG{n}{php}
    \PYG{o}{/}\PYG{n}{etc}\PYG{o}{/}\PYG{n}{apparmor}\PYG{o}{.}\PYG{n}{d}\PYG{o}{/}\PYG{n}{abstractions}\PYG{o}{/}\PYG{n}{php5}
\PYG{n}{rm} \PYG{o}{/}\PYG{n}{etc}\PYG{o}{/}\PYG{n}{php} \PYG{o}{\PYGZhy{}}\PYG{n}{rf}
\end{sphinxVerbatim}

\end{itemize}


\subsubsection{install\&remove other}
\label{\detokenize{linux/apt:install-remove-other}}\begin{itemize}
\item {} 
install

\begin{sphinxVerbatim}[commandchars=\\\{\}]
\PYG{n}{apt} \PYG{n}{install} \PYG{n}{php7}\PYG{o}{.}\PYG{l+m+mi}{0}\PYG{o}{\PYGZhy{}}\PYG{n}{curl} \PYG{o}{\PYGZhy{}}\PYG{n}{y}
\PYG{n}{find} \PYG{o}{/}\PYG{n}{etc} \PYG{o}{\PYGZhy{}}\PYG{n}{name} \PYG{l+s+s2}{\PYGZdq{}}\PYG{l+s+s2}{*curl*}\PYG{l+s+s2}{\PYGZdq{}}
    \PYG{o}{/}\PYG{n}{etc}\PYG{o}{/}\PYG{n}{php}\PYG{o}{/}\PYG{l+m+mf}{7.0}\PYG{o}{/}\PYG{n}{mods}\PYG{o}{\PYGZhy{}}\PYG{n}{available}\PYG{o}{/}\PYG{n}{curl}\PYG{o}{.}\PYG{n}{ini}
    \PYG{o}{/}\PYG{n}{etc}\PYG{o}{/}\PYG{n}{php}\PYG{o}{/}\PYG{l+m+mf}{7.0}\PYG{o}{/}\PYG{n}{fpm}\PYG{o}{/}\PYG{n}{conf}\PYG{o}{.}\PYG{n}{d}\PYG{o}{/}\PYG{l+m+mi}{20}\PYG{o}{\PYGZhy{}}\PYG{n}{curl}\PYG{o}{.}\PYG{n}{ini}
    \PYG{o}{/}\PYG{n}{etc}\PYG{o}{/}\PYG{n}{php}\PYG{o}{/}\PYG{l+m+mf}{7.0}\PYG{o}{/}\PYG{n}{cli}\PYG{o}{/}\PYG{n}{conf}\PYG{o}{.}\PYG{n}{d}\PYG{o}{/}\PYG{l+m+mi}{20}\PYG{o}{\PYGZhy{}}\PYG{n}{curl}\PYG{o}{.}\PYG{n}{ini}
\PYG{n}{find} \PYG{o}{/}\PYG{n}{usr} \PYG{o}{\PYGZhy{}}\PYG{n}{name} \PYG{l+s+s2}{\PYGZdq{}}\PYG{l+s+s2}{*php7.0\PYGZhy{}curl*}\PYG{l+s+s2}{\PYGZdq{}}
    \PYG{o}{/}\PYG{n}{usr}\PYG{o}{/}\PYG{n}{share}\PYG{o}{/}\PYG{n}{doc}\PYG{o}{/}\PYG{n}{php7}\PYG{o}{.}\PYG{l+m+mi}{0}\PYG{o}{\PYGZhy{}}\PYG{n}{curl}
    \PYG{o}{/}\PYG{n}{usr}\PYG{o}{/}\PYG{n}{share}\PYG{o}{/}\PYG{n}{bug}\PYG{o}{/}\PYG{n}{php7}\PYG{o}{.}\PYG{l+m+mi}{0}\PYG{o}{\PYGZhy{}}\PYG{n}{curl}
    \PYG{o}{/}\PYG{n}{usr}\PYG{o}{/}\PYG{n}{share}\PYG{o}{/}\PYG{n}{php7}\PYG{o}{.}\PYG{l+m+mi}{0}\PYG{o}{\PYGZhy{}}\PYG{n}{curl}
    \PYG{o}{/}\PYG{n}{usr}\PYG{o}{/}\PYG{n}{share}\PYG{o}{/}\PYG{n}{lintian}\PYG{o}{/}\PYG{n}{overrides}\PYG{o}{/}\PYG{n}{php7}\PYG{o}{.}\PYG{l+m+mi}{0}\PYG{o}{\PYGZhy{}}\PYG{n}{curl}
\PYG{n}{find} \PYG{o}{/}\PYG{n}{lib} \PYG{o}{\PYGZhy{}}\PYG{n}{name} \PYG{l+s+s2}{\PYGZdq{}}\PYG{l+s+s2}{*curl*}\PYG{l+s+s2}{\PYGZdq{}}
\PYG{n}{dpkg} \PYG{o}{\PYGZhy{}}\PYG{n}{l} \PYG{o}{\textbar{}} \PYG{n}{grep} \PYG{n}{php7}\PYG{o}{.}\PYG{l+m+mi}{0}\PYG{o}{\PYGZhy{}}\PYG{n}{curl}
    \PYG{n}{ii}  \PYG{n}{php7}\PYG{o}{.}\PYG{l+m+mi}{0}\PYG{o}{\PYGZhy{}}\PYG{n}{curl}                        \PYG{l+m+mf}{7.0}\PYG{o}{.}\PYG{l+m+mi}{22}\PYG{o}{\PYGZhy{}}\PYG{l+m+mi}{0}\PYG{n}{ubuntu0}\PYG{o}{.}\PYG{l+m+mf}{16.04}\PYG{o}{.}\PYG{l+m+mi}{1}                    \PYG{n}{amd64}        \PYG{n}{CURL} \PYG{n}{module} \PYG{k}{for} \PYG{n}{PHP}

\PYG{n}{apt} \PYG{n}{install} \PYG{n}{php7}\PYG{o}{.}\PYG{l+m+mi}{0}\PYG{o}{\PYGZhy{}}\PYG{n}{mbstring} \PYG{o}{\PYGZhy{}}\PYG{n}{y}
\PYG{n}{find} \PYG{o}{/}\PYG{n}{etc} \PYG{o}{\PYGZhy{}}\PYG{n}{name} \PYG{l+s+s2}{\PYGZdq{}}\PYG{l+s+s2}{*mbstring*}\PYG{l+s+s2}{\PYGZdq{}}
    \PYG{o}{/}\PYG{n}{etc}\PYG{o}{/}\PYG{n}{php}\PYG{o}{/}\PYG{l+m+mf}{7.0}\PYG{o}{/}\PYG{n}{mods}\PYG{o}{\PYGZhy{}}\PYG{n}{available}\PYG{o}{/}\PYG{n}{mbstring}\PYG{o}{.}\PYG{n}{ini}
    \PYG{o}{/}\PYG{n}{etc}\PYG{o}{/}\PYG{n}{php}\PYG{o}{/}\PYG{l+m+mf}{7.0}\PYG{o}{/}\PYG{n}{fpm}\PYG{o}{/}\PYG{n}{conf}\PYG{o}{.}\PYG{n}{d}\PYG{o}{/}\PYG{l+m+mi}{20}\PYG{o}{\PYGZhy{}}\PYG{n}{mbstring}\PYG{o}{.}\PYG{n}{ini}
    \PYG{o}{/}\PYG{n}{etc}\PYG{o}{/}\PYG{n}{php}\PYG{o}{/}\PYG{l+m+mf}{7.0}\PYG{o}{/}\PYG{n}{cli}\PYG{o}{/}\PYG{n}{conf}\PYG{o}{.}\PYG{n}{d}\PYG{o}{/}\PYG{l+m+mi}{20}\PYG{o}{\PYGZhy{}}\PYG{n}{mbstring}\PYG{o}{.}\PYG{n}{ini}
\PYG{n}{find} \PYG{o}{/}\PYG{n}{usr} \PYG{o}{\PYGZhy{}}\PYG{n}{name} \PYG{l+s+s2}{\PYGZdq{}}\PYG{l+s+s2}{*mbstring*}\PYG{l+s+s2}{\PYGZdq{}}
    \PYG{o}{/}\PYG{n}{usr}\PYG{o}{/}\PYG{n}{lib}\PYG{o}{/}\PYG{n}{php}\PYG{o}{/}\PYG{l+m+mi}{20151012}\PYG{o}{/}\PYG{n}{mbstring}\PYG{o}{.}\PYG{n}{so}
    \PYG{o}{/}\PYG{n}{usr}\PYG{o}{/}\PYG{n}{share}\PYG{o}{/}\PYG{n}{doc}\PYG{o}{/}\PYG{n}{php7}\PYG{o}{.}\PYG{l+m+mi}{0}\PYG{o}{\PYGZhy{}}\PYG{n}{mbstring}
    \PYG{o}{/}\PYG{n}{usr}\PYG{o}{/}\PYG{n}{share}\PYG{o}{/}\PYG{n}{php7}\PYG{o}{.}\PYG{l+m+mi}{0}\PYG{o}{\PYGZhy{}}\PYG{n}{mbstring}
    \PYG{o}{/}\PYG{n}{usr}\PYG{o}{/}\PYG{n}{share}\PYG{o}{/}\PYG{n}{php7}\PYG{o}{.}\PYG{l+m+mi}{0}\PYG{o}{\PYGZhy{}}\PYG{n}{mbstring}\PYG{o}{/}\PYG{n}{mbstring}
    \PYG{o}{/}\PYG{n}{usr}\PYG{o}{/}\PYG{n}{share}\PYG{o}{/}\PYG{n}{php7}\PYG{o}{.}\PYG{l+m+mi}{0}\PYG{o}{\PYGZhy{}}\PYG{n}{mbstring}\PYG{o}{/}\PYG{n}{mbstring}\PYG{o}{/}\PYG{n}{mbstring}\PYG{o}{.}\PYG{n}{ini}
    \PYG{o}{/}\PYG{n}{usr}\PYG{o}{/}\PYG{n}{share}\PYG{o}{/}\PYG{n}{bug}\PYG{o}{/}\PYG{n}{php7}\PYG{o}{.}\PYG{l+m+mi}{0}\PYG{o}{\PYGZhy{}}\PYG{n}{mbstring}
    \PYG{o}{/}\PYG{n}{usr}\PYG{o}{/}\PYG{n}{share}\PYG{o}{/}\PYG{n}{lintian}\PYG{o}{/}\PYG{n}{overrides}\PYG{o}{/}\PYG{n}{php7}\PYG{o}{.}\PYG{l+m+mi}{0}\PYG{o}{\PYGZhy{}}\PYG{n}{mbstring}
\PYG{n}{find} \PYG{o}{/}\PYG{n}{lib} \PYG{o}{\PYGZhy{}}\PYG{n}{name} \PYG{l+s+s2}{\PYGZdq{}}\PYG{l+s+s2}{*mbstring*}\PYG{l+s+s2}{\PYGZdq{}}
\PYG{n}{dpkg} \PYG{o}{\PYGZhy{}}\PYG{n}{l} \PYG{o}{\textbar{}} \PYG{n}{grep} \PYG{n}{mbstring}
    \PYG{n}{ii}  \PYG{n}{php7}\PYG{o}{.}\PYG{l+m+mi}{0}\PYG{o}{\PYGZhy{}}\PYG{n}{mbstring}                    \PYG{l+m+mf}{7.0}\PYG{o}{.}\PYG{l+m+mi}{22}\PYG{o}{\PYGZhy{}}\PYG{l+m+mi}{0}\PYG{n}{ubuntu0}\PYG{o}{.}\PYG{l+m+mf}{16.04}\PYG{o}{.}\PYG{l+m+mi}{1}                    \PYG{n}{amd64}        \PYG{n}{MBSTRING} \PYG{n}{module} \PYG{k}{for} \PYG{n}{PHP}

\PYG{n}{apt} \PYG{n}{install} \PYG{n}{libapache2}\PYG{o}{\PYGZhy{}}\PYG{n}{mod}\PYG{o}{\PYGZhy{}}\PYG{n}{php7}\PYG{o}{.}\PYG{l+m+mi}{0} \PYG{o}{\PYGZhy{}}\PYG{n}{y}
\PYG{n}{vi} \PYG{o}{/}\PYG{n}{etc}\PYG{o}{/}\PYG{n}{apache2}\PYG{o}{/}\PYG{n}{mods}\PYG{o}{\PYGZhy{}}\PYG{n}{enabled}\PYG{o}{/}\PYG{n}{php7}\PYG{o}{.}\PYG{l+m+mf}{0.}\PYG{n}{load}
    \PYG{n}{LoadModule} \PYG{n}{php7\PYGZus{}module} \PYG{o}{/}\PYG{n}{usr}\PYG{o}{/}\PYG{n}{lib}\PYG{o}{/}\PYG{n}{apache2}\PYG{o}{/}\PYG{n}{modules}\PYG{o}{/}\PYG{n}{libphp7}\PYG{o}{.}\PYG{l+m+mf}{0.}\PYG{n}{so}
\PYG{n}{find} \PYG{o}{/}\PYG{n}{etc} \PYG{o}{\PYGZhy{}}\PYG{n}{name} \PYG{l+s+s2}{\PYGZdq{}}\PYG{l+s+s2}{*libapache2*}\PYG{l+s+s2}{\PYGZdq{}}
\PYG{n}{find} \PYG{o}{/}\PYG{n}{usr} \PYG{o}{\PYGZhy{}}\PYG{n}{name} \PYG{l+s+s2}{\PYGZdq{}}\PYG{l+s+s2}{*libapache2*}\PYG{l+s+s2}{\PYGZdq{}}
\PYG{n}{find} \PYG{o}{/}\PYG{n}{lib} \PYG{o}{\PYGZhy{}}\PYG{n}{name} \PYG{l+s+s2}{\PYGZdq{}}\PYG{l+s+s2}{*libapache2*}\PYG{l+s+s2}{\PYGZdq{}}
\PYG{n}{dpkg} \PYG{o}{\PYGZhy{}}\PYG{n}{l} \PYG{o}{\textbar{}} \PYG{n}{grep} \PYG{n}{libapache2}
    \PYG{n}{ii}  \PYG{n}{libapache2}\PYG{o}{\PYGZhy{}}\PYG{n}{mod}\PYG{o}{\PYGZhy{}}\PYG{n}{php7}\PYG{o}{.}\PYG{l+m+mi}{0}              \PYG{l+m+mf}{7.0}\PYG{o}{.}\PYG{l+m+mi}{22}\PYG{o}{\PYGZhy{}}\PYG{l+m+mi}{0}\PYG{n}{ubuntu0}\PYG{o}{.}\PYG{l+m+mf}{16.04}\PYG{o}{.}\PYG{l+m+mi}{1}                    \PYG{n}{amd64}        \PYG{n}{server}\PYG{o}{\PYGZhy{}}\PYG{n}{side}\PYG{p}{,} \PYG{n}{HTML}\PYG{o}{\PYGZhy{}}\PYG{n}{embedded} \PYG{n}{scripting} \PYG{n}{language} \PYG{p}{(}\PYG{n}{Apache} \PYG{l+m+mi}{2} \PYG{n}{module}\PYG{p}{)}

\PYG{n}{apt} \PYG{n}{install} \PYG{n}{php7}\PYG{o}{.}\PYG{l+m+mi}{0}\PYG{o}{\PYGZhy{}}\PYG{n}{gd} \PYG{o}{\PYGZhy{}}\PYG{n}{y}
\PYG{n}{find} \PYG{o}{/}\PYG{n}{etc} \PYG{o}{\PYGZhy{}}\PYG{n}{name} \PYG{l+s+s2}{\PYGZdq{}}\PYG{l+s+s2}{*gd.ini}\PYG{l+s+s2}{\PYGZdq{}}
    \PYG{o}{/}\PYG{n}{etc}\PYG{o}{/}\PYG{n}{php}\PYG{o}{/}\PYG{l+m+mf}{7.0}\PYG{o}{/}\PYG{n}{apache2}\PYG{o}{/}\PYG{n}{conf}\PYG{o}{.}\PYG{n}{d}\PYG{o}{/}\PYG{l+m+mi}{20}\PYG{o}{\PYGZhy{}}\PYG{n}{gd}\PYG{o}{.}\PYG{n}{ini}
    \PYG{o}{/}\PYG{n}{etc}\PYG{o}{/}\PYG{n}{php}\PYG{o}{/}\PYG{l+m+mf}{7.0}\PYG{o}{/}\PYG{n}{mods}\PYG{o}{\PYGZhy{}}\PYG{n}{available}\PYG{o}{/}\PYG{n}{gd}\PYG{o}{.}\PYG{n}{ini}
    \PYG{o}{/}\PYG{n}{etc}\PYG{o}{/}\PYG{n}{php}\PYG{o}{/}\PYG{l+m+mf}{7.0}\PYG{o}{/}\PYG{n}{fpm}\PYG{o}{/}\PYG{n}{conf}\PYG{o}{.}\PYG{n}{d}\PYG{o}{/}\PYG{l+m+mi}{20}\PYG{o}{\PYGZhy{}}\PYG{n}{gd}\PYG{o}{.}\PYG{n}{ini}
    \PYG{o}{/}\PYG{n}{etc}\PYG{o}{/}\PYG{n}{php}\PYG{o}{/}\PYG{l+m+mf}{7.0}\PYG{o}{/}\PYG{n}{cli}\PYG{o}{/}\PYG{n}{conf}\PYG{o}{.}\PYG{n}{d}\PYG{o}{/}\PYG{l+m+mi}{20}\PYG{o}{\PYGZhy{}}\PYG{n}{gd}\PYG{o}{.}\PYG{n}{ini}
\PYG{n}{find} \PYG{o}{/}\PYG{n}{usr} \PYG{o}{\PYGZhy{}}\PYG{n}{name} \PYG{l+s+s2}{\PYGZdq{}}\PYG{l+s+s2}{*gd}\PYG{l+s+s2}{\PYGZdq{}}
    \PYG{o}{/}\PYG{n}{usr}\PYG{o}{/}\PYG{n}{sbin}\PYG{o}{/}\PYG{n}{rsyslogd}
    \PYG{o}{/}\PYG{n}{usr}\PYG{o}{/}\PYG{n}{share}\PYG{o}{/}\PYG{n}{doc}\PYG{o}{/}\PYG{n}{php7}\PYG{o}{.}\PYG{l+m+mi}{0}\PYG{o}{\PYGZhy{}}\PYG{n}{gd}
    \PYG{o}{/}\PYG{n}{usr}\PYG{o}{/}\PYG{n}{share}\PYG{o}{/}\PYG{n}{bug}\PYG{o}{/}\PYG{n}{php7}\PYG{o}{.}\PYG{l+m+mi}{0}\PYG{o}{\PYGZhy{}}\PYG{n}{gd}
    \PYG{o}{/}\PYG{n}{usr}\PYG{o}{/}\PYG{n}{share}\PYG{o}{/}\PYG{n}{locale}\PYG{o}{/}\PYG{n}{gd}
    \PYG{o}{/}\PYG{n}{usr}\PYG{o}{/}\PYG{n}{share}\PYG{o}{/}\PYG{n}{lintian}\PYG{o}{/}\PYG{n}{overrides}\PYG{o}{/}\PYG{n}{php7}\PYG{o}{.}\PYG{l+m+mi}{0}\PYG{o}{\PYGZhy{}}\PYG{n}{gd}
    \PYG{o}{/}\PYG{n}{usr}\PYG{o}{/}\PYG{n}{share}\PYG{o}{/}\PYG{n}{php7}\PYG{o}{.}\PYG{l+m+mi}{0}\PYG{o}{\PYGZhy{}}\PYG{n}{gd}
    \PYG{o}{/}\PYG{n}{usr}\PYG{o}{/}\PYG{n}{share}\PYG{o}{/}\PYG{n}{php7}\PYG{o}{.}\PYG{l+m+mi}{0}\PYG{o}{\PYGZhy{}}\PYG{n}{gd}\PYG{o}{/}\PYG{n}{gd}
\PYG{n}{ls} \PYG{o}{/}\PYG{n}{usr}\PYG{o}{/}\PYG{n}{lib}\PYG{o}{/}\PYG{n}{php}\PYG{o}{/}\PYG{l+m+mi}{20151012}\PYG{o}{/}\PYG{n}{gd}\PYG{o}{*}
    \PYG{o}{/}\PYG{n}{usr}\PYG{o}{/}\PYG{n}{lib}\PYG{o}{/}\PYG{n}{php}\PYG{o}{/}\PYG{l+m+mi}{20151012}\PYG{o}{/}\PYG{n}{gd}\PYG{o}{.}\PYG{n}{so}
\PYG{n}{dpkg} \PYG{o}{\PYGZhy{}}\PYG{n}{l} \PYG{o}{\textbar{}} \PYG{n}{grep} \PYG{n}{php7}\PYG{o}{.}\PYG{l+m+mi}{0}\PYG{o}{\PYGZhy{}}\PYG{n}{gd}
    \PYG{n}{ii}  \PYG{n}{php7}\PYG{o}{.}\PYG{l+m+mi}{0}\PYG{o}{\PYGZhy{}}\PYG{n}{gd}                          \PYG{l+m+mf}{7.0}\PYG{o}{.}\PYG{l+m+mi}{22}\PYG{o}{\PYGZhy{}}\PYG{l+m+mi}{0}\PYG{n}{ubuntu0}\PYG{o}{.}\PYG{l+m+mf}{16.04}\PYG{o}{.}\PYG{l+m+mi}{1}                    \PYG{n}{amd64}        \PYG{n}{GD} \PYG{n}{module} \PYG{k}{for} \PYG{n}{PHP}

\PYG{n}{apt} \PYG{n}{install} \PYG{n}{php7}\PYG{o}{.}\PYG{l+m+mi}{0}\PYG{o}{\PYGZhy{}}\PYG{n}{mysql} \PYG{o}{\PYGZhy{}}\PYG{n}{y}
\PYG{n}{ls} \PYG{o}{/}\PYG{n}{usr}\PYG{o}{/}\PYG{n}{lib}\PYG{o}{/}\PYG{n}{php}\PYG{o}{/}\PYG{l+m+mi}{20151012}\PYG{o}{/}\PYG{n}{mysql}\PYG{o}{*}
    \PYG{o}{/}\PYG{n}{usr}\PYG{o}{/}\PYG{n}{lib}\PYG{o}{/}\PYG{n}{php}\PYG{o}{/}\PYG{l+m+mi}{20151012}\PYG{o}{/}\PYG{n}{mysqli}\PYG{o}{.}\PYG{n}{so}  \PYG{o}{/}\PYG{n}{usr}\PYG{o}{/}\PYG{n}{lib}\PYG{o}{/}\PYG{n}{php}\PYG{o}{/}\PYG{l+m+mi}{20151012}\PYG{o}{/}\PYG{n}{mysqlnd}\PYG{o}{.}\PYG{n}{so}
\PYG{n}{find} \PYG{o}{/}\PYG{n}{etc} \PYG{o}{\PYGZhy{}}\PYG{n}{name} \PYG{l+s+s2}{\PYGZdq{}}\PYG{l+s+s2}{mysql*.ini}\PYG{l+s+s2}{\PYGZdq{}}
    \PYG{o}{/}\PYG{n}{etc}\PYG{o}{/}\PYG{n}{php}\PYG{o}{/}\PYG{l+m+mf}{7.0}\PYG{o}{/}\PYG{n}{mods}\PYG{o}{\PYGZhy{}}\PYG{n}{available}\PYG{o}{/}\PYG{n}{mysqli}\PYG{o}{.}\PYG{n}{ini}
    \PYG{o}{/}\PYG{n}{etc}\PYG{o}{/}\PYG{n}{php}\PYG{o}{/}\PYG{l+m+mf}{7.0}\PYG{o}{/}\PYG{n}{mods}\PYG{o}{\PYGZhy{}}\PYG{n}{available}\PYG{o}{/}\PYG{n}{mysqlnd}\PYG{o}{.}\PYG{n}{ini}
\PYG{n}{dpkg} \PYG{o}{\PYGZhy{}}\PYG{n}{l} \PYG{o}{\textbar{}} \PYG{n}{grep} \PYG{n}{php7}\PYG{o}{.}\PYG{l+m+mi}{0}\PYG{o}{\PYGZhy{}}\PYG{n}{mysql}
    \PYG{n}{ii}  \PYG{n}{php7}\PYG{o}{.}\PYG{l+m+mi}{0}\PYG{o}{\PYGZhy{}}\PYG{n}{mysql}                       \PYG{l+m+mf}{7.0}\PYG{o}{.}\PYG{l+m+mi}{22}\PYG{o}{\PYGZhy{}}\PYG{l+m+mi}{0}\PYG{n}{ubuntu0}\PYG{o}{.}\PYG{l+m+mf}{16.04}\PYG{o}{.}\PYG{l+m+mi}{1}                    \PYG{n}{amd64}        \PYG{n}{MySQL} \PYG{n}{module} \PYG{k}{for} \PYG{n}{PHP}

\PYG{n}{apt} \PYG{n}{install} \PYG{n}{php}\PYG{o}{\PYGZhy{}}\PYG{n}{redis} \PYG{o}{\PYGZhy{}}\PYG{n}{y}
\PYG{n}{ls} \PYG{o}{/}\PYG{n}{usr}\PYG{o}{/}\PYG{n}{lib}\PYG{o}{/}\PYG{n}{php}\PYG{o}{/}\PYG{l+m+mi}{20151012}\PYG{o}{/}\PYG{n}{redis}\PYG{o}{*}
    \PYG{o}{/}\PYG{n}{usr}\PYG{o}{/}\PYG{n}{lib}\PYG{o}{/}\PYG{n}{php}\PYG{o}{/}\PYG{l+m+mi}{20151012}\PYG{o}{/}\PYG{n}{redis}\PYG{o}{.}\PYG{n}{so}
\PYG{n}{find} \PYG{o}{/}\PYG{n}{etc} \PYG{o}{\PYGZhy{}}\PYG{n}{name} \PYG{l+s+s2}{\PYGZdq{}}\PYG{l+s+s2}{redis*.ini}\PYG{l+s+s2}{\PYGZdq{}}
    \PYG{o}{/}\PYG{n}{etc}\PYG{o}{/}\PYG{n}{php}\PYG{o}{/}\PYG{l+m+mf}{7.0}\PYG{o}{/}\PYG{n}{mods}\PYG{o}{\PYGZhy{}}\PYG{n}{available}\PYG{o}{/}\PYG{n}{redis}\PYG{o}{.}\PYG{n}{ini}
\PYG{n}{dpkg} \PYG{o}{\PYGZhy{}}\PYG{n}{l} \PYG{o}{\textbar{}} \PYG{n}{grep} \PYG{n}{php}\PYG{o}{\PYGZhy{}}\PYG{n}{redis}
    \PYG{n}{ii}  \PYG{n}{php}\PYG{o}{\PYGZhy{}}\PYG{n}{redis}                          \PYG{l+m+mf}{2.2}\PYG{o}{.}\PYG{l+m+mi}{7}\PYG{o}{\PYGZhy{}}\PYG{l+m+mi}{389}\PYG{o}{\PYGZhy{}}\PYG{n}{g2887ad1}\PYG{o}{+}\PYG{l+m+mf}{2.2}\PYG{o}{.}\PYG{l+m+mi}{7}\PYG{o}{\PYGZhy{}}\PYG{l+m+mi}{1}                 \PYG{n}{amd64}        \PYG{n}{PHP} \PYG{n}{extension} \PYG{k}{for} \PYG{n}{interfacing} \PYG{k}{with} \PYG{n}{Redis}

\PYG{n}{apt} \PYG{n}{install} \PYG{n}{mcrypt} \PYG{o}{\PYGZhy{}}\PYG{n}{y}
\PYG{n}{apt} \PYG{n}{install} \PYG{n}{libmcrypt}\PYG{o}{\PYGZhy{}}\PYG{n}{dev} \PYG{o}{\PYGZhy{}}\PYG{n}{y}

\PYG{n}{apt} \PYG{n}{install} \PYG{n}{php}\PYG{o}{\PYGZhy{}}\PYG{n}{mcrypt} \PYG{o}{\PYGZhy{}}\PYG{n}{y}
\PYG{n}{ls} \PYG{o}{/}\PYG{n}{usr}\PYG{o}{/}\PYG{n}{lib}\PYG{o}{/}\PYG{n}{php}\PYG{o}{/}\PYG{l+m+mi}{20151012}\PYG{o}{/}\PYG{n}{mcrypt}\PYG{o}{*}
    \PYG{o}{/}\PYG{n}{usr}\PYG{o}{/}\PYG{n}{lib}\PYG{o}{/}\PYG{n}{php}\PYG{o}{/}\PYG{l+m+mi}{20151012}\PYG{o}{/}\PYG{n}{mcrypt}\PYG{o}{.}\PYG{n}{so}
\PYG{n}{find} \PYG{o}{/}\PYG{n}{etc} \PYG{o}{\PYGZhy{}}\PYG{n}{name} \PYG{l+s+s2}{\PYGZdq{}}\PYG{l+s+s2}{mcrypt*.ini}\PYG{l+s+s2}{\PYGZdq{}}
    \PYG{o}{/}\PYG{n}{etc}\PYG{o}{/}\PYG{n}{php}\PYG{o}{/}\PYG{l+m+mf}{7.0}\PYG{o}{/}\PYG{n}{mods}\PYG{o}{\PYGZhy{}}\PYG{n}{available}\PYG{o}{/}\PYG{n}{mcrypt}\PYG{o}{.}\PYG{n}{ini}
\PYG{n}{dpkg} \PYG{o}{\PYGZhy{}}\PYG{n}{l} \PYG{o}{\textbar{}} \PYG{n}{grep} \PYG{n}{mcrypt} \PYG{o}{\textbar{}} \PYG{n}{grep} \PYG{l+s+s2}{\PYGZdq{}}\PYG{l+s+s2}{php}\PYG{l+s+s2}{\PYGZdq{}}
    \PYG{n}{ii}  \PYG{n}{php}\PYG{o}{\PYGZhy{}}\PYG{n}{mcrypt}                         \PYG{l+m+mi}{1}\PYG{p}{:}\PYG{l+m+mf}{7.0}\PYG{o}{+}\PYG{l+m+mi}{35}\PYG{n}{ubuntu6}                            \PYG{n+nb}{all}          \PYG{n}{libmcrypt} \PYG{n}{module} \PYG{k}{for} \PYG{n}{PHP} \PYG{p}{[}\PYG{n}{default}\PYG{p}{]}
    \PYG{n}{ii}  \PYG{n}{php7}\PYG{o}{.}\PYG{l+m+mi}{0}\PYG{o}{\PYGZhy{}}\PYG{n}{mcrypt}                      \PYG{l+m+mf}{7.0}\PYG{o}{.}\PYG{l+m+mi}{22}\PYG{o}{\PYGZhy{}}\PYG{l+m+mi}{0}\PYG{n}{ubuntu0}\PYG{o}{.}\PYG{l+m+mf}{16.04}\PYG{o}{.}\PYG{l+m+mi}{1}                    \PYG{n}{amd64}        \PYG{n}{libmcrypt} \PYG{n}{module} \PYG{k}{for} \PYG{n}{PHP}
\end{sphinxVerbatim}

\item {} 
remove

\begin{sphinxVerbatim}[commandchars=\\\{\}]
先删除依赖mysql,apache的包,mcrypt和libmcrypt\PYGZhy{}dev是系统包,不删除
apt autoremove \PYGZhy{}\PYGZhy{}purge libapache2\PYGZhy{}mod\PYGZhy{}php7.0 \PYGZhy{}y
ls /etc/apache2/mods\PYGZhy{}enabled/php*
dpkg \PYGZhy{}l \textbar{} grep libapache2
ls /usr/lib/apache2/modules/libphp7.0.so
执行命令均未发现残余
apt autoremove \PYGZhy{}\PYGZhy{}purge php7.0\PYGZhy{}mysql \PYGZhy{}y
执行命令均未发现残余
删除php
apt autoremove \PYGZhy{}\PYGZhy{}purge php\PYGZhy{}common \PYGZhy{}y
ls /usr/lib/php
    ls: cannot access \PYGZsq{}/usr/lib/php\PYGZsq{}: No such file or directory
find /etc \PYGZhy{}name \PYGZdq{}*php*\PYGZdq{}
    /etc/php
    /etc/apparmor.d/abstractions/php5
rm /etc/php \PYGZhy{}rf
\end{sphinxVerbatim}

\end{itemize}

\begin{sphinxadmonition}{attention}{Attention:}
以上软件是基于SPI系统测试,运行在ubuntu16.04上!
\end{sphinxadmonition}


\section{shell}
\label{\detokenize{linux/shell:shell}}\label{\detokenize{linux/shell::doc}}

\subsection{shell语法}
\label{\detokenize{linux/shell:id1}}\begin{itemize}
\item {} 
空函数

\begin{sphinxVerbatim}[commandchars=\\\{\}]
\PYG{n}{init}\PYG{p}{(}\PYG{p}{)}\PYG{p}{\PYGZob{}} \PYG{p}{:}\PYG{p}{;} \PYG{p}{\PYGZcb{}}
\end{sphinxVerbatim}

\item {} 
case

\begin{sphinxVerbatim}[commandchars=\\\{\}]
case \PYGZdq{}\PYGZdl{}optname\PYGZdq{} in
   \PYGZdq{}a\PYGZdq{})
       \PYGZsh{} apache 发布端口
      apache\PYGZus{}publish\PYGZus{}port=\PYGZdl{}OPTARG
       ;;
    *)
           exit
           ;;
esac
\end{sphinxVerbatim}

\end{itemize}


\subsection{shell参数处理}
\label{\detokenize{linux/shell:id2}}\begin{itemize}
\item {} 
refer
\begin{quote}

\sphinxurl{http://www.jb51.net/article/48686.htm}
\end{quote}

\item {} 
处理方式
\begin{quote}

例如:./test.sh -f config.conf -v \textendash{}prefix=/home
\begin{itemize}
\item {} 
手工处理

\begin{sphinxVerbatim}[commandchars=\\\{\}]
* \PYGZdl{}0 : ./test.sh,即命令本身,相当于c/c++中的argv[0]
* \PYGZdl{}1 : \PYGZhy{}f,第一个参数.
* \PYGZdl{}2 : config.conf
* \PYGZdl{}3, \PYGZdl{}4 ... :类推。
* \PYGZdl{}\PYGZsh{}  参数的个数,不包括命令本身,上例中\PYGZdl{}\PYGZsh{}为4.
* \PYGZdl{}@ :参数本身的列表,也不包括命令本身,如上例为 \PYGZhy{}f config.conf \PYGZhy{}v \PYGZhy{}\PYGZhy{}prefix=/home
* \PYGZdl{}* :和\PYGZdl{}@相同,但\PYGZdq{}\PYGZdl{}*\PYGZdq{} 和 \PYGZdq{}\PYGZdl{}@\PYGZdq{}(加引号)并不同,\PYGZdq{}\PYGZdl{}*\PYGZdq{}将所有的参数解释成一个字符串,而\PYGZdq{}\PYGZdl{}@\PYGZdq{}是一个参数数组。
\PYGZsh{}!/bin/bash
for arg in \PYGZdq{}\PYGZdl{}*\PYGZdq{}
do
  echo \PYGZdl{}arg
done
for arg in \PYGZdq{}\PYGZdl{}@\PYGZdq{}
do
 echo \PYGZdl{}arg
done
\end{sphinxVerbatim}

\item {} 
getopts

\begin{sphinxVerbatim}[commandchars=\\\{\}]
不支持长选项
\PYGZsh{}!/bin/bash
while getopts \PYGZdq{}a:bc\PYGZdq{} arg \PYGZsh{}选项后面的冒号表示该选项需要参数
do
        case \PYGZdl{}arg in
             a)
                echo \PYGZdq{}a\PYGZsq{}s arg:\PYGZdl{}optarg\PYGZdq{} \PYGZsh{}参数存在\PYGZdl{}optarg中

             b)
                echo \PYGZdq{}b\PYGZdq{}

             c)
                echo \PYGZdq{}c\PYGZdq{}

             ?)  \PYGZsh{}当有不认识的选项的时候arg为?
            echo \PYGZdq{}unkonw argument\PYGZdq{}
        exit 1

        esac
done
\end{sphinxVerbatim}

\item {} 
getopt

\begin{sphinxVerbatim}[commandchars=\\\{\}]
要点:
1. getopt 用法
语法:getopt [options] [\PYGZhy{}\PYGZhy{}] optstring parameters
例如:getopt ab:cd \PYGZhy{}a \PYGZhy{}b he free cat
输出:\PYGZhy{}a \PYGZhy{}b he \PYGZhy{}\PYGZhy{} free cat
            getopt 根据 ab:cd 将选项和参数 \PYGZhy{}a \PYGZhy{}b he free cat  解析为如下格式:
            \PYGZhy{}a \PYGZhy{}b he \PYGZhy{}\PYGZhy{} free cat
             其中 \PYGZhy{}\PYGZhy{} 将选项与非选项参数分开 free 和 cat 就时非选项参数
2. set \PYGZhy{}\PYGZhy{}
\PYGZhy{}\PYGZhy{} Do not change any of the flags; useful in setting \PYGZdl{}1 to \PYGZhy{}.
set \PYGZhy{}\PYGZhy{}  主要是影响特殊变量\PYGZdl{}1 \PYGZdl{}2 等,其实在上面的脚本中就是将\PYGZdl{}1 \PYGZdl{}2 等参数变量重新组合
例如:
set \PYGZhy{}\PYGZhy{} a b c
shell中的特殊位置变量\PYGZdl{}1 为a \PYGZdl{}2 为 b \PYGZdl{}3 为 c
3.如上脚本为build.sh 用法如下:
./build.sh \PYGZhy{}a \PYGZhy{}c \PYGZhy{}d \PYGZhy{}e dog
./build.sh \PYGZhy{}acde  dog
上面两个命令执行结果相同。
\end{sphinxVerbatim}

\end{itemize}
\end{quote}

\end{itemize}


\subsection{awk}
\label{\detokenize{linux/shell:awk}}

\subsubsection{example}
\label{\detokenize{linux/shell:example}}\begin{itemize}
\item {} 
遍历文件相加

\begin{sphinxVerbatim}[commandchars=\\\{\}]
\PYG{n}{echo} \PYG{o}{\PYGZhy{}}\PYG{n}{e} \PYG{l+s+s2}{\PYGZdq{}}\PYG{l+s+s2}{1}\PYG{l+s+se}{\PYGZbs{}n}\PYG{l+s+s2}{2}\PYG{l+s+se}{\PYGZbs{}n}\PYG{l+s+s2}{3}\PYG{l+s+se}{\PYGZbs{}n}\PYG{l+s+s2}{4}\PYG{l+s+s2}{\PYGZdq{}} \PYG{o}{\PYGZgt{}} \PYG{n}{test}\PYG{o}{.}\PYG{n}{txt}
\PYG{n}{awk} \PYG{l+s+s1}{\PYGZsq{}}\PYG{l+s+s1}{\PYGZob{}}\PYG{l+s+s1}{ sum += \PYGZdl{}1 \PYGZcb{}; END }\PYG{l+s+s1}{\PYGZob{}}\PYG{l+s+s1}{ print sum \PYGZcb{}}\PYG{l+s+s1}{\PYGZsq{}} \PYG{n}{test}\PYG{o}{.}\PYG{n}{txt}
\PYG{l+m+mi}{10}
\end{sphinxVerbatim}

\item {} 
按:分割,打印所有用户名

\begin{sphinxVerbatim}[commandchars=\\\{\}]
\PYG{n}{awk} \PYG{o}{\PYGZhy{}}\PYG{n}{F}\PYG{p}{:} \PYG{l+s+s1}{\PYGZsq{}}\PYG{l+s+s1}{\PYGZob{}}\PYG{l+s+s1}{ print \PYGZdl{}1 \PYGZcb{}}\PYG{l+s+s1}{\PYGZsq{}} \PYG{o}{/}\PYG{n}{etc}\PYG{o}{/}\PYG{n}{passwd}
\end{sphinxVerbatim}

\item {} 
读取配置文件,并输出到变量里

\begin{sphinxVerbatim}[commandchars=\\\{\}]
echo \PYGZhy{}e \PYGZdq{}DB\PYGZus{}PWD=123456\PYGZbs{}nDB\PYGZus{}USER=root\PYGZdq{} \PYGZgt{} test.txt
cat test.txt \textbar{}sed \PYGZdq{}s\PYGZsh{}//\PYGZsh{} \PYGZsh{}g\PYGZdq{} \textbar{} sed \PYGZdq{}s/=/ /g\PYGZdq{} \textbar{} awk  \PYGZsq{}\PYGZob{}if(NF\PYGZgt{}1)\PYGZob{}printf(\PYGZdq{}\PYGZpc{}s=\PYGZbs{}\PYGZdq{}\PYGZpc{}s\PYGZbs{}\PYGZdq{};\PYGZdq{},\PYGZdl{}1,\PYGZdl{}2)\PYGZcb{}\PYGZcb{}\PYGZsq{}
DB\PYGZus{}PWD=\PYGZdq{}123456\PYGZdq{};DB\PYGZus{}USER=\PYGZdq{}root\PYGZdq{};
eval \PYGZdl{}(cat test.txt \textbar{}sed \PYGZdq{}s\PYGZsh{}//\PYGZsh{} \PYGZsh{}g\PYGZdq{} \textbar{} sed \PYGZdq{}s/=/ /g\PYGZdq{} \textbar{} awk  \PYGZsq{}\PYGZob{}if(NF\PYGZgt{}1)\PYGZob{}printf(\PYGZdq{}\PYGZpc{}s=\PYGZbs{}\PYGZdq{}\PYGZpc{}s\PYGZbs{}\PYGZdq{};\PYGZdq{},\PYGZdl{}1,\PYGZdl{}2)\PYGZcb{}\PYGZcb{}\PYGZsq{})
echo \PYGZdl{}DB\PYGZus{}PWD
123456
\end{sphinxVerbatim}

\item {} 
分析url

\begin{sphinxVerbatim}[commandchars=\\\{\}]
param={}`echo \PYGZdq{}\PYGZdl{}redirectUrl\PYGZdq{} \textbar{} awk \PYGZhy{}F \PYGZsq{}?\PYGZsq{} \PYGZsq{}\PYGZob{}print \PYGZdl{}2\PYGZcb{}\PYGZsq{}{}`;
echo \PYGZdq{}\PYGZdl{}param\PYGZdq{} \textbar{} awk \PYGZhy{}F \PYGZsq{}\PYGZam{}\PYGZsq{} \PYGZsq{}\PYGZob{}i=1;while(i\PYGZlt{}=NF)\PYGZob{}n=split(\PYGZdl{}i,array,\PYGZdq{}=\PYGZdq{});if(array[1]==\PYGZdq{}code\PYGZdq{})\PYGZob{}print array[2];break;\PYGZcb{};i++;\PYGZcb{}\PYGZcb{}\PYGZsq{}
\end{sphinxVerbatim}

\item {} 
分析php配置文件,找出session节host配置项

\begin{sphinxVerbatim}[commandchars=\\\{\}]
\PYG{n}{awk} \PYG{o}{\PYGZhy{}}\PYG{n}{F} \PYG{l+s+s1}{\PYGZsq{}}\PYG{l+s+s1}{=\PYGZgt{}}\PYG{l+s+s1}{\PYGZsq{}} \PYG{l+s+s1}{\PYGZsq{}}\PYG{l+s+s1}{\PYGZob{}}\PYG{l+s+s1}{if(\PYGZdl{}0\PYGZti{}/}\PYG{l+s+se}{\PYGZbs{}047}\PYG{l+s+s1}{session}\PYG{l+s+se}{\PYGZbs{}047}\PYG{l+s+s1}{/)}\PYG{l+s+s1}{\PYGZob{}}\PYG{l+s+s1}{find=1;fl=0;fr=0\PYGZcb{}if(find)}\PYG{l+s+s1}{\PYGZob{}}\PYG{l+s+s1}{if(\PYGZdl{}0\PYGZti{}/}\PYG{l+s+s1}{\PYGZbs{}}\PYG{l+s+s1}{[/)}\PYG{l+s+s1}{\PYGZob{}}\PYG{l+s+s1}{fl++;\PYGZcb{};if(\PYGZdl{}0\PYGZti{}/}\PYG{l+s+s1}{\PYGZbs{}}\PYG{l+s+s1}{]/)}\PYG{l+s+s1}{\PYGZob{}}\PYG{l+s+s1}{fr++;\PYGZcb{};if(\PYGZdl{}1\PYGZti{}/}\PYG{l+s+se}{\PYGZbs{}047}\PYG{l+s+s1}{host}\PYG{l+s+se}{\PYGZbs{}047}\PYG{l+s+s1}{/)}\PYG{l+s+s1}{\PYGZob{}}\PYG{l+s+s1}{print \PYGZdl{}2;exit 0;\PYGZcb{}if(fl==fr)}\PYG{l+s+s1}{\PYGZob{}}\PYG{l+s+s1}{exit 99\PYGZcb{}\PYGZcb{}\PYGZcb{}}\PYG{l+s+s1}{\PYGZsq{}} \PYG{o}{.}\PYG{o}{.}\PYG{o}{/}\PYG{n}{application}\PYG{o}{/}\PYG{n}{config}\PYG{o}{.}\PYG{n}{php} \PYG{o}{\textbar{}} \PYG{n}{awk} \PYG{o}{\PYGZhy{}}\PYG{n}{F} \PYG{l+s+s1}{\PYGZsq{}}\PYG{l+s+se}{\PYGZbs{}047}\PYG{l+s+s1}{\PYGZsq{}} \PYG{l+s+s1}{\PYGZsq{}}\PYG{l+s+s1}{\PYGZob{}}\PYG{l+s+s1}{print \PYGZdl{}2,exit 0\PYGZcb{}}\PYG{l+s+s1}{\PYGZsq{}}
\PYG{l+s+s1}{\PYGZsq{}}\PYG{l+s+s1}{127.0.0.1}\PYG{l+s+s1}{\PYGZsq{}}\PYG{p}{,}
\end{sphinxVerbatim}

\end{itemize}


\subsection{curl}
\label{\detokenize{linux/shell:curl}}

\subsubsection{usage}
\label{\detokenize{linux/shell:usage}}\begin{itemize}
\item {} 
静默方式-s,-k不检查证实 -c 创建cookie文件

\begin{sphinxVerbatim}[commandchars=\\\{\}]
result={}`curl \PYGZhy{}s \PYGZhy{}k \PYGZhy{}c cxl\PYGZus{}cookie https://10.0.0.42:1066{}`
 echo \PYGZdl{}result
\PYGZob{}\PYGZdq{}status\PYGZdq{}:0,\PYGZdq{}info\PYGZdq{}:\PYGZdq{}调用接口成功\PYGZdq{},\PYGZdq{}data\PYGZdq{}:\PYGZdq{}home\PYGZdq{}\PYGZcb{}
\end{sphinxVerbatim}

\item {} 
通过使用 -v 和 -trace获取更多的链接信息

\begin{sphinxVerbatim}[commandchars=\\\{\}]
\PYG{n}{curl}  \PYG{o}{\PYGZhy{}}\PYG{n}{v}  \PYG{o}{\PYGZhy{}}\PYG{n}{k} \PYG{o}{\PYGZhy{}}\PYG{n}{c} \PYG{l+s+s2}{\PYGZdq{}}\PYG{l+s+s2}{\PYGZdq{}} \PYG{o}{\PYGZhy{}}\PYG{n}{d} \PYG{l+s+s2}{\PYGZdq{}}\PYG{l+s+s2}{\PYGZdq{}} \PYG{o}{\PYGZhy{}}\PYG{n}{H} \PYG{l+s+s2}{\PYGZdq{}}\PYG{l+s+s2}{\PYGZdq{}} \PYG{n}{https}\PYG{p}{:}\PYG{o}{/}\PYG{o}{/}\PYG{l+m+mf}{10.0}\PYG{o}{.}\PYG{l+m+mf}{0.42}\PYG{p}{:}\PYG{l+m+mi}{1066}
\end{sphinxVerbatim}

\item {} 
打印返回的header(-i)

\begin{sphinxVerbatim}[commandchars=\\\{\}]
\PYG{n}{curl}  \PYG{o}{\PYGZhy{}}\PYG{n}{i}  \PYG{o}{\PYGZhy{}}\PYG{n}{k} \PYG{o}{\PYGZhy{}}\PYG{n}{c} \PYG{l+s+s2}{\PYGZdq{}}\PYG{l+s+s2}{\PYGZdq{}} \PYG{o}{\PYGZhy{}}\PYG{n}{d} \PYG{l+s+s2}{\PYGZdq{}}\PYG{l+s+s2}{\PYGZdq{}} \PYG{o}{\PYGZhy{}}\PYG{n}{H} \PYG{l+s+s2}{\PYGZdq{}}\PYG{l+s+s2}{\PYGZdq{}} \PYG{n}{https}\PYG{p}{:}\PYG{o}{/}\PYG{o}{/}\PYG{l+m+mf}{10.0}\PYG{o}{.}\PYG{l+m+mf}{0.42}\PYG{p}{:}\PYG{l+m+mi}{1066}
\end{sphinxVerbatim}

\end{itemize}


\subsection{cut}
\label{\detokenize{linux/shell:cut}}

\subsubsection{usage}
\label{\detokenize{linux/shell:id3}}\begin{itemize}
\item {} 
按字节剪切

\begin{sphinxVerbatim}[commandchars=\\\{\}]
\PYG{n}{echo} \PYG{l+s+s2}{\PYGZdq{}}\PYG{l+s+s2}{test}\PYG{l+s+s2}{\PYGZdq{}} \PYG{o}{\textbar{}} \PYG{n}{cut} \PYG{o}{\PYGZhy{}}\PYG{o}{\PYGZhy{}}\PYG{n}{b} \PYG{l+m+mi}{2}\PYG{o}{\PYGZhy{}}\PYG{l+m+mi}{4}
\PYG{n}{est}
\end{sphinxVerbatim}

\item {} 
按字符剪切

\begin{sphinxVerbatim}[commandchars=\\\{\}]
\PYG{n}{echo} \PYG{l+s+s2}{\PYGZdq{}}\PYG{l+s+s2}{test}\PYG{l+s+s2}{\PYGZdq{}} \PYG{o}{\textbar{}} \PYG{n}{cut} \PYG{o}{\PYGZhy{}}\PYG{n}{c} \PYG{l+m+mi}{2}\PYG{o}{\PYGZhy{}}\PYG{l+m+mi}{4}
\PYG{n}{est}
\end{sphinxVerbatim}

\item {} 
多个分段用逗号分开

\begin{sphinxVerbatim}[commandchars=\\\{\}]
\PYG{n}{echo} \PYG{l+s+s2}{\PYGZdq{}}\PYG{l+s+s2}{test test}\PYG{l+s+s2}{\PYGZdq{}}\PYG{o}{\textbar{}}\PYG{n}{cut} \PYG{o}{\PYGZhy{}}\PYG{n}{b} \PYG{l+m+mi}{3}\PYG{o}{\PYGZhy{}}\PYG{l+m+mi}{5}\PYG{p}{,}\PYG{l+m+mi}{8}
\PYG{n}{st} \PYG{n}{s}
\end{sphinxVerbatim}

\item {} 
从文件获取

\begin{sphinxVerbatim}[commandchars=\\\{\}]
\PYG{n}{echo} \PYG{o}{\PYGZhy{}}\PYG{n}{e} \PYG{l+s+s2}{\PYGZdq{}}\PYG{l+s+s2}{星期一}\PYG{l+s+se}{\PYGZbs{}n}\PYG{l+s+s2}{星期二}\PYG{l+s+se}{\PYGZbs{}n}\PYG{l+s+s2}{星期三}\PYG{l+s+s2}{\PYGZdq{}} \PYG{o}{\PYGZgt{}} \PYG{n}{test}\PYG{o}{.}\PYG{n}{txt}
\PYG{n}{cut} \PYG{o}{\PYGZhy{}}\PYG{n}{b} \PYG{l+m+mi}{1}\PYG{o}{\PYGZhy{}}\PYG{l+m+mi}{3} \PYG{n}{test}\PYG{o}{.}\PYG{n}{txt}
\PYG{n}{星}
\PYG{n}{星}
\PYG{n}{星}
\end{sphinxVerbatim}

\item {} 
中文

\begin{sphinxVerbatim}[commandchars=\\\{\}]
\PYG{n}{echo} \PYG{l+s+s2}{\PYGZdq{}}\PYG{l+s+s2}{星期四 星期三}\PYG{l+s+s2}{\PYGZdq{}} \PYG{o}{\textbar{}} \PYG{n}{cut} \PYG{o}{\PYGZhy{}}\PYG{n}{c} \PYG{l+m+mi}{1}\PYG{o}{\PYGZhy{}}\PYG{l+m+mi}{3}
\PYG{n}{星}
\end{sphinxVerbatim}

\item {} 
按特殊字符分割

\begin{sphinxVerbatim}[commandchars=\\\{\}]
\PYG{n}{echo} \PYG{l+s+s2}{\PYGZdq{}}\PYG{l+s+s2}{root:test:ok}\PYG{l+s+s2}{\PYGZdq{}} \PYG{o}{\textbar{}} \PYG{n}{cut} \PYG{o}{\PYGZhy{}}\PYG{n}{d} \PYG{p}{:} \PYG{o}{\PYGZhy{}}\PYG{n}{f} \PYG{l+m+mi}{1}
\PYG{n}{root}
\end{sphinxVerbatim}

\item {} 
综合应用

\begin{sphinxVerbatim}[commandchars=\\\{\}]
\PYG{n}{echo} \PYG{l+s+s2}{\PYGZdq{}}\PYG{l+s+s2}{root:test:ok}\PYG{l+s+s2}{\PYGZdq{}} \PYG{o}{\textbar{}} \PYG{n}{cut} \PYG{o}{\PYGZhy{}}\PYG{n}{d} \PYG{p}{:} \PYG{o}{\PYGZhy{}}\PYG{n}{f} \PYG{l+m+mi}{3} \PYG{o}{\textbar{}} \PYG{n}{cut} \PYG{o}{\PYGZhy{}}\PYG{n}{c} \PYG{l+m+mi}{1}
\PYG{n}{o}
\end{sphinxVerbatim}

\end{itemize}


\subsection{find}
\label{\detokenize{linux/shell:find}}

\subsection{grep}
\label{\detokenize{linux/shell:grep}}

\subsubsection{usage}
\label{\detokenize{linux/shell:id4}}\begin{itemize}
\item {} 
去不命中的行

\begin{sphinxVerbatim}[commandchars=\\\{\}]
\PYG{n}{crontab} \PYG{o}{\PYGZhy{}}\PYG{n}{l} \PYG{o}{\textbar{}} \PYG{n}{grep} \PYG{o}{\PYGZhy{}}\PYG{n}{v} \PYG{l+s+s2}{\PYGZdq{}}\PYG{l+s+s2}{php }\PYG{l+s+s2}{\PYGZdq{}}
\end{sphinxVerbatim}

\item {} 
添加crontab

\begin{sphinxVerbatim}[commandchars=\\\{\}]
\PYG{p}{(}\PYG{n}{crontab} \PYG{o}{\PYGZhy{}}\PYG{n}{l} \PYG{l+m+mi}{2}\PYG{o}{\PYGZgt{}}\PYG{o}{/}\PYG{n}{dev}\PYG{o}{/}\PYG{n}{null} \PYG{o}{\textbar{}} \PYG{n}{grep} \PYG{o}{\PYGZhy{}}\PYG{n}{v} \PYG{l+s+s2}{\PYGZdq{}}\PYG{l+s+s2}{php.*cleanAccesstoken}\PYG{l+s+s2}{\PYGZdq{}}\PYG{p}{;}\PYG{n}{echo} \PYG{l+s+s2}{\PYGZdq{}}\PYG{l+s+s2}{* 1 * * * php /home/cxl/git\PYGZhy{}svn/spi\PYGZhy{}php/think cleanAccesstoken}\PYG{l+s+s2}{\PYGZdq{}}\PYG{p}{)} \PYG{o}{\textbar{}} \PYG{n}{crontab} \PYG{o}{\PYGZhy{}}
\end{sphinxVerbatim}

\end{itemize}


\subsection{jq}
\label{\detokenize{linux/shell:jq}}

\subsubsection{install}
\label{\detokenize{linux/shell:install}}
apt install jq -y


\subsubsection{usage}
\label{\detokenize{linux/shell:id5}}\begin{itemize}
\item {} 
get value

\begin{sphinxVerbatim}[commandchars=\\\{\}]
\PYG{n}{echo} \PYG{l+s+s1}{\PYGZsq{}}\PYG{l+s+s1}{\PYGZob{}}\PYG{l+s+s1}{\PYGZdq{}}\PYG{l+s+s1}{status}\PYG{l+s+s1}{\PYGZdq{}}\PYG{l+s+s1}{:0,}\PYG{l+s+s1}{\PYGZdq{}}\PYG{l+s+s1}{info}\PYG{l+s+s1}{\PYGZdq{}}\PYG{l+s+s1}{:}\PYG{l+s+s1}{\PYGZdq{}}\PYG{l+s+s1}{we are success}\PYG{l+s+s1}{\PYGZdq{}}\PYG{l+s+s1}{,}\PYG{l+s+s1}{\PYGZdq{}}\PYG{l+s+s1}{data}\PYG{l+s+s1}{\PYGZdq{}}\PYG{l+s+s1}{:}\PYG{l+s+s1}{\PYGZob{}}\PYG{l+s+s1}{\PYGZdq{}}\PYG{l+s+s1}{client}\PYG{l+s+s1}{\PYGZdq{}}\PYG{l+s+s1}{:}\PYG{l+s+s1}{\PYGZob{}}\PYG{l+s+s1}{\PYGZdq{}}\PYG{l+s+s1}{client\PYGZus{}id}\PYG{l+s+s1}{\PYGZdq{}}\PYG{l+s+s1}{:}\PYG{l+s+s1}{\PYGZdq{}}\PYG{l+s+s1}{123456}\PYG{l+s+s1}{\PYGZdq{}}\PYG{l+s+s1}{,}\PYG{l+s+s1}{\PYGZdq{}}\PYG{l+s+s1}{client\PYGZus{}secret}\PYG{l+s+s1}{\PYGZdq{}}\PYG{l+s+s1}{:}\PYG{l+s+s1}{\PYGZdq{}}\PYG{l+s+s1}{567890}\PYG{l+s+s1}{\PYGZdq{}}\PYG{l+s+s1}{\PYGZcb{}\PYGZcb{}\PYGZcb{}}\PYG{l+s+s1}{\PYGZsq{}} \PYG{o}{\textbar{}} \PYG{n}{jq} \PYG{l+s+s2}{\PYGZdq{}}\PYG{l+s+s2}{.status}\PYG{l+s+s2}{\PYGZdq{}}
\PYG{l+m+mi}{0}
\PYG{n}{echo} \PYG{l+s+s1}{\PYGZsq{}}\PYG{l+s+s1}{\PYGZob{}}\PYG{l+s+s1}{\PYGZdq{}}\PYG{l+s+s1}{status}\PYG{l+s+s1}{\PYGZdq{}}\PYG{l+s+s1}{:0,}\PYG{l+s+s1}{\PYGZdq{}}\PYG{l+s+s1}{info}\PYG{l+s+s1}{\PYGZdq{}}\PYG{l+s+s1}{:}\PYG{l+s+s1}{\PYGZdq{}}\PYG{l+s+s1}{we are success}\PYG{l+s+s1}{\PYGZdq{}}\PYG{l+s+s1}{,}\PYG{l+s+s1}{\PYGZdq{}}\PYG{l+s+s1}{data}\PYG{l+s+s1}{\PYGZdq{}}\PYG{l+s+s1}{:}\PYG{l+s+s1}{\PYGZob{}}\PYG{l+s+s1}{\PYGZdq{}}\PYG{l+s+s1}{client}\PYG{l+s+s1}{\PYGZdq{}}\PYG{l+s+s1}{:}\PYG{l+s+s1}{\PYGZob{}}\PYG{l+s+s1}{\PYGZdq{}}\PYG{l+s+s1}{client\PYGZus{}id}\PYG{l+s+s1}{\PYGZdq{}}\PYG{l+s+s1}{:}\PYG{l+s+s1}{\PYGZdq{}}\PYG{l+s+s1}{123456}\PYG{l+s+s1}{\PYGZdq{}}\PYG{l+s+s1}{,}\PYG{l+s+s1}{\PYGZdq{}}\PYG{l+s+s1}{client\PYGZus{}secret}\PYG{l+s+s1}{\PYGZdq{}}\PYG{l+s+s1}{:}\PYG{l+s+s1}{\PYGZdq{}}\PYG{l+s+s1}{567890}\PYG{l+s+s1}{\PYGZdq{}}\PYG{l+s+s1}{\PYGZcb{}\PYGZcb{}\PYGZcb{}}\PYG{l+s+s1}{\PYGZsq{}} \PYG{o}{\textbar{}} \PYG{n}{jq} \PYG{l+s+s2}{\PYGZdq{}}\PYG{l+s+s2}{.info}\PYG{l+s+s2}{\PYGZdq{}} \PYG{o}{\textbar{}} \PYG{n}{sed} \PYG{l+s+s1}{\PYGZsq{}}\PYG{l+s+s1}{s/}\PYG{l+s+s1}{\PYGZdq{}}\PYG{l+s+s1}{//g}\PYG{l+s+s1}{\PYGZsq{}}
\PYG{n}{we} \PYG{n}{are} \PYG{n}{success}
\PYG{n}{echo} \PYG{l+s+s1}{\PYGZsq{}}\PYG{l+s+s1}{\PYGZob{}}\PYG{l+s+s1}{\PYGZdq{}}\PYG{l+s+s1}{status}\PYG{l+s+s1}{\PYGZdq{}}\PYG{l+s+s1}{:0,}\PYG{l+s+s1}{\PYGZdq{}}\PYG{l+s+s1}{info}\PYG{l+s+s1}{\PYGZdq{}}\PYG{l+s+s1}{:}\PYG{l+s+s1}{\PYGZdq{}}\PYG{l+s+s1}{we are success}\PYG{l+s+s1}{\PYGZdq{}}\PYG{l+s+s1}{,}\PYG{l+s+s1}{\PYGZdq{}}\PYG{l+s+s1}{data}\PYG{l+s+s1}{\PYGZdq{}}\PYG{l+s+s1}{:}\PYG{l+s+s1}{\PYGZob{}}\PYG{l+s+s1}{\PYGZdq{}}\PYG{l+s+s1}{client}\PYG{l+s+s1}{\PYGZdq{}}\PYG{l+s+s1}{:}\PYG{l+s+s1}{\PYGZob{}}\PYG{l+s+s1}{\PYGZdq{}}\PYG{l+s+s1}{client\PYGZus{}id}\PYG{l+s+s1}{\PYGZdq{}}\PYG{l+s+s1}{:}\PYG{l+s+s1}{\PYGZdq{}}\PYG{l+s+s1}{123456}\PYG{l+s+s1}{\PYGZdq{}}\PYG{l+s+s1}{,}\PYG{l+s+s1}{\PYGZdq{}}\PYG{l+s+s1}{client\PYGZus{}secret}\PYG{l+s+s1}{\PYGZdq{}}\PYG{l+s+s1}{:}\PYG{l+s+s1}{\PYGZdq{}}\PYG{l+s+s1}{567890}\PYG{l+s+s1}{\PYGZdq{}}\PYG{l+s+s1}{\PYGZcb{}\PYGZcb{}\PYGZcb{}}\PYG{l+s+s1}{\PYGZsq{}} \PYG{o}{\textbar{}} \PYG{n}{jq} \PYG{l+s+s2}{\PYGZdq{}}\PYG{l+s+s2}{.data.client.client\PYGZus{}id}\PYG{l+s+s2}{\PYGZdq{}} \PYG{o}{\textbar{}} \PYG{n}{sed} \PYG{l+s+s1}{\PYGZsq{}}\PYG{l+s+s1}{s/}\PYG{l+s+s1}{\PYGZdq{}}\PYG{l+s+s1}{//g}\PYG{l+s+s1}{\PYGZsq{}}
\PYG{l+m+mi}{123456}
\end{sphinxVerbatim}

\item {} 
filter data

\begin{sphinxVerbatim}[commandchars=\\\{\}]
\PYG{n}{echo} \PYG{l+s+s1}{\PYGZsq{}}\PYG{l+s+s1}{\PYGZob{}}\PYG{l+s+s1}{\PYGZdq{}}\PYG{l+s+s1}{status}\PYG{l+s+s1}{\PYGZdq{}}\PYG{l+s+s1}{:0,}\PYG{l+s+s1}{\PYGZdq{}}\PYG{l+s+s1}{info}\PYG{l+s+s1}{\PYGZdq{}}\PYG{l+s+s1}{:}\PYG{l+s+s1}{\PYGZdq{}}\PYG{l+s+s1}{we are success}\PYG{l+s+s1}{\PYGZdq{}}\PYG{l+s+s1}{,}\PYG{l+s+s1}{\PYGZdq{}}\PYG{l+s+s1}{data}\PYG{l+s+s1}{\PYGZdq{}}\PYG{l+s+s1}{:}\PYG{l+s+s1}{\PYGZob{}}\PYG{l+s+s1}{\PYGZdq{}}\PYG{l+s+s1}{client}\PYG{l+s+s1}{\PYGZdq{}}\PYG{l+s+s1}{:}\PYG{l+s+s1}{\PYGZob{}}\PYG{l+s+s1}{\PYGZdq{}}\PYG{l+s+s1}{client\PYGZus{}id}\PYG{l+s+s1}{\PYGZdq{}}\PYG{l+s+s1}{:}\PYG{l+s+s1}{\PYGZdq{}}\PYG{l+s+s1}{123456}\PYG{l+s+s1}{\PYGZdq{}}\PYG{l+s+s1}{,}\PYG{l+s+s1}{\PYGZdq{}}\PYG{l+s+s1}{client\PYGZus{}secret}\PYG{l+s+s1}{\PYGZdq{}}\PYG{l+s+s1}{:}\PYG{l+s+s1}{\PYGZdq{}}\PYG{l+s+s1}{567890}\PYG{l+s+s1}{\PYGZdq{}}\PYG{l+s+s1}{\PYGZcb{}\PYGZcb{}\PYGZcb{}}\PYG{l+s+s1}{\PYGZsq{}} \PYG{o}{\textbar{}} \PYG{n}{jq} \PYG{l+s+s1}{\PYGZsq{}}\PYG{l+s+s1}{.data\textbar{}.client}\PYG{l+s+s1}{\PYGZsq{}}
\PYG{p}{\PYGZob{}}
  \PYG{l+s+s2}{\PYGZdq{}}\PYG{l+s+s2}{client\PYGZus{}id}\PYG{l+s+s2}{\PYGZdq{}}\PYG{p}{:} \PYG{l+s+s2}{\PYGZdq{}}\PYG{l+s+s2}{123456}\PYG{l+s+s2}{\PYGZdq{}}\PYG{p}{,}
  \PYG{l+s+s2}{\PYGZdq{}}\PYG{l+s+s2}{client\PYGZus{}secret}\PYG{l+s+s2}{\PYGZdq{}}\PYG{p}{:} \PYG{l+s+s2}{\PYGZdq{}}\PYG{l+s+s2}{567890}\PYG{l+s+s2}{\PYGZdq{}}
\PYG{p}{\PYGZcb{}}
\end{sphinxVerbatim}

\end{itemize}


\subsection{mktemp}
\label{\detokenize{linux/shell:mktemp}}

\subsubsection{usage}
\label{\detokenize{linux/shell:id6}}\begin{itemize}
\item {} 
生成随机文件

\begin{sphinxVerbatim}[commandchars=\\\{\}]
\PYG{n}{mktemp} \PYG{n}{ttXXXX}\PYG{o}{.}\PYG{n}{tmp}
\end{sphinxVerbatim}

\item {} 
生成随机目录

\begin{sphinxVerbatim}[commandchars=\\\{\}]
\PYG{n}{mktemp} \PYG{o}{\PYGZhy{}}\PYG{n}{d} \PYG{n}{tempXXXXXX}
\end{sphinxVerbatim}

\end{itemize}


\subsection{nano}
\label{\detokenize{linux/shell:nano}}

\subsubsection{usage}
\label{\detokenize{linux/shell:id7}}

\subsection{sed}
\label{\detokenize{linux/shell:sed}}

\subsubsection{usage}
\label{\detokenize{linux/shell:id8}}\begin{itemize}
\item {} 
替换字符

\begin{sphinxVerbatim}[commandchars=\\\{\}]
\PYG{n}{echo} \PYG{l+s+s2}{\PYGZdq{}}\PYG{l+s+s2}{this ReplaceWord success}\PYG{l+s+s2}{\PYGZdq{}} \PYG{o}{\textbar{}} \PYG{n}{sed} \PYG{l+s+s2}{\PYGZdq{}}\PYG{l+s+s2}{s/ReplaceWord/is/}\PYG{l+s+s2}{\PYGZdq{}}
\PYG{n}{this} \PYG{o+ow}{is} \PYG{n}{success}
\end{sphinxVerbatim}

\item {} 
替换文件

\begin{sphinxVerbatim}[commandchars=\\\{\}]
\PYG{n}{echo} \PYG{l+s+s2}{\PYGZdq{}}\PYG{l+s+s2}{this ReplaceWord success}\PYG{l+s+s2}{\PYGZdq{}} \PYG{o}{\PYGZgt{}} \PYG{n}{test}\PYG{o}{.}\PYG{n}{txt}
\PYG{n}{cat} \PYG{n}{test}\PYG{o}{.}\PYG{n}{txt}
\PYG{n}{sed} \PYG{o}{\PYGZhy{}}\PYG{n}{i} \PYG{l+s+s2}{\PYGZdq{}}\PYG{l+s+s2}{s/ReplaceWord/is/}\PYG{l+s+s2}{\PYGZdq{}} \PYG{n}{test}\PYG{o}{.}\PYG{n}{txt}
\PYG{n}{cat} \PYG{n}{test}\PYG{o}{.}\PYG{n}{txt}
\end{sphinxVerbatim}

\item {} 
使用正则

\begin{sphinxVerbatim}[commandchars=\\\{\}]
\PYG{n}{echo} \PYG{l+s+s2}{\PYGZdq{}}\PYG{l+s+s2}{this 1234 success}\PYG{l+s+s2}{\PYGZdq{}} \PYG{o}{\textbar{}} \PYG{n}{sed} \PYG{o}{\PYGZhy{}}\PYG{n}{r} \PYG{l+s+s2}{\PYGZdq{}}\PYG{l+s+s2}{s/[0\PYGZhy{}9]+/is/}\PYG{l+s+s2}{\PYGZdq{}}
\PYG{n}{this} \PYG{o+ow}{is} \PYG{n}{success}
\end{sphinxVerbatim}

\item {} 
-g表示所有匹配的都执行

\begin{sphinxVerbatim}[commandchars=\\\{\}]
\PYG{n}{echo} \PYG{l+s+s2}{\PYGZdq{}}\PYG{l+s+s2}{this 1234 success 1234}\PYG{l+s+s2}{\PYGZdq{}} \PYG{o}{\textbar{}} \PYG{n}{sed} \PYG{o}{\PYGZhy{}}\PYG{n}{r} \PYG{l+s+s2}{\PYGZdq{}}\PYG{l+s+s2}{s/[0\PYGZhy{}9]+/is/g}\PYG{l+s+s2}{\PYGZdq{}}
\PYG{n}{this} \PYG{o+ow}{is} \PYG{n}{success} \PYG{o+ow}{is}
\end{sphinxVerbatim}

\item {} 
按行号删除文件内容

\begin{sphinxVerbatim}[commandchars=\\\{\}]
sed \PYGZhy{}i \PYGZsq{}1,50000d\PYGZsq{} \PYGZdq{}\PYGZdl{}cxl\PYGZus{}log\PYGZus{}file\PYGZdq{} \PYGZsh{} 删除50000万行
sed \PYGZhy{}i \PYGZsq{}1d\PYGZsq{} a.txt删首行
sed \PYGZhy{}i \PYGZsq{}\PYGZdl{}d\PYGZsq{} b.txt删尾行
sed \PYGZhy{}i \PYGZsq{}s/[ ]*//g\PYGZsq{} c.txt删空格
sed \PYGZhy{}i \PYGZsq{}/\PYGZca{}\PYGZdl{}/d\PYGZsq{} d.txt删空行
sed \PYGZhy{}i ‘/\PYGZca{}[0\PYGZhy{}9]*\PYGZdl{}/d\PYGZsq{} a.txt删包含数字的行
sed \PYGZhy{}i ‘1,2d’a.txt删2行
sed \PYGZhy{}i ‘/love/d’ a.txt删包含string的行
\end{sphinxVerbatim}

\item {} 
修改制定行内容
\begin{quote}

sed -r  ‘s/( \sphinxstyleemphasis{‘host’ *).}/1”test”/’  ../application/config.php
\end{quote}

\end{itemize}


\subsection{service}
\label{\detokenize{linux/shell:service}}

\subsubsection{usage}
\label{\detokenize{linux/shell:id9}}\begin{itemize}
\item {} 
service \textendash{}status-all

\item {} 
service supervisor status

\end{itemize}


\subsection{tr}
\label{\detokenize{linux/shell:tr}}

\subsubsection{usage}
\label{\detokenize{linux/shell:id10}}\begin{itemize}
\item {} 
转换为大写

\begin{sphinxVerbatim}[commandchars=\\\{\}]
\PYG{n}{echo} \PYG{l+s+s2}{\PYGZdq{}}\PYG{l+s+s2}{hello world}\PYG{l+s+s2}{\PYGZdq{}} \PYG{o}{\textbar{}} \PYG{n}{tr} \PYG{p}{[}\PYG{p}{:}\PYG{n}{lower}\PYG{p}{:}\PYG{p}{]} \PYG{p}{[}\PYG{p}{:}\PYG{n}{upper}\PYG{p}{:}\PYG{p}{]}
\PYG{n}{HELLO} \PYG{n}{WORLD}
\end{sphinxVerbatim}

\item {} 
删除空格、数字和-号

\begin{sphinxVerbatim}[commandchars=\\\{\}]
\PYG{n}{echo} \PYG{l+s+s2}{\PYGZdq{}}\PYG{l+s+s2}{hello\PYGZhy{}123\PYGZhy{}world  empty}\PYG{l+s+s2}{\PYGZdq{}} \PYG{o}{\textbar{}} \PYG{n}{tr} \PYG{o}{\PYGZhy{}}\PYG{n}{d} \PYG{l+s+s1}{\PYGZsq{}}\PYG{l+s+s1}{[:blank:][:digit:]\PYGZhy{}}\PYG{l+s+s1}{\PYGZsq{}}
\PYG{n}{helloworldempty}
\end{sphinxVerbatim}

\item {} 
删除补级以外的字符,和上一例子正相反

\begin{sphinxVerbatim}[commandchars=\\\{\}]
\PYG{n}{echo} \PYG{l+s+s2}{\PYGZdq{}}\PYG{l+s+s2}{hello \PYGZhy{} 123\PYGZhy{}world  empty}\PYG{l+s+s2}{\PYGZdq{}} \PYG{o}{\textbar{}} \PYG{n}{tr} \PYG{o}{\PYGZhy{}}\PYG{n}{d} \PYG{o}{\PYGZhy{}}\PYG{n}{c} \PYG{l+s+s1}{\PYGZsq{}}\PYG{l+s+s1}{[:blank:][:digit:]\PYGZhy{}}\PYG{l+s+s1}{\PYGZsq{}}
\PYG{o}{\PYGZhy{}} \PYG{l+m+mi}{123}\PYG{o}{\PYGZhy{}}
\end{sphinxVerbatim}

\item {} 
去掉连续重复字符

\begin{sphinxVerbatim}[commandchars=\\\{\}]
\PYG{n}{echo} \PYG{l+s+s2}{\PYGZdq{}}\PYG{l+s+s2}{helloooo oooo    isssso ookk}\PYG{l+s+s2}{\PYGZdq{}} \PYG{o}{\textbar{}} \PYG{n}{tr} \PYG{o}{\PYGZhy{}}\PYG{n}{s} \PYG{l+s+s2}{\PYGZdq{}}\PYG{l+s+s2}{ os}\PYG{l+s+s2}{\PYGZdq{}}
\PYG{n}{hello} \PYG{n}{o} \PYG{n}{iso} \PYG{n}{okk}
\end{sphinxVerbatim}

\item {} 
删除window文件造成\textasciicircum{}M字符

\begin{sphinxVerbatim}[commandchars=\\\{\}]
\PYG{n}{cat} \PYG{n}{file} \PYG{o}{\textbar{}} \PYG{n}{tr} \PYG{o}{\PYGZhy{}}\PYG{n}{s} \PYG{l+s+s2}{\PYGZdq{}}\PYG{l+s+se}{\PYGZbs{}r}\PYG{l+s+s2}{\PYGZdq{}} \PYG{l+s+s2}{\PYGZdq{}}\PYG{l+s+se}{\PYGZbs{}n}\PYG{l+s+s2}{\PYGZdq{}} \PYG{o}{\PYGZgt{}} \PYG{n}{new\PYGZus{}file}
\PYG{n}{cat} \PYG{n}{file} \PYG{o}{\textbar{}} \PYG{n}{tr} \PYG{o}{\PYGZhy{}}\PYG{n}{d} \PYG{l+s+s2}{\PYGZdq{}}\PYG{l+s+se}{\PYGZbs{}r}\PYG{l+s+s2}{\PYGZdq{}} \PYG{o}{\PYGZgt{}} \PYG{n}{new\PYGZus{}file}
\end{sphinxVerbatim}

\end{itemize}


\subsection{uname}
\label{\detokenize{linux/shell:uname}}

\subsubsection{usage}
\label{\detokenize{linux/shell:id11}}\begin{itemize}
\item {} 
打印全部

\begin{sphinxVerbatim}[commandchars=\\\{\}]
\PYG{n}{uname} \PYG{o}{\PYGZhy{}}\PYG{n}{a}
\PYG{n}{Linux} \PYG{l+m+mi}{1604}\PYG{n}{developer} \PYG{l+m+mf}{4.4}\PYG{o}{.}\PYG{l+m+mi}{0}\PYG{o}{\PYGZhy{}}\PYG{l+m+mi}{89}\PYG{o}{\PYGZhy{}}\PYG{n}{generic} \PYG{c+c1}{\PYGZsh{}112\PYGZhy{}Ubuntu SMP Mon Jul 31 19:38:41 UTC 2017 x86\PYGZus{}64 x86\PYGZus{}64 x86\PYGZus{}64 GNU/Linux}
\end{sphinxVerbatim}

\item {} 
打印kernel-name

\begin{sphinxVerbatim}[commandchars=\\\{\}]
\PYG{n}{uname} \PYG{o}{\PYGZhy{}}\PYG{n}{s}
\PYG{n}{Linux}
\end{sphinxVerbatim}

\item {} 
打印机器网络名

\begin{sphinxVerbatim}[commandchars=\\\{\}]
\PYG{n}{uname} \PYG{o}{\PYGZhy{}}\PYG{n}{n}
\PYG{l+m+mi}{1604}\PYG{n}{developer}
\end{sphinxVerbatim}

\item {} 
打印kernel-release

\begin{sphinxVerbatim}[commandchars=\\\{\}]
\PYG{n}{uname} \PYG{o}{\PYGZhy{}}\PYG{n}{r}
\PYG{l+m+mf}{4.4}\PYG{o}{.}\PYG{l+m+mi}{0}\PYG{o}{\PYGZhy{}}\PYG{l+m+mi}{89}\PYG{o}{\PYGZhy{}}\PYG{n}{generic}
\end{sphinxVerbatim}

\item {} 
打印kernel-version

\begin{sphinxVerbatim}[commandchars=\\\{\}]
\PYG{n}{uname} \PYG{o}{\PYGZhy{}}\PYG{n}{v}
\PYG{c+c1}{\PYGZsh{}112\PYGZhy{}Ubuntu SMP Mon Jul 31 19:38:41 UTC 2017}
\end{sphinxVerbatim}

\item {} 
打印机器硬件名称

\begin{sphinxVerbatim}[commandchars=\\\{\}]
\PYG{n}{uname} \PYG{o}{\PYGZhy{}}\PYG{n}{m}
\PYG{n}{x86\PYGZus{}64}
\end{sphinxVerbatim}

\item {} 
打印处理器类型

\begin{sphinxVerbatim}[commandchars=\\\{\}]
\PYG{n}{uname} \PYG{o}{\PYGZhy{}}\PYG{n}{p}
\PYG{n}{x86\PYGZus{}64}
\end{sphinxVerbatim}

\item {} 
打印硬件平台

\begin{sphinxVerbatim}[commandchars=\\\{\}]
\PYG{n}{uname} \PYG{o}{\PYGZhy{}}\PYG{n}{i}
\PYG{n}{x86\PYGZus{}64}
\end{sphinxVerbatim}

\item {} 
打印操作系统

\begin{sphinxVerbatim}[commandchars=\\\{\}]
\PYG{n}{uname} \PYG{o}{\PYGZhy{}}\PYG{n}{o}
\PYG{n}{GNU}\PYG{o}{/}\PYG{n}{Linux}
\end{sphinxVerbatim}

\end{itemize}


\subsection{update-rc.d}
\label{\detokenize{linux/shell:update-rc-d}}

\subsubsection{usage}
\label{\detokenize{linux/shell:id12}}\begin{itemize}
\item {} 
添加启动项

\begin{sphinxVerbatim}[commandchars=\\\{\}]
\PYG{n}{sudo} \PYG{n}{update}\PYG{o}{\PYGZhy{}}\PYG{n}{rc}\PYG{o}{.}\PYG{n}{d}   \PYG{n}{apache2} \PYG{n}{defaults}
\PYG{n}{sudo} \PYG{n}{update}\PYG{o}{\PYGZhy{}}\PYG{n}{rc}\PYG{o}{.}\PYG{n}{d}   \PYG{n}{nginx} \PYG{n}{defaults}
\PYG{n}{sudo} \PYG{n}{update}\PYG{o}{\PYGZhy{}}\PYG{n}{rc}\PYG{o}{.}\PYG{n}{d}   \PYG{n}{redis\PYGZus{}6379} \PYG{n}{defaults}
\end{sphinxVerbatim}

\item {} 
删除启动项

\begin{sphinxVerbatim}[commandchars=\\\{\}]
\PYG{n}{sudo} \PYG{n}{update}\PYG{o}{\PYGZhy{}}\PYG{n}{rc}\PYG{o}{.}\PYG{n}{d} \PYG{o}{\PYGZhy{}}\PYG{n}{f} \PYG{n}{apache2} \PYG{n}{remove}
\PYG{n}{sudo} \PYG{n}{update}\PYG{o}{\PYGZhy{}}\PYG{n}{rc}\PYG{o}{.}\PYG{n}{d} \PYG{o}{\PYGZhy{}}\PYG{n}{f} \PYG{n}{nginx} \PYG{n}{remove}
\PYG{n}{sudo} \PYG{n}{update}\PYG{o}{\PYGZhy{}}\PYG{n}{rc}\PYG{o}{.}\PYG{n}{d} \PYG{o}{\PYGZhy{}}\PYG{n}{f} \PYG{n}{redis\PYGZus{}6379} \PYG{n}{remove}
\end{sphinxVerbatim}

\end{itemize}


\subsection{vi}
\label{\detokenize{linux/shell:vi}}

\subsubsection{usage}
\label{\detokenize{linux/shell:id13}}\begin{itemize}
\item {} 
翻下页ctrl+f

\item {} 
翻上页ctrl+b

\item {} 
设置行号 set nu

\end{itemize}


\subsection{xargs}
\label{\detokenize{linux/shell:xargs}}

\subsubsection{usage}
\label{\detokenize{linux/shell:id14}}\begin{enumerate}
\item {} 
当你尝试用rm 删除太多的文件,你可能得到一个错误信息:/bin/rm Argument list too long. 用xargs 去避免这个问题

\begin{sphinxVerbatim}[commandchars=\\\{\}]
find \PYGZti{} \PYGZhy{}name ‘*.log\PYGZsq{} \PYGZhy{}print0 \textbar{} xargs \PYGZhy{}0 rm \PYGZhy{}f
\end{sphinxVerbatim}

\item {} 
获得/etc/ 下所有*.conf 结尾的文件列表,有几种不同的方法能得到相同的结果,下面的例子仅仅是示范怎么实用xargs ,在这个例子中实用 xargs将find 命令的输出传递给ls -l

\begin{sphinxVerbatim}[commandchars=\\\{\}]
find /etc \PYGZhy{}name \PYGZdq{}*.conf\PYGZdq{} \textbar{} xargs ls \textendash{}l
\end{sphinxVerbatim}

\item {} 
假如你有一个文件包含了很多你希望下载的URL, 你能够使用xargs 下载所有链接

\begin{sphinxVerbatim}[commandchars=\\\{\}]
cat url\PYGZhy{}list.txt \textbar{} xargs wget \textendash{}c
\end{sphinxVerbatim}

\item {} 
查找所有的jpg 文件,并且压缩它

\begin{sphinxVerbatim}[commandchars=\\\{\}]
\PYG{n}{find} \PYG{o}{/} \PYG{o}{\PYGZhy{}}\PYG{n}{name} \PYG{o}{*}\PYG{o}{.}\PYG{n}{jpg} \PYG{o}{\PYGZhy{}}\PYG{n+nb}{type} \PYG{n}{f} \PYG{o}{\PYGZhy{}}\PYG{n+nb}{print} \PYG{o}{\textbar{}} \PYG{n}{xargs} \PYG{n}{tar} \PYG{o}{\PYGZhy{}}\PYG{n}{cvzf} \PYG{n}{images}\PYG{o}{.}\PYG{n}{tar}\PYG{o}{.}\PYG{n}{gz}
\end{sphinxVerbatim}

\item {} 
拷贝所有的图片文件到一个外部的硬盘驱动

\begin{sphinxVerbatim}[commandchars=\\\{\}]
\PYG{n}{ls} \PYG{o}{*}\PYG{o}{.}\PYG{n}{jpg} \PYG{o}{\textbar{}} \PYG{n}{xargs} \PYG{o}{\PYGZhy{}}\PYG{n}{n1} \PYG{o}{\PYGZhy{}}\PYG{n}{i} \PYG{n}{cp} \PYG{p}{\PYGZob{}}\PYG{p}{\PYGZcb{}} \PYG{o}{/}\PYG{n}{external}\PYG{o}{\PYGZhy{}}\PYG{n}{hard}\PYG{o}{\PYGZhy{}}\PYG{n}{drive}\PYG{o}{/}\PYG{n}{directory}
\end{sphinxVerbatim}

\item {} 
查找另一个目录同名文件,-n1表示每行一个参数 -i表示用\{\}表示输出的每行记录

\begin{sphinxVerbatim}[commandchars=\\\{\}]
\PYG{n}{ls} \PYG{o}{/}\PYG{n}{home}\PYG{o}{/}\PYG{n}{cxl}\PYG{o}{/}\PYG{n}{git}\PYG{o}{\PYGZhy{}}\PYG{n}{svn}\PYG{o}{/}\PYG{n}{spi}\PYG{o}{/}\PYG{n}{spi}\PYG{o}{\PYGZhy{}}\PYG{n}{php}\PYG{o}{/}\PYG{n+nb}{bin}\PYG{o}{/}\PYG{n}{res}\PYG{o}{/}\PYG{n}{supervisor} \PYG{o}{\textbar{}} \PYG{n}{xargs} \PYG{o}{\PYGZhy{}}\PYG{n}{n1} \PYG{o}{\PYGZhy{}}\PYG{n}{i} \PYG{n}{echo} \PYG{p}{\PYGZob{}}\PYG{p}{\PYGZcb{}}
\end{sphinxVerbatim}

\end{enumerate}


\subsection{环境变量}
\label{\detokenize{linux/shell:id15}}

\subsubsection{refer}
\label{\detokenize{linux/shell:refer}}
\sphinxurl{http://www.cnblogs.com/zhaofeng555/p/4895517.html}


\subsubsection{usage}
\label{\detokenize{linux/shell:id16}}
先将export LANG=zh\_CN加入/etc/profile ,退出系统重新登录,登录提示显示英文。将/etc/profile 中的export LANG=zh\_CN删除,将LNAG=zh\_CN加入/etc/environment,退出系统重新登录,登录提示显示中文。用户环境建立的过程中总是先执行/etc/profile然后在读取/etc/environment。为什么会有如上所叙的不同呢?

应该是先执行/etc/environment,后执行/etc/profile。
/etc/environment是设置整个系统的环境,而/etc/profile是设置所有用户的环境,前者与登录用户无关,后者与登录用户有关。  www.2cto.com
系统应用程序的执行与用户环境可以是无关的,但与系统环境是相关的,所以当你登录时,你看到的提示信息,象日期、时间信息的显示格式与系统环境的LANG是相关的,缺省LANG=en\_US,如果系统环境LANG=zh\_CN,则提示信息是中文的,否则是英文的。

对于用户的SHELL初始化而言是先执行/etc/profile,再读取文件/etc/environment.对整个系统而言是先执行/etc/environment。这样理解正确吗?
/etc/enviroment \textendash{}\textgreater{} /etc/profile \textendash{}\textgreater{} \$HOME/.profile   \textendash{}\textgreater{}\$HOME/.env (如果存在)

/etc/profile 是所有用户的环境变量
/etc/enviroment是系统的环境变量
登陆系统时shell读取的顺序应该是   www.2cto.com
\begin{quote}

/etc/profile -\textgreater{}/etc/enviroment \textendash{}\textgreater{}\$HOME/.profile   \textendash{}\textgreater{}\$HOME/.env
\end{quote}

原因应该是jtw所说的用户环境和系统环境的区别了
如果同一个变量在用户环境(/etc/profile)和系统环境(/etc/environment)有不同的值那应该是以用户环境为准了。

(1)/etc/profile: 此文件为系统的每个用户设置环境信息,当用户第一次登录时,该文件被执行. 并从/etc/profile.d目录的配置文件中搜集shell的设置。

(2)/etc/bashrc: 为每一个运行bash shell的用户执行此文件.当bash shell被打开时,该文件被读取。  www.2cto.com

(3)\textasciitilde{}/.bash\_profile: 每个用户都可使用该文件输入专用于自己使用的shell信息,当用户登录时,该文件仅仅执行一次!默认情况下,他设置一些环境变量,执行用户的.bashrc文件。

(4)\textasciitilde{}/.bashrc: 该文件包含专用于你的bash shell的bash信息,当登录时以及每次打开新的shell时,该该文件被读取。

(5) \textasciitilde{}/.bash\_logout:当每次退出系统(退出bash shell)时,执行该文件. 另外,/etc/profile中设定的变量(全局)的可以作用于任何用户,而\textasciitilde{}/.bashrc等中设定的变量(局部)只能继承 /etc/profile中的变量,他们是”父子”关系。

(6)\textasciitilde{}/.bash\_profile 是交互式、login 方式进入 bash 运行的\textasciitilde{}/.bashrc 是交互式 non-login 方式进入 bash 运行的通常二者设置大致相同,所以通常前者会调用后者。


\section{virtualbox}
\label{\detokenize{linux/virtualbox::doc}}\label{\detokenize{linux/virtualbox:virtualbox}}

\subsection{install}
\label{\detokenize{linux/virtualbox:install}}
\begin{DUlineblock}{0em}
\item[] 1、编辑source.list添加如下内容deb \sphinxurl{http://download.virtualbox.org/virtualbox/debian} trusty contrib
\item[] 2、wget -q \sphinxurl{https://www.virtualbox.org/download/oracle\_vbox\_2016.asc} -O- \textbar{} sudo apt-key add -
\item[] 3、wget -q \sphinxurl{https://www.virtualbox.org/download/oracle\_vbox.asc} -O- \textbar{} sudo apt-key add -
\item[] 4、apt-get update
\item[] 5、apt-get install virtualbox-5.1
\item[] 6、wget \sphinxurl{http://download.virtualbox.org/virtualbox/5.1.0/Oracle\_VM\_VirtualBox\_Extension\_Pack-5.1.0-108711.vbox-extpack} 下载扩展
\item[] 7、vboxmanage extpack install Oracle\_VM\_VirtualBox\_Extension\_Pack-5.1.0-108711.vbox-extpack 安装扩展
\item[] 8、vboxmanage list ostypes 查看支持的虚拟机类型
\item[] 9、vboxmanage list vms 查看已存在的虚拟机
\end{DUlineblock}


\subsection{example}
\label{\detokenize{linux/virtualbox:example}}\begin{itemize}
\item {} 
create vm

\begin{sphinxVerbatim}[commandchars=\\\{\}]
vboxmanage createvm \PYGZhy{}\PYGZhy{}name \PYGZdq{}ubuntu1204\PYGZhy{}base\PYGZdq{} \PYGZhy{}\PYGZhy{}ostype \PYGZdq{}Ubuntu\PYGZus{}64\PYGZdq{} \PYGZhy{}\PYGZhy{}basefolder \PYGZdq{}/home/cxl/virtualbox/vm/\PYGZdq{} \PYGZhy{}\PYGZhy{}register 创建虚拟机
vboxmanage createhd \PYGZhy{}\PYGZhy{}filename /home/cxl/virtualbox/vm/ubuntu1204\PYGZhy{}base/ubuntu1204\PYGZhy{}base \PYGZhy{}\PYGZhy{}size 8000 创建虚拟硬盘
vboxmanage storagectl \PYGZdq{}ubuntu1204\PYGZhy{}base\PYGZdq{} \PYGZhy{}\PYGZhy{}add ide  \PYGZhy{}\PYGZhy{}name \PYGZdq{}IDE Controller\PYGZdq{} \PYGZhy{}\PYGZhy{}bootable on  创建ide接口
vboxmanage storageattach \PYGZdq{}ubuntu1204\PYGZhy{}base\PYGZdq{} \PYGZhy{}\PYGZhy{}storagectl \PYGZdq{}IDE Controller\PYGZdq{} \PYGZhy{}\PYGZhy{}port 0 \PYGZhy{}\PYGZhy{}device 0 \PYGZhy{}\PYGZhy{}type hdd \PYGZhy{}\PYGZhy{}medium \PYGZdq{}/home/cxl/virtualbox/vm/ubuntu1204\PYGZhy{}base/ubuntu1204\PYGZhy{}base.vdi\PYGZdq{} 虚拟机关联硬盘
vboxmanage storageattach \PYGZdq{}ubuntu1204\PYGZhy{}base\PYGZdq{} \PYGZhy{}\PYGZhy{}storagectl \PYGZdq{}IDE Controller\PYGZdq{} \PYGZhy{}\PYGZhy{}port 1 \PYGZhy{}\PYGZhy{}device 0 \PYGZhy{}\PYGZhy{}type dvddrive \PYGZhy{}\PYGZhy{}medium \PYGZdq{}/home/cxl/virtualbox/iso/ubuntu\PYGZhy{}12.04.1\PYGZhy{}server\PYGZhy{}amd64.iso\PYGZdq{} 虚拟机关联光驱,加入iso安装文件
vboxmanage modifyvm \PYGZdq{}ubuntu1204\PYGZhy{}base\PYGZdq{} \PYGZhy{}\PYGZhy{}vrde on \PYGZhy{}\PYGZhy{}vrdeport 5001
vboxmanage startvm \PYGZdq{}buildserver\PYGZdq{} \PYGZhy{}\PYGZhy{}type headless
\end{sphinxVerbatim}

\item {} 
delete vm

\begin{sphinxVerbatim}[commandchars=\\\{\}]
\PYG{n}{vboxmanage} \PYG{n}{unregistervm} \PYG{n}{ubuntu1204}\PYG{o}{\PYGZhy{}}\PYG{n}{base} \PYG{o}{\PYGZhy{}}\PYG{o}{\PYGZhy{}}\PYG{n}{delete} \PYG{n}{删除虚拟机}
\end{sphinxVerbatim}

\item {} 
clone vm

\begin{sphinxVerbatim}[commandchars=\\\{\}]
\PYG{n}{vboxmanage} \PYG{n}{clonevm} \PYG{l+s+s2}{\PYGZdq{}}\PYG{l+s+s2}{ubuntu1204\PYGZhy{}base}\PYG{l+s+s2}{\PYGZdq{}} \PYG{o}{\PYGZhy{}}\PYG{o}{\PYGZhy{}}\PYG{n}{name} \PYG{l+s+s2}{\PYGZdq{}}\PYG{l+s+s2}{buildserver}\PYG{l+s+s2}{\PYGZdq{}} \PYG{o}{\PYGZhy{}}\PYG{o}{\PYGZhy{}}\PYG{n}{register} \PYG{o}{\PYGZhy{}}\PYG{o}{\PYGZhy{}}\PYG{n}{basefolder} \PYG{l+s+s2}{\PYGZdq{}}\PYG{l+s+s2}{/home/cxl/virtualbox/vm\PYGZhy{}root/}\PYG{l+s+s2}{\PYGZdq{}}
\PYG{n}{vboxmanage} \PYG{n}{startvm} \PYG{l+s+s2}{\PYGZdq{}}\PYG{l+s+s2}{buildserver}\PYG{l+s+s2}{\PYGZdq{}} \PYG{o}{\PYGZhy{}}\PYG{o}{\PYGZhy{}}\PYG{n+nb}{type} \PYG{n}{headless}
\end{sphinxVerbatim}

\end{itemize}


\section{supervisor}
\label{\detokenize{linux/supervisor:supervisor}}\label{\detokenize{linux/supervisor::doc}}

\subsection{refer}
\label{\detokenize{linux/supervisor:refer}}\begin{quote}

\sphinxurl{http://supervisord.org/}
\sphinxurl{http://www.ttlsa.com/linux/using-supervisor-control-program/}
\end{quote}


\subsection{install}
\label{\detokenize{linux/supervisor:install}}\begin{quote}

Supervisor: A Process Control System
apt install supervisor -y
\end{quote}


\subsection{usage}
\label{\detokenize{linux/supervisor:usage}}\begin{itemize}
\item {} 
example

\begin{sphinxVerbatim}[commandchars=\\\{\}]
cd /etc/supervisor/conf.d
echo\PYGZus{}supervisord\PYGZus{}conf \PYGZgt{} /etc/supervisord.conf
[unix\PYGZus{}http\PYGZus{}server]
file=/tmp/supervisor.sock   ; UNIX socket 文件,supervisorctl 会使用
;chmod=0700                 ; socket 文件的 mode,默认是 0700
;chown=nobody:nogroup       ; socket 文件的 owner,格式: uid:gid
;[inet\PYGZus{}http\PYGZus{}server]         ; HTTP 服务器,提供 web 管理界面
;port=127.0.0.1:9001        ; Web 管理后台运行的 IP 和端口,如果开放到公网,需要注意安全性
;username=user              ; 登录管理后台的用户名
;password=123               ; 登录管理后台的密码

[supervisord]
logfile=/tmp/supervisord.log ; 日志文件,默认是 \PYGZdl{}CWD/supervisord.log
logfile\PYGZus{}maxbytes=50MB        ; 日志文件大小,超出会 rotate,默认 50MB
logfile\PYGZus{}backups=10           ; 日志文件保留备份数量默认 10
loglevel=info                ; 日志级别,默认 info,其它: debug,warn,trace
pidfile=/tmp/supervisord.pid ; pid 文件
nodaemon=false               ; 是否在前台启动,默认是 false,即以 daemon 的方式启动
minfds=1024                  ; 可以打开的文件描述符的最小值,默认 1024
minprocs=200                 ; 可以打开的进程数的最小值,默认 200

; the below section must remain in the config file for RPC
; (supervisorctl/web interface) to work, additional interfaces may be
; added by defining them in separate rpcinterface: sections
[rpcinterface:supervisor]
supervisor.rpcinterface\PYGZus{}factory = supervisor.rpcinterface:make\PYGZus{}main\PYGZus{}rpcinterface

[supervisorctl]
serverurl=unix:///tmp/supervisor.sock ; 通过 UNIX socket 连接 supervisord,路径与 unix\PYGZus{}http\PYGZus{}server 部分的 file 一致
;serverurl=http://127.0.0.1:9001 ; 通过 HTTP 的方式连接 supervisord

; 包含其他的配置文件
[include]
files = relative/directory/*.ini    ; 可以是 *.conf 或 *.ini

增加一个管理进程
[program:usercenter]
directory = /home/leon/projects/usercenter ; 程序的启动目录
command = gunicorn \PYGZhy{}c gunicorn.py wsgi:app  ; 启动命令,可以看出与手动在命令行启动的命令是一样的
autostart = true     ; 在 supervisord 启动的时候也自动启动
startsecs = 5        ; 启动 5 秒后没有异常退出,就当作已经正常启动了
autorestart = true   ; 程序异常退出后自动重启
startretries = 3     ; 启动失败自动重试次数,默认是 3
user = leon          ; 用哪个用户启动
redirect\PYGZus{}stderr = true  ; 把 stderr 重定向到 stdout,默认 false
stdout\PYGZus{}logfile\PYGZus{}maxbytes = 20MB  ; stdout 日志文件大小,默认 50MB
stdout\PYGZus{}logfile\PYGZus{}backups = 20     ; stdout 日志文件备份数
; stdout 日志文件,需要注意当指定目录不存在时无法正常启动,所以需要手动创建目录(supervisord 会自动创建日志文件)
stdout\PYGZus{}logfile = /data/logs/usercenter\PYGZus{}stdout.log

; 可以通过 environment 来添加需要的环境变量,一种常见的用法是修改 PYTHONPATH
; environment=PYTHONPATH=\PYGZdl{}PYTHONPATH:/path/to/somewhere

[program:cat]
command=/bin/cat
process\PYGZus{}name=\PYGZpc{}(program\PYGZus{}name)s
numprocs=1
directory=/tmp
umask=022
priority=999
autostart=true
autorestart=true
startsecs=10
startretries=3
exitcodes=0,2
stopsignal=TERM
stopwaitsecs=10
user=chrism
redirect\PYGZus{}stderr=false
stdout\PYGZus{}logfile=/a/path
stdout\PYGZus{}logfile\PYGZus{}maxbytes=1MB
stdout\PYGZus{}logfile\PYGZus{}backups=10
stdout\PYGZus{}capture\PYGZus{}maxbytes=1MB
stderr\PYGZus{}logfile=/a/path
stderr\PYGZus{}logfile\PYGZus{}maxbytes=1MB
stderr\PYGZus{}logfile\PYGZus{}backups=10
stderr\PYGZus{}capture\PYGZus{}maxbytes=1MB
environment=A=\PYGZdq{}1\PYGZdq{},B=\PYGZdq{}2\PYGZdq{}
serverurl=AUTO

;*为必须填写项
;*[program:应用名称]
[program:cat]

;*命令路径,如果使用python启动的程序应该为 python /home/test.py,
;不建议放入/home/user/, 对于非user用户一般情况下是不能访问
command=/bin/cat

;当numprocs为1时,process\PYGZus{}name=\PYGZpc{}(program\PYGZus{}name)s
;当numprocs\PYGZgt{}=2时,\PYGZpc{}(program\PYGZus{}name)s\PYGZus{}\PYGZpc{}(process\PYGZus{}num)02d
process\PYGZus{}name=\PYGZpc{}(program\PYGZus{}name)s

;进程数量
numprocs=1

;执行目录,若有/home/supervisor\PYGZus{}test/test1.py
;将directory设置成/home/supervisor\PYGZus{}test
;则command只需设置成python test1.py
;否则command必须设置成绝对执行目录
directory=/tmp

;掩码:\PYGZhy{}\PYGZhy{}\PYGZhy{} \PYGZhy{}w\PYGZhy{} \PYGZhy{}w\PYGZhy{}, 转换后rwx r\PYGZhy{}x w\PYGZhy{}x
umask=022

;优先级,值越高,最后启动,最先被关闭,默认值999
priority=999

;如果是true,当supervisor启动时,程序将会自动启动
autostart=true

;*自动重启
autorestart=true

;启动延时执行,默认1秒
startsecs=10

;启动尝试次数,默认3次
startretries=3

;当退出码是0,2时,执行重启,默认值0,2
exitcodes=0,2

;停止信号,默认TERM
;中断:INT(类似于Ctrl+C)(kill \PYGZhy{}INT pid),退出后会将写文件或日志(推荐)
;终止:TERM(kill \PYGZhy{}TERM pid)
;挂起:HUP(kill \PYGZhy{}HUP pid),注意与Ctrl+Z/kill \PYGZhy{}stop pid不同
;从容停止:QUIT(kill \PYGZhy{}QUIT pid)
;KILL, USR1, USR2其他见命令(kill \PYGZhy{}l),说明1
stopsignal=TERM

stopwaitsecs=10

;*以root用户执行
user=root

;重定向
redirect\PYGZus{}stderr=false

stdout\PYGZus{}logfile=/a/path
stdout\PYGZus{}logfile\PYGZus{}maxbytes=1MB
stdout\PYGZus{}logfile\PYGZus{}backups=10
stdout\PYGZus{}capture\PYGZus{}maxbytes=1MB
stderr\PYGZus{}logfile=/a/path
stderr\PYGZus{}logfile\PYGZus{}maxbytes=1MB
stderr\PYGZus{}logfile\PYGZus{}backups=10
stderr\PYGZus{}capture\PYGZus{}maxbytes=1MB

;环境变量设置
environment=A=\PYGZdq{}1\PYGZdq{},B=\PYGZdq{}2\PYGZdq{}

serverurl=AUTO

command:启动程序使用的命令,可以是绝对路径或者相对路径
process\PYGZus{}name:一个python字符串表达式,用来表示supervisor进程启动的这个的名称,默认值是\PYGZpc{}(program\PYGZus{}name)s
numprocs:Supervisor启动这个程序的多个实例,如果numprocs\PYGZgt{}1,则process\PYGZus{}name的表达式必须包含\PYGZpc{}(process\PYGZus{}num)s,默认是1
numprocs\PYGZus{}start:一个int偏移值,当启动实例的时候用来计算numprocs的值
priority:权重,可以控制程序启动和关闭时的顺序,权重越低:越早启动,越晚关闭。默认值是999
autostart:如果设置为true,当supervisord启动的时候,进程会自动重启。
autorestart:值可以是false、true、unexpected。false:进程不会自动重启,unexpected:当程序退出时的退出码不是exitcodes中定义的时,进程会重启,true:进程会无条件重启当退出的时候。
startsecs:程序启动后等待多长时间后才认为程序启动成功
startretries:supervisord尝试启动一个程序时尝试的次数。默认是3
exitcodes:一个预期的退出返回码,默认是0,2。
stopsignal:当收到stop请求的时候,发送信号给程序,默认是TERM信号,也可以是 HUP, INT, QUIT, KILL, USR1, or USR2。
stopwaitsecs:在操作系统给supervisord发送SIGCHILD信号时等待的时间
stopasgroup:如果设置为true,则会使supervisor发送停止信号到整个进程组
killasgroup:如果设置为true,则在给程序发送SIGKILL信号的时候,会发送到整个进程组,它的子进程也会受到影响。
user:如果supervisord以root运行,则会使用这个设置用户启动子程序
redirect\PYGZus{}stderr:如果设置为true,进程则会把标准错误输出到supervisord后台的标准输出文件描述符。
stdout\PYGZus{}logfile:把进程的标准输出写入文件中,如果stdout\PYGZus{}logfile没有设置或者设置为AUTO,则supervisor会自动选择一个文件位置。
stdout\PYGZus{}logfile\PYGZus{}maxbytes:标准输出log文件达到多少后自动进行轮转,单位是KB、MB、GB。如果设置为0则表示不限制日志文件大小
stdout\PYGZus{}logfile\PYGZus{}backups:标准输出日志轮转备份的数量,默认是10,如果设置为0,则不备份
stdout\PYGZus{}capture\PYGZus{}maxbytes:当进程处于stderr capture mode模式的时候,写入FIFO队列的最大bytes值,单位可以是KB、MB、GB
stdout\PYGZus{}events\PYGZus{}enabled:如果设置为true,当进程在写它的stderr到文件描述符的时候,PROCESS\PYGZus{}LOG\PYGZus{}STDERR事件会被触发
stderr\PYGZus{}logfile:把进程的错误日志输出一个文件中,除非redirect\PYGZus{}stderr参数被设置为true
stderr\PYGZus{}logfile\PYGZus{}maxbytes:错误log文件达到多少后自动进行轮转,单位是KB、MB、GB。如果设置为0则表示不限制日志文件大小
stderr\PYGZus{}logfile\PYGZus{}backups:错误日志轮转备份的数量,默认是10,如果设置为0,则不备份
stderr\PYGZus{}capture\PYGZus{}maxbytes:当进程处于stderr capture mode模式的时候,写入FIFO队列的最大bytes值,单位可以是KB、MB、GB
stderr\PYGZus{}events\PYGZus{}enabled:如果设置为true,当进程在写它的stderr到文件描述符的时候,PROCESS\PYGZus{}LOG\PYGZus{}STDERR事件会被触发
environment:一个k/v对的list列表
directory:supervisord在生成子进程的时候会切换到该目录
umask:设置进程的umask
serverurl:是否允许子进程和内部的HTTP服务通讯,如果设置为AUTO,supervisor会自动的构造一个url
eg:


[program:import]
directory=/home/cxl/git\PYGZhy{}svn/spi/spi\PYGZhy{}php
command=php think queue:work \PYGZhy{}\PYGZhy{}queue=\PYGZdq{}importQueue\PYGZdq{} \PYGZhy{}\PYGZhy{}tries=1 \PYGZhy{}\PYGZhy{}daemon
process\PYGZus{}name=import\PYGZus{}\PYGZpc{}(process\PYGZus{}num)s
numprocs=2
numprocs\PYGZus{}start=1
autostart=true
startsecs=5
autorestart=true
startretries=3
user=www\PYGZhy{}data
;redirect\PYGZus{}stderr=true
;stdout\PYGZus{}logfile\PYGZus{}maxbytes = 20MB
;stdout\PYGZus{}logfile\PYGZus{}backups = 20
;stdout\PYGZus{}logfile = /data/logs/usercenter\PYGZus{}stdout.log
;environment=PYTHONPATH=\PYGZdl{}PYTHONPATH:/path/to/somewhere
导入队列,使用两个进程处理,循环处理
\end{sphinxVerbatim}

\end{itemize}


\chapter{PROGRAM}
\label{\detokenize{program/index:program}}\label{\detokenize{program/index::doc}}

\section{php}
\label{\detokenize{program/php:php}}\label{\detokenize{program/php::doc}}

\subsection{php7功能点}
\label{\detokenize{program/php:php7}}\begin{enumerate}
\item {} 
花括号 \{\}

\begin{sphinxVerbatim}[commandchars=\\\{\}]
1. 表示\PYGZob{}\PYGZcb{}里面的是一个变量  ,执行时按照变量来处理
2. 在字符串中引用变量使用的特殊包括方式,这样就可以不使用.运算符,从而减少代码的输入量了。
\PYGZdl{}s = \PYGZdq{}Di, \PYGZdq{};
echo (\PYGZdq{}\PYGZdl{}\PYGZob{}s\PYGZcb{}omething\PYGZdq{});  == echo \PYGZdl{}s.\PYGZdq{}omething\PYGZdq{};
//Output: Di, omething

PHP 变量后面加上一个大括号\PYGZob{}\PYGZcb{},里面填上数字,就是指 PHP 变量相应序号的字符。
例如:
\PYGZdl{}str = \PYGZsq{}hello\PYGZsq{};
echo \PYGZdl{}str\PYGZob{}0\PYGZcb{}; // 输出为 h
echo \PYGZdl{}str\PYGZob{}1\PYGZcb{}; // 输出为 e
如果要检查某个字符串是否满足多少长度,可以考虑用这种大括号(花括号)加 isset 的方式替代 strlen 函数,因为 isset 是语言结构,strlen 是函数,所以使用 isset 比使用 strlen 效率更高。
比如判断一个字符串的长度是否小于 5:
if ( !isset ( \PYGZdl{}str\PYGZob{}5\PYGZcb{} ) ) 就比 if ( strlen ( \PYGZdl{}str ) \PYGZlt{} 5 ) 好。
\end{sphinxVerbatim}

\item {} 
call\_user\_func \& call\_user\_func\_array

\begin{sphinxVerbatim}[commandchars=\\\{\}]
区别 调用方式不同
    call\PYGZus{}user\PYGZus{}func(array(\PYGZdl{}class,\PYGZdl{}method),param1,param2);
    call\PYGZus{}user\PYGZus{}func\PYGZus{}array(array(\PYGZdl{}class,\PYGZdl{}method),array(param1,param2));
注意如果上面\PYGZdl{}class为字符串,\PYGZdl{}method为非静态方法,则必须要先实例化,比如 \PYGZdl{}class = new \PYGZbs{}app\PYGZbs{}class;
\end{sphinxVerbatim}

\item {} 
method\_exists \& is\_callable

\begin{sphinxVerbatim}[commandchars=\\\{\}]
调用方式不同,is\PYGZus{}callable
    if ( is\PYGZus{}callable( array( \PYGZdl{}obj, \PYGZdl{}method ) ) )
    \PYGZob{}
    /*要操作的代码段*/
    \PYGZcb{}
    method\PYGZus{}exists(\PYGZdl{}obj,\PYGZdl{}method)
    \PYGZdl{}result = is\PYGZus{}callable([\PYGZsq{}\PYGZbs{}app\PYGZbs{}sms\PYGZbs{}model\PYGZbs{}NoDisturbNumber\PYGZsq{},\PYGZsq{}clearData\PYGZsq{}]);
    dump(\PYGZdl{}result);
    \PYGZdl{}result = method\PYGZus{}exists(\PYGZsq{}appsms\PYGZbs{}model\PYGZbs{}NoDisturbNumber\PYGZsq{}, \PYGZdq{}clearData\PYGZdq{});
    dump(\PYGZdl{}result);
其他
    php函数method\PYGZus{}exists()与is\PYGZus{}callable()的区别在于在php5中,一个方法存在并不意味着它就可以被调用。对于 private,protected和public类型的方法,method\PYGZus{}exits()会返回true,但是is\PYGZus{}callable()会检查存在其是否可以访问,如果是private,protected类型的,它会返回false。
\end{sphinxVerbatim}

\item {} 
trait

\begin{sphinxVerbatim}[commandchars=\\\{\}]
属性:如果 trait 定义了一个属性,那类将不能定义同样名称的属性,否则会产生一个错误。
如果类的定义是兼容的(同样的可见性和初始值)则错误的级别是 E\PYGZus{}STRICT,否则是一个致命错误。
\end{sphinxVerbatim}

\item {} 
php7新功能

\begin{sphinxVerbatim}[commandchars=\\\{\}]
1、运算符(NULL 合并运算符)
    \PYGZdl{}a = \PYGZdl{}\PYGZus{}GET[\PYGZsq{}a\PYGZsq{}] ?? 1; 相当于 \PYGZdl{}a = isset(\PYGZdl{}\PYGZus{}GET[\PYGZsq{}a\PYGZsq{}]) ? \PYGZdl{}\PYGZus{}GET[\PYGZsq{}a\PYGZsq{}] : 1;
2、函数返回值类型声明
  \PYGZsh{} declare(strict\PYGZus{}types=1);
    function foo(\PYGZdl{}a):int\PYGZob{}
        return \PYGZdl{}a
    \PYGZcb{}
    foo(1.0); foo 函数返回 int 1,没有任何错误
    去掉上面代码的注释,采用严格模式,则会出发一个 TypeError 的 Fatal error。
3、标量类型声明
    function sumOfints(int ...\PYGZdl{}ints)\PYGZob{}
        return array\PYGZus{}sum(\PYGZdl{}ints)
    \PYGZcb{}
    var\PYGZus{}dump(sumOfInts(2, \PYGZsq{}3\PYGZsq{}, 4.1));
4、use 批量声明
    use some/namespace/\PYGZob{}ClassA, ClassB, ClassC as C\PYGZcb{};
    use function some/namespace/\PYGZob{}fn\PYGZus{}a, fn\PYGZus{}b, fn\PYGZus{}c\PYGZcb{};
    use const some/namespace/\PYGZob{}ConstA, ConstB, ConstC\PYGZcb{};
5、spaceship(\PYGZlt{}=\PYGZgt{})操作符
    用来比较两个表达式,左边小于、等于、大于右边时分别返回\PYGZhy{}1,0,1
6、常量array可以使用define定义
    define(\PYGZsq{}ANIMALS\PYGZsq{}, [
\PYGZsq{}dog\PYGZsq{},
\PYGZsq{}cat\PYGZsq{},
\PYGZsq{}bird\PYGZsq{}
    ]);
    echo ANIMALS[1]; // outputs \PYGZdq{}cat\PYGZdq{}
7、匿名类
    interface Logger \PYGZob{}
public function log(string \PYGZdl{}msg);
    \PYGZcb{}
    class Application \PYGZob{}
        private \PYGZdl{}logger;

        public function getLogger(): Logger \PYGZob{}
             return \PYGZdl{}this\PYGZhy{}\PYGZgt{}logger;
        \PYGZcb{}

        public function setLogger(Logger \PYGZdl{}logger) \PYGZob{}
             \PYGZdl{}this\PYGZhy{}\PYGZgt{}logger = \PYGZdl{}logger;
        \PYGZcb{}
    \PYGZcb{}
    \PYGZdl{}app = new Application;
    \PYGZdl{}app\PYGZhy{}\PYGZgt{}setLogger(new class implements Logger \PYGZob{}
        public function log(string \PYGZdl{}msg) \PYGZob{}
            echo \PYGZdl{}msg;
        \PYGZcb{}
    \PYGZcb{});
    var\PYGZus{}dump(\PYGZdl{}app\PYGZhy{}\PYGZgt{}getLogger());
    The above example will output:
    object(class@anonymous)\PYGZsh{}2 (0) \PYGZob{}
    \PYGZcb{}
8、闭包( Closure)增加了一个 call 方法
    class A \PYGZob{}private \PYGZdl{}x = 1;\PYGZcb{}
    // Pre PHP 7 code
    \PYGZdl{}getX = function() \PYGZob{}return \PYGZdl{}this\PYGZhy{}\PYGZgt{}x;\PYGZcb{};
    \PYGZdl{}getXCB = \PYGZdl{}getX\PYGZhy{}\PYGZgt{}bindTo(new A, \PYGZsq{}A\PYGZsq{}); // intermediate closure
    echo \PYGZdl{}getXCB();

    // PHP 7+ code
    \PYGZdl{}getX = function() \PYGZob{}return \PYGZdl{}this\PYGZhy{}\PYGZgt{}x;\PYGZcb{};
    echo \PYGZdl{}getX\PYGZhy{}\PYGZgt{}call(new A);
\end{sphinxVerbatim}

\end{enumerate}


\subsection{tp5}
\label{\detokenize{program/php:tp5}}

\subsubsection{queue}
\label{\detokenize{program/php:queue}}\begin{itemize}
\item {} 
refer

\begin{sphinxVerbatim}[commandchars=\\\{\}]
\PYG{n}{https}\PYG{p}{:}\PYG{o}{/}\PYG{o}{/}\PYG{n}{github}\PYG{o}{.}\PYG{n}{com}\PYG{o}{/}\PYG{n}{coolseven}\PYG{o}{/}\PYG{n}{notes}\PYG{o}{/}\PYG{n}{blob}\PYG{o}{/}\PYG{n}{master}\PYG{o}{/}\PYG{n}{thinkphp}\PYG{o}{\PYGZhy{}}\PYG{n}{queue}\PYG{o}{/}\PYG{n}{README}\PYG{o}{.}\PYG{n}{md}
\end{sphinxVerbatim}

\item {} 
install

\begin{sphinxVerbatim}[commandchars=\\\{\}]
\PYG{n}{composer} \PYG{n}{require} \PYG{n}{topthink}\PYG{o}{/}\PYG{n}{think}\PYG{o}{\PYGZhy{}}\PYG{n}{queue}
\end{sphinxVerbatim}

\item {} 
例子
\begin{itemize}
\item {} 
配置

\begin{sphinxVerbatim}[commandchars=\\\{\}]
application/extra/queue.php
return [
        \PYGZsq{}connector\PYGZsq{}  =\PYGZgt{} \PYGZsq{}Redis\PYGZsq{},         // Redis 驱动
        \PYGZsq{}expire\PYGZsq{}     =\PYGZgt{} 60,             // 任务的过期时间,默认为60秒; 若要禁用,则设置为 null
        \PYGZsq{}default\PYGZsq{}    =\PYGZgt{} \PYGZsq{}default\PYGZsq{},      // 默认的队列名称
        \PYGZsq{}host\PYGZsq{}       =\PYGZgt{} \PYGZsq{}127.0.0.1\PYGZsq{},        // redis 主机ip
        \PYGZsq{}port\PYGZsq{}       =\PYGZgt{} 6379,           // redis 端口
        \PYGZsq{}password\PYGZsq{}   =\PYGZgt{} \PYGZsq{}\PYGZsq{},             // redis 密码
        \PYGZsq{}select\PYGZsq{}     =\PYGZgt{} 2,                  // 使用哪一个 db,默认为 db1
        \PYGZsq{}timeout\PYGZsq{}    =\PYGZgt{} 0,              // redis连接的超时时间
        \PYGZsq{}persistent\PYGZsq{} =\PYGZgt{} false,          // 是否是长连接
];
\end{sphinxVerbatim}

\item {} 
建立生产者

\begin{sphinxVerbatim}[commandchars=\\\{\}]
public function test()\PYGZob{}
    // 1.当前任务将由哪个类来负责处理。
    //   当轮到该任务时,系统将生成一个该类的实例,并调用其 fire 方法
    \PYGZdl{}jobHandlerClassName  = \PYGZsq{}app\PYGZbs{}job\PYGZbs{}Import\PYGZsq{};
    // 2.当前任务归属的队列名称,如果为新队列,会自动创建
    \PYGZdl{}jobQueueName     = \PYGZdq{}importQueue\PYGZdq{};
    // 3.当前任务所需的业务数据 . 不能为 resource 类型,其他类型最终将转化为json形式的字符串
    //   ( jobData 为对象时,需要在先在此处手动序列化,否则只存储其public属性的键值对)
    \PYGZdl{}jobData          = [ \PYGZsq{}ts\PYGZsq{} =\PYGZgt{} time(), \PYGZsq{}bizId\PYGZsq{} =\PYGZgt{} uniqid() , \PYGZsq{}a\PYGZsq{} =\PYGZgt{} 1 ] ;
    // 4.将该任务推送到消息队列,等待对应的消费者去执行
    \PYGZdl{}isPushed = Queue::push( \PYGZdl{}jobHandlerClassName , \PYGZdl{}jobData , \PYGZdl{}jobQueueName );
    // database 驱动时,返回值为 1\textbar{}false  ;   redis 驱动时,返回值为 随机字符串\textbar{}false
    if( \PYGZdl{}isPushed !== false )\PYGZob{}
        echo date(\PYGZsq{}Y\PYGZhy{}m\PYGZhy{}d H:i:s\PYGZsq{}) . \PYGZdq{} a new Hello Job is Pushed to the MQ [\PYGZdl{}isPushed]\PYGZdq{}.\PYGZdq{}\PYGZlt{}br\PYGZgt{}\PYGZdq{};
    \PYGZcb{}else\PYGZob{}
        echo \PYGZsq{}Oops, something went wrong.\PYGZsq{};
    \PYGZcb{}
\PYGZcb{}
\end{sphinxVerbatim}

\item {} 
查看queue

\begin{sphinxVerbatim}[commandchars=\\\{\}]
\PYG{n}{llen} \PYG{l+s+s2}{\PYGZdq{}}\PYG{l+s+s2}{queues:helloJobQueue}\PYG{l+s+s2}{\PYGZdq{}} \PYG{n}{长度}
\PYG{n}{lpop} \PYG{l+s+s2}{\PYGZdq{}}\PYG{l+s+s2}{queues:helloJobQueue}\PYG{l+s+s2}{\PYGZdq{}} \PYG{n}{出栈}
\PYG{n}{lrange} \PYG{l+s+s2}{\PYGZdq{}}\PYG{l+s+s2}{queues:helloJobQueue}\PYG{l+s+s2}{\PYGZdq{}} \PYG{l+m+mi}{0} \PYG{o}{\PYGZhy{}}\PYG{l+m+mi}{1} \PYG{n}{查询队列内容}
\end{sphinxVerbatim}

\item {} 
建立消费者

\begin{sphinxVerbatim}[commandchars=\\\{\}]
namespace app\PYGZbs{}job;
use think\PYGZbs{}queue\PYGZbs{}Job;
use think\PYGZbs{}Log;

class Import\PYGZob{}

    /**
     * fire方法是消息队列默认调用的方法
     * @param Job            \PYGZdl{}job      当前的任务对象
     * @param array\textbar{}mixed    \PYGZdl{}data     发布任务时自定义的数据
     */
    public function fire(Job \PYGZdl{}job,\PYGZdl{}data)\PYGZob{}
        \PYGZdl{}isJobDone = \PYGZdl{}this\PYGZhy{}\PYGZgt{}doHelloJob(\PYGZdl{}data);

        if (\PYGZdl{}isJobDone) \PYGZob{}
            //如果任务执行成功, 记得删除任务
            \PYGZdl{}job\PYGZhy{}\PYGZgt{}delete();
            print(\PYGZdq{}\PYGZlt{}info\PYGZgt{}Hello Job has been done and deleted\PYGZdq{}.\PYGZdq{}\PYGZlt{}/info\PYGZgt{}\PYGZbs{}n\PYGZdq{});
        \PYGZcb{}else\PYGZob{}
            if (\PYGZdl{}job\PYGZhy{}\PYGZgt{}attempts() \PYGZgt{} 3) \PYGZob{}
                //通过这个方法可以检查这个任务已经重试了几次了
                print(\PYGZdq{}\PYGZlt{}warn\PYGZgt{}Hello Job has been retried more than 3 times!\PYGZdq{}.\PYGZdq{}\PYGZlt{}/warn\PYGZgt{}\PYGZbs{}n\PYGZdq{});
                \PYGZdl{}job\PYGZhy{}\PYGZgt{}delete();
                // 也可以重新发布这个任务
                //print(\PYGZdq{}\PYGZlt{}info\PYGZgt{}Hello Job will be availabe again after 2s.\PYGZdq{}.\PYGZdq{}\PYGZlt{}/info\PYGZgt{}\PYGZbs{}n\PYGZdq{});
                //\PYGZdl{}job\PYGZhy{}\PYGZgt{}release(2); //\PYGZdl{}delay为延迟时间,表示该任务延迟2秒后再执行
            \PYGZcb{}
        \PYGZcb{}
    \PYGZcb{}

    /**
     * 根据消息中的数据进行实际的业务处理
     * @param array\textbar{}mixed    \PYGZdl{}data     发布任务时自定义的数据
     * @return boolean                 任务执行的结果
     */
    private function doHelloJob(\PYGZdl{}data) \PYGZob{}
        // 根据消息中的数据进行实际的业务处理...
        print(\PYGZdq{}\PYGZlt{}info\PYGZgt{}Hello Job Started. job Data is: \PYGZdq{}.var\PYGZus{}export(\PYGZdl{}data,true).\PYGZdq{}\PYGZlt{}/info\PYGZgt{} \PYGZbs{}n\PYGZdq{});
        print(\PYGZdq{}\PYGZlt{}info\PYGZgt{}Hello Job is Fired at \PYGZdq{} . date(\PYGZsq{}Y\PYGZhy{}m\PYGZhy{}d H:i:s\PYGZsq{}) .\PYGZdq{}\PYGZlt{}/info\PYGZgt{} \PYGZbs{}n\PYGZdq{});
        print(\PYGZdq{}\PYGZlt{}info\PYGZgt{}Hello Job is Done!\PYGZdq{}.\PYGZdq{}\PYGZlt{}/info\PYGZgt{} \PYGZbs{}n\PYGZdq{});
        return true;
    \PYGZcb{}

    public function failed(\PYGZdl{}data)\PYGZob{}

        // ...任务达到最大重试次数后,失败了
        print(\PYGZdq{}\PYGZlt{}info\PYGZgt{}Hello Job is failed!\PYGZdq{}.\PYGZdq{}\PYGZlt{}/info\PYGZgt{} \PYGZbs{}n\PYGZdq{});
        Log::write(\PYGZdq{}has fail in import:\PYGZdq{});
    \PYGZcb{}
\PYGZcb{}
\end{sphinxVerbatim}

\item {} 
全局错误类

\begin{sphinxVerbatim}[commandchars=\\\{\}]
1. tags.php 标签配置
// 任务失败统一回调,有四种定义方式
\PYGZsq{}queue\PYGZus{}failed\PYGZsq{}=\PYGZgt{} [
            [\PYGZsq{}app\PYGZbs{}\PYGZbs{}common\PYGZbs{}\PYGZbs{}behavior\PYGZbs{}\PYGZbs{}MyQueueFailedLogger\PYGZsq{}, \PYGZsq{}logAllFailedQueues\PYGZsq{}]
        ],
2. fail方法
    namespace app\PYGZbs{}common\PYGZbs{}behavior;
    use think\PYGZbs{}Log;
    class MyQueueFailedLogger\PYGZob{}
        const should\PYGZus{}run\PYGZus{}hook\PYGZus{}callback = true;

        /**
         * @param \PYGZdl{}jobObject   \PYGZbs{}think\PYGZbs{}queue\PYGZbs{}Job   //任务对象,保存了该任务的执行情况和业务数据
         * @return bool     true                  //是否需要删除任务并触发其failed() 方法
         */
        public function logAllFailedQueues(\PYGZam{}\PYGZdl{}jobObject)\PYGZob{}

            /* \PYGZdl{}failedJobLog = [
                    \PYGZsq{}jobHandlerClassName\PYGZsq{}   =\PYGZgt{} \PYGZdl{}jobObject\PYGZhy{}\PYGZgt{}getName(), // \PYGZsq{}application\PYGZbs{}index\PYGZbs{}job\PYGZbs{}Hello\PYGZsq{}
                    \PYGZsq{}queueName\PYGZsq{} =\PYGZgt{} \PYGZdl{}jobObject\PYGZhy{}\PYGZgt{}getQueue(),             // \PYGZsq{}helloJobQueue\PYGZsq{}
                    \PYGZsq{}jobData\PYGZsq{}   =\PYGZgt{} \PYGZdl{}jobObject\PYGZhy{}\PYGZgt{}getRawBody()[\PYGZsq{}data\PYGZsq{}],  // \PYGZsq{}\PYGZob{}\PYGZsq{}a\PYGZsq{}: 1 \PYGZcb{}\PYGZsq{}
                    \PYGZsq{}attempts\PYGZsq{}  =\PYGZgt{} \PYGZdl{}jobObject\PYGZhy{}\PYGZgt{}attempts(),            // 3
            ];
            var\PYGZus{}export(json\PYGZus{}encode(\PYGZdl{}failedJobLog,true)); */
            Log::write(\PYGZdq{}i am in failed method\PYGZdq{});
            //Log::write(\PYGZdq{}failed in \PYGZdq{}.json\PYGZus{}encode(\PYGZdl{}failedJobLog,true));
            // \PYGZdl{}jobObject\PYGZhy{}\PYGZgt{}release();     //重发任务
            //\PYGZdl{}jobObject\PYGZhy{}\PYGZgt{}delete();         //删除任务
            //\PYGZdl{}jobObject\PYGZhy{}\PYGZgt{}failed();   //通知消费者类任务执行失败

            return self::should\PYGZus{}run\PYGZus{}hook\PYGZus{}callback;
        \PYGZcb{}
    \PYGZcb{}
\end{sphinxVerbatim}

\item {} 
测试

\begin{sphinxVerbatim}[commandchars=\\\{\}]
\PYG{l+m+mf}{1.} \PYG{n}{https}\PYG{p}{:}\PYG{o}{/}\PYG{o}{/}\PYG{l+m+mf}{10.0}\PYG{o}{.}\PYG{l+m+mf}{0.42}\PYG{p}{:}\PYG{l+m+mi}{1066}\PYG{o}{/}\PYG{n}{home}\PYG{o}{/}\PYG{n}{index}\PYG{o}{/}\PYG{n}{test}
    \PYG{l+m+mi}{2017}\PYG{o}{\PYGZhy{}}\PYG{l+m+mi}{09}\PYG{o}{\PYGZhy{}}\PYG{l+m+mi}{23} \PYG{l+m+mi}{10}\PYG{p}{:}\PYG{l+m+mi}{04}\PYG{p}{:}\PYG{l+m+mi}{18} \PYG{n}{a} \PYG{n}{new} \PYG{n}{Hello} \PYG{n}{Job} \PYG{o+ow}{is} \PYG{n}{Pushed} \PYG{n}{to} \PYG{n}{the} \PYG{n}{MQ} \PYG{p}{[}\PYG{n}{A25ZVpTcRIDUMyUgwSiRy76ebkLPuck8}\PYG{p}{]}
\PYG{l+m+mf}{2.} \PYG{n}{php} \PYG{n}{think} \PYG{n}{queue}\PYG{p}{:}\PYG{n}{work} \PYG{o}{\PYGZhy{}}\PYG{o}{\PYGZhy{}}\PYG{n}{queue}\PYG{o}{=}\PYG{l+s+s2}{\PYGZdq{}}\PYG{l+s+s2}{importQueue}\PYG{l+s+s2}{\PYGZdq{}} \PYG{o}{\PYGZhy{}}\PYG{o}{\PYGZhy{}}\PYG{n}{daemon}
\end{sphinxVerbatim}

\item {} 
使用supervisor管理进程

\begin{sphinxVerbatim}[commandchars=\\\{\}]
[program:import]
directory=/home/cxl/git\PYGZhy{}svn/spi/spi\PYGZhy{}php
command=php think queue:work \PYGZhy{}\PYGZhy{}queue=\PYGZdq{}importQueue\PYGZdq{} \PYGZhy{}\PYGZhy{}tries=1 \PYGZhy{}\PYGZhy{}daemon
process\PYGZus{}name=import\PYGZus{}\PYGZpc{}(process\PYGZus{}num)s
numprocs=2
numprocs\PYGZus{}start=1
autostart=true
startsecs=5
autorestart=true
startretries=3
user=www\PYGZhy{}data
;redirect\PYGZus{}stderr=true
;stdout\PYGZus{}logfile\PYGZus{}maxbytes = 20MB
;stdout\PYGZus{}logfile\PYGZus{}backups = 20
;stdout\PYGZus{}logfile = /data/logs/usercenter\PYGZus{}stdout.log
;environment=PYTHONPATH=\PYGZdl{}PYTHONPATH:/path/to/somewhere
导入队列,使用两个进程处理,循环处理
kill一个进程会自动重启
\end{sphinxVerbatim}

\end{itemize}

\end{itemize}


\section{shell}
\label{\detokenize{program/shell:shell}}\label{\detokenize{program/shell::doc}}

\section{mysql}
\label{\detokenize{program/mysql::doc}}\label{\detokenize{program/mysql:mysql}}

\subsection{mysqldump}
\label{\detokenize{program/mysql:mysqldump}}

\subsubsection{refer}
\label{\detokenize{program/mysql:refer}}\begin{quote}

\sphinxurl{https://www.cnblogs.com/chenmh/p/5300370.html}
\end{quote}


\subsubsection{example}
\label{\detokenize{program/mysql:example}}

\subsection{功能点}
\label{\detokenize{program/mysql:id1}}

\subsubsection{索引长度限制}
\label{\detokenize{program/mysql:id2}}\begin{itemize}
\item {} \begin{description}
\item[{refer}] \leavevmode
\sphinxurl{http://blog.csdn.net/shaochenshuo/article/details/51064685}

\end{description}

\item {} \begin{description}
\item[{explain}] \leavevmode
From the manual at \sphinxurl{http://dev.mysql.com/doc/refman/5.6/en/create-table.html} \textgreater{}\textgreater{}从5.6的官方文档中我们能找到如下双引号中解释
“For CHAR, VARCHAR, BINARY, and VARBINARY columns, indexes can be created that use only the leading part of column values, using col\_name(length) syntax to specify an index prefix length.
…
Prefixes can be up to 1000 bytes long (767 bytes for InnoDB tables). Note that prefix limits are measured in bytes, whereas the prefix length in CREATE TABLE statements is interpreted as number of characters …”\textgreater{}\textgreater{}\textgreater{}对于myisam和innodb存储引擎,prefixes的长度限制分别为1000 bytes和767 bytes。注意prefix的单位是bytes,但是建表时我们指定的长度单位是字符。
A utf8 character can use up to 3 bytes. Hence you cannot index columns or prefixes of columns longer than 333 (MyISAM) or 255 (InnoDB) utf8 characters.  \textgreater{}\textgreater{}以utf8字符集为例,一个字符占3个bytes。因此在utf8字符集下,对myisam和innodb存储引擎创建索引的单列长度不能超过333个字符和255个字符

\end{description}

\end{itemize}


\subsubsection{修改mysql root密码}
\label{\detokenize{program/mysql:mysql-root}}\begin{enumerate}
\item {} 
use mysql;update user set password=PASSWORD(“C1oudP8x\&2017”) where user=”root”;

\item {} 
flush privileges;

\item {} 
select Host,User,Password,authentication\_string from user;

\end{enumerate}


\subsubsection{语法}
\label{\detokenize{program/mysql:id3}}\begin{itemize}
\item {} 
insert into … on duplicate key update …

\begin{sphinxVerbatim}[commandchars=\\\{\}]
\PYG{n}{INSERT} \PYG{n}{INTO} \PYG{n}{table} \PYG{p}{(}\PYG{n}{a}\PYG{p}{,}\PYG{n}{b}\PYG{p}{,}\PYG{n}{c}\PYG{p}{)} \PYG{n}{VALUES} \PYG{p}{(}\PYG{l+m+mi}{1}\PYG{p}{,}\PYG{l+m+mi}{2}\PYG{p}{,}\PYG{l+m+mi}{3}\PYG{p}{)} \PYG{n}{ON} \PYG{n}{DUPLICATE} \PYG{n}{KEY} \PYG{n}{UPDATE} \PYG{n}{c}\PYG{o}{=}\PYG{n}{c}\PYG{o}{+}\PYG{l+m+mi}{1} \PYG{n}{UPDATE} \PYG{n}{table} \PYG{n}{SET} \PYG{n}{c}\PYG{o}{=}\PYG{n}{c}\PYG{o}{+}\PYG{l+m+mi}{1} \PYG{n}{WHERE} \PYG{n}{a}\PYG{o}{=}\PYG{l+m+mi}{1}\PYG{p}{;}
\end{sphinxVerbatim}

\end{itemize}


\section{vue}
\label{\detokenize{program/vue:vue}}\label{\detokenize{program/vue::doc}}

\subsection{install}
\label{\detokenize{program/vue:install}}

\subsection{get\_start}
\label{\detokenize{program/vue:get-start}}

\subsubsection{创建项目     ::}
\label{\detokenize{program/vue:id1}}
vue init webpack vue\_test


\section{redis}
\label{\detokenize{program/redis:redis}}\label{\detokenize{program/redis::doc}}

\subsection{refer}
\label{\detokenize{program/redis:refer}}\begin{quote}

\sphinxurl{http://doc.redisfans.com/}
\sphinxurl{http://www.redis.net.cn/tutorial/3501.html}
\sphinxurl{http://www.cnblogs.com/liuconglin/p/5847568.html}
这是一篇讲各种数据类型的应用场景的文章,可做参考:\sphinxurl{http://blog.csdn.net/gaogaoshan/article/details/41039581/}
redislive 是一款redis使用状态的软件 具体可参考:\sphinxurl{http://blog.csdn.net/hz\_blog/article/details/41822825}
这篇文章讲了此工具如何安装 具体不做说明 地址:\sphinxurl{http://www.cnblogs.com/hepingqingfeng/p/6107809.html}
\end{quote}


\subsection{install\&research}
\label{\detokenize{program/redis:install-research}}

\subsubsection{安装}
\label{\detokenize{program/redis:id1}}
\begin{DUlineblock}{0em}
\item[] \$ wget \sphinxurl{http://download.redis.io/releases/redis-3.2.6.tar.gz}
\item[] \$ tar xzf redis-3.2.6.tar.gz
\item[] \$ cd redis-3.2.6
\item[] \$ make
\item[] \$ src/redis-server 运行redis服务
\item[] \$ src/redis-cli
\item[] redis\textgreater{} set foo bar
\item[] OK
\item[] redis\textgreater{} get foo
\item[] “bar”
\end{DUlineblock}


\subsubsection{配置}
\label{\detokenize{program/redis:id2}}\begin{itemize}
\item {} 
redis配置

\begin{sphinxVerbatim}[commandchars=\\\{\}]
vi redis.conf
bind 10.0.0.23
daemonize yes
logfile \PYGZdq{}/var/pbx/tmp/Logs/redis/redis\PYGZus{}6379.log\PYGZdq{}
dbfilename dump\PYGZus{}6379.rdb
//生产环境
rename\PYGZhy{}command FLUSHALL \PYGZdq{}\PYGZdq{}
rename\PYGZhy{}command FLUSHDB \PYGZdq{}\PYGZdq{}
rename\PYGZhy{}command KEYS \PYGZdq{}\PYGZdq{}
rename\PYGZhy{}command CONFIG \PYGZdq{}\PYGZdq{}
maxmemory 2gb //占用的最大内存 \PYGZlt{}span style=\PYGZsq{}color:green;font\PYGZhy{}weight:bold;\PYGZsq{}\PYGZgt{}\PYGZlt{}一般推荐Redis设置内存为最大物理内存的四分之三,设置最大内存后一般需设置过期策略\PYGZgt{}\PYGZlt{}/span\PYGZgt{}
appendonly yes
appendfilename \PYGZdq{}appendonly\PYGZus{}6379.aof\PYGZdq{}
requirepass \PYGZdq{}prettydogKnockTheDoor\PYGZdq{} //需要密码才能访问
将redis加入到系统自启动的脚本中

Redis使用超过设置的最大值
打开debug模式下的页面,提示错误:OOM command not allowed when used memory \PYGZgt{} ‘maxmemory’.
设置了maxmemory的选项,redis内存使用达到上限。可以通过设置LRU算法来删除部分key,释放空间。默认是按照过期时间的,如果set时候没有加上过期时间就会导致数据写满maxmemory。
如果不设置maxmemory或者设置为0,64位系统不限制内存,32位系统最多使用3GB内存。

LRU是Least Recently Used 近期最少使用算法。
volatile\PYGZhy{}lru \PYGZhy{}\PYGZgt{} 根据LRU算法生成的过期时间来删除。
allkeys\PYGZhy{}lru \PYGZhy{}\PYGZgt{} 根据LRU算法删除任何key。
volatile\PYGZhy{}random \PYGZhy{}\PYGZgt{} 根据过期设置来随机删除key。
allkeys\PYGZhy{}\PYGZgt{}random \PYGZhy{}\PYGZgt{} 无差别随机删。
volatile\PYGZhy{}ttl \PYGZhy{}\PYGZgt{} 根据最近过期时间来删除(辅以TTL)
noeviction \PYGZhy{}\PYGZgt{} 谁也不删,直接在写操作时返回错误。
\end{sphinxVerbatim}

\end{itemize}

\begin{sphinxadmonition}{note}{Note:}
详细配置可参考文章:\sphinxtitleref{这篇文章主要讲的是配置的含义和优化 \textless{}http://www.tuicool.com/articles/MvMBf2\textgreater{}}
\end{sphinxadmonition}
\begin{itemize}
\item {} 
iptable配置

\begin{sphinxVerbatim}[commandchars=\\\{\}]
考虑到安全,不允许外网用户访问redis

10.0.0.0/24 表示C类地址 24掩码前面的0
iptables \PYGZhy{}A INPUT ! \PYGZhy{}s 10.0.0.0/24 \PYGZhy{}p tcp \PYGZhy{}\PYGZhy{}dport 6379 \PYGZhy{}j DROP //拒绝所有非本网段的机器访问redis
iptables \PYGZhy{}A INPUT ! \PYGZhy{}s 10.0.0.0/24 \PYGZhy{}p udp \PYGZhy{}\PYGZhy{}dport 6379 \PYGZhy{}j DROP
保存iptables规则下次启动时自动启动规则
iptables\PYGZhy{}save \PYGZgt{} /etc/iptables.up.rules //保存规则
vi /etc/network/interfaces
pre\PYGZhy{}up iptables\PYGZhy{}restore \PYGZlt{} /etc/iptables.up.rules  //恢复规则
iptables \PYGZhy{}F
重启网络
/etc/init.d/networking restart
iptables \PYGZhy{}L \PYGZhy{}v
重启机器
iptables \PYGZhy{}L \PYGZhy{}v
可以看到刚刚配置的规则都在
\end{sphinxVerbatim}

\end{itemize}


\subsubsection{测试}
\label{\detokenize{program/redis:id3}}\begin{itemize}
\item {} 
redis数据类型测试

\begin{sphinxVerbatim}[commandchars=\\\{\}]
设置值,获取值
* set foo \PYGZdq{}bar\PYGZdq{}
* get foo
* expire foo 120
* ttl foo 查询过期时间 返回\PYGZhy{}2值不存在 \PYGZhy{}1 值永不过期

自增变量
set connections 10
incr connections
del connections

list 队列
rpush friends \PYGZdq{}alice\PYGZdq{} 添加到队尾
rpush friends \PYGZdq{}bob\PYGZdq{}
lpush friends \PYGZdq{}sam\PYGZdq{} 添加到队首
lrange friends 0 \PYGZhy{}1 获取全部队列
lrange friends 0 1 获取0\PYGZhy{}1
llen friends 队列长度
lpop friends 弹出首个对象
rpop friends 弹出队尾对象

set 与list相似,但没有顺序且对象只能出现一次
sadd superpowers \PYGZdq{}flight\PYGZdq{}   添加值
sadd superpowers \PYGZdq{}x\PYGZhy{}ray vision\PYGZdq{}
sadd superpowers \PYGZdq{}reflexes\PYGZdq{}
srem superpowers \PYGZdq{}reflexes\PYGZdq{}  删除值
sismember superpowers \PYGZdq{}reflexes\PYGZdq{}  判断是否存在0不存在1存在
smembers superpowers
sadd birdpowers \PYGZdq{}flight\PYGZdq{}
sadd birdpowers \PYGZdq{}pecking\PYGZdq{}
sunion superpowers birdpowers 合并set

sorted set 可排序set
zadd hackers 1940 \PYGZdq{}alan key\PYGZdq{}
zadd hackers 1980 \PYGZdq{}Grace hopper\PYGZdq{}
zadd hackers 1945 \PYGZdq{}richard stallman\PYGZdq{}
zadd hackers 1944 \PYGZdq{}macker\PYGZdq{}
zrange hackers 0 \PYGZhy{}1

hashes
hset user:1000 name \PYGZdq{}john smith\PYGZdq{}
hset user:1000 email \PYGZdq{}john.smith@example.com\PYGZdq{}
hset user:1000 password \PYGZdq{}s3cret\PYGZdq{}
hgetall user:1000
HMSET user:1001 name \PYGZdq{}Mary Jones\PYGZdq{} password \PYGZdq{}hidden\PYGZdq{} email \PYGZdq{}mjones@example.com\PYGZdq{}
SET user:1000 visits 10
HINCRBY user:1000 visits 1 加1
hdel user:1000 visits 删除值
\end{sphinxVerbatim}

\end{itemize}

\begin{sphinxadmonition}{note}{Note:}
\sphinxtitleref{测试参考文档 \textless{}http://www.cnblogs.com/silent2012/p/4514901.html\textgreater{}}
\end{sphinxadmonition}
\begin{itemize}
\item {} 
安全测试
\begin{itemize}
\item {} 
只有使用密码才能访问

\begin{sphinxVerbatim}[commandchars=\\\{\}]
\PYG{n}{root}\PYG{n+nd}{@1604developer}\PYG{p}{:}\PYG{o}{/}\PYG{n}{var}\PYG{c+c1}{\PYGZsh{} telnet 10.0.0.23 6379}
\PYG{n}{Trying} \PYG{l+m+mf}{10.0}\PYG{o}{.}\PYG{l+m+mf}{0.23}\PYG{o}{.}\PYG{o}{.}\PYG{o}{.}
\PYG{n}{telnet}\PYG{p}{:} \PYG{n}{Unable} \PYG{n}{to} \PYG{n}{connect} \PYG{n}{to} \PYG{n}{remote} \PYG{n}{host}\PYG{p}{:} \PYG{n}{Connection} \PYG{n}{refused}
\PYG{n}{root}\PYG{n+nd}{@1604developer}\PYG{p}{:}\PYG{o}{/}\PYG{n}{var}\PYG{c+c1}{\PYGZsh{} telnet 10.0.0.23 6379}
\PYG{n}{Trying} \PYG{l+m+mf}{10.0}\PYG{o}{.}\PYG{l+m+mf}{0.23}\PYG{o}{.}\PYG{o}{.}\PYG{o}{.}
\PYG{n}{Connected} \PYG{n}{to} \PYG{l+m+mf}{10.0}\PYG{o}{.}\PYG{l+m+mf}{0.23}\PYG{o}{.}
\PYG{n}{Escape} \PYG{n}{character} \PYG{o+ow}{is} \PYG{l+s+s1}{\PYGZsq{}}\PYG{l+s+s1}{\PYGZca{}]}\PYG{l+s+s1}{\PYGZsq{}}\PYG{o}{.}
\PYG{n}{auth} \PYG{n}{prettydogKnockTheDoor}
\PYG{o}{+}\PYG{n}{OK}
\PYG{n}{keys} \PYG{o}{*}
\PYG{o}{*}\PYG{l+m+mi}{0}
\end{sphinxVerbatim}

\item {} 
指定ip才能访问

\begin{sphinxVerbatim}[commandchars=\\\{\}]
iptables \PYGZhy{}L \PYGZhy{}v \PYGZhy{}\PYGZhy{}line\PYGZhy{}num
配置iptables只允许10.0.0.1\PYGZhy{}32的ip地址访问redis
iptables \PYGZhy{}A INPUT ! \PYGZhy{}s 10.0.0.0/27 \PYGZhy{}p tcp \PYGZhy{}\PYGZhy{}dport 6379 \PYGZhy{}j DROP
iptables \PYGZhy{}A INPUT ! \PYGZhy{}s 10.0.0.0/27 \PYGZhy{}p udp \PYGZhy{}\PYGZhy{}dport 6379 \PYGZhy{}j DROP
在本机访问
root@ubuntu1204base:/home/cxl/redis/redis\PYGZhy{}3.2.6\PYGZsh{} telnet 10.0.0.23 6379
Trying 10.0.0.23...
Connected to 10.0.0.23.
Escape character is \PYGZsq{}\PYGZca{}]\PYGZsq{}.
auth prettydogKnockTheDoor
+OK
keys *
*0
在10.0.0.237机器访问
root@php1 16:51:00:\PYGZti{}\PYGZsh{} telnet 10.0.0.23 6379
Trying 10.0.0.23...
如果策略修改为
iptables \PYGZhy{}A INPUT ! \PYGZhy{}s 10.0.0.0/27 \PYGZhy{}p tcp \PYGZhy{}\PYGZhy{}dport 6379 \PYGZhy{}j REJECT
iptables \PYGZhy{}A INPUT ! \PYGZhy{}s 10.0.0.0/27 \PYGZhy{}p udp \PYGZhy{}\PYGZhy{}dport 6379 \PYGZhy{}j REJECT
再次在10.0.0.237机器访问
root@php1 16:53:23:\PYGZti{}\PYGZsh{} telnet 10.0.0.23 6379
Trying 10.0.0.23...
telnet: Unable to connect to remote host: Connection refused
\end{sphinxVerbatim}

\end{itemize}

\item {} 
数据恢复测试
\begin{itemize}
\item {} 
kill redis 进程

\begin{sphinxVerbatim}[commandchars=\\\{\}]
root@ubuntu1204base:/home/cxl/redis/redis\PYGZhy{}3.2.6\PYGZsh{} src/redis\PYGZhy{}cli \PYGZhy{}h 10.0.0.23
10.0.0.23:6379\PYGZgt{} auth prettydogKnockTheDoor
OK
10.0.0.23:6379\PYGZgt{} set \PYGZdq{}first Key\PYGZdq{} \PYGZdq{}first Value\PYGZdq{}
OK
10.0.0.23:6379\PYGZgt{} get \PYGZdq{}first Key\PYGZdq{}
\PYGZdq{}first Value\PYGZdq{}
10.0.0.23:6379\PYGZgt{} quit
root@ubuntu1204base:/home/cxl/redis/redis\PYGZhy{}3.2.6\PYGZsh{} ps \PYGZhy{}aux \textbar{}grep redis
Warning: bad ps syntax, perhaps a bogus \PYGZsq{}\PYGZhy{}\PYGZsq{}? See http://procps.sf.net/faq.html
root      4418  0.1  0.3  36124  1564 ?        Ssl  17:54   0:00 src/redis\PYGZhy{}server 10.0.0.23:6379
root      4497  0.0  0.1   9380   936 pts/2    S+   17:57   0:00 grep \PYGZhy{}\PYGZhy{}color=auto redis
root@ubuntu1204base:/home/cxl/redis/redis\PYGZhy{}3.2.6\PYGZsh{} kill 4418
root@ubuntu1204base:/home/cxl/redis/redis\PYGZhy{}3.2.6\PYGZsh{} src/redis\PYGZhy{}server redis.conf
root@ubuntu1204base:/home/cxl/redis/redis\PYGZhy{}3.2.6\PYGZsh{} src/redis\PYGZhy{}cli \PYGZhy{}h 10.0.0.23
10.0.0.23:6379\PYGZgt{} auth prettydogKnockTheDoor
OK
10.0.0.23:6379\PYGZgt{} get \PYGZdq{}first Key\PYGZdq{}
\PYGZdq{}first Value\PYGZdq{}
10.0.0.23:6379\PYGZgt{} quit
\end{sphinxVerbatim}

\item {} 
重启redis进程,查看redis数据

\begin{sphinxVerbatim}[commandchars=\\\{\}]
\PYG{n}{root}\PYG{n+nd}{@ubuntu1204base}\PYG{p}{:}\PYG{o}{/}\PYG{n}{home}\PYG{o}{/}\PYG{n}{cxl}\PYG{o}{/}\PYG{n}{redis}\PYG{o}{/}\PYG{n}{redis}\PYG{o}{\PYGZhy{}}\PYG{l+m+mf}{3.2}\PYG{o}{.}\PYG{l+m+mi}{6}\PYG{c+c1}{\PYGZsh{} reboot}
\PYG{n}{root}\PYG{n+nd}{@ubuntu1204base}\PYG{p}{:}\PYG{o}{/}\PYG{n}{home}\PYG{o}{/}\PYG{n}{cxl}\PYG{o}{/}\PYG{n}{redis}\PYG{o}{/}\PYG{n}{redis}\PYG{o}{\PYGZhy{}}\PYG{l+m+mf}{3.2}\PYG{o}{.}\PYG{l+m+mi}{6}\PYG{c+c1}{\PYGZsh{} src/redis\PYGZhy{}server redis.conf}
\PYG{n}{root}\PYG{n+nd}{@ubuntu1204base}\PYG{p}{:}\PYG{o}{/}\PYG{n}{home}\PYG{o}{/}\PYG{n}{cxl}\PYG{o}{/}\PYG{n}{redis}\PYG{o}{/}\PYG{n}{redis}\PYG{o}{\PYGZhy{}}\PYG{l+m+mf}{3.2}\PYG{o}{.}\PYG{l+m+mi}{6}\PYG{c+c1}{\PYGZsh{} src/redis\PYGZhy{}cli \PYGZhy{}h 10.0.0.23}
\PYG{l+m+mf}{10.0}\PYG{o}{.}\PYG{l+m+mf}{0.23}\PYG{p}{:}\PYG{l+m+mi}{6379}\PYG{o}{\PYGZgt{}} \PYG{n}{auth} \PYG{n}{prettydogKnockTheDoor}
\PYG{n}{OK}
\PYG{l+m+mf}{10.0}\PYG{o}{.}\PYG{l+m+mf}{0.23}\PYG{p}{:}\PYG{l+m+mi}{6379}\PYG{o}{\PYGZgt{}} \PYG{n}{get} \PYG{l+s+s2}{\PYGZdq{}}\PYG{l+s+s2}{first Key}\PYG{l+s+s2}{\PYGZdq{}}
\PYG{l+s+s2}{\PYGZdq{}}\PYG{l+s+s2}{first Value}\PYG{l+s+s2}{\PYGZdq{}}
\PYG{l+m+mf}{10.0}\PYG{o}{.}\PYG{l+m+mf}{0.23}\PYG{p}{:}\PYG{l+m+mi}{6379}\PYG{o}{\PYGZgt{}}
\end{sphinxVerbatim}

\end{itemize}

\item {} 
性能测试
\begin{itemize}
\item {} 
内存占用测试

\begin{sphinxVerbatim}[commandchars=\\\{\}]
vi redis.conf
maxmemory 1m //为了测试方便设置为1mb
vi addKey.sh
\PYGZsh{} !/usr/bin/bash

echo \PYGZdq{}AUTH prettydogKnockTheDoor\PYGZdq{}
sleep 1

outstr=\PYGZdq{}\PYGZdq{}
value10=\PYGZdq{}1\PYGZdq{}
j=700
while [ \PYGZdl{}j \PYGZhy{}gt 690 ]
do
for i in \PYGZob{}100..199\PYGZcb{}
do
outstr=\PYGZdq{}\PYGZdl{}\PYGZob{}outstr\PYGZcb{}set key\PYGZdl{}j\PYGZhy{}\PYGZdl{}i \PYGZdl{}\PYGZob{}value10\PYGZcb{}\PYGZbs{}r\PYGZbs{}n\PYGZdq{}
done
printf \PYGZdq{}\PYGZdl{}outstr\PYGZdq{}
outstr=\PYGZdq{}\PYGZdq{}
j={}`expr \PYGZdl{}j \PYGZhy{} 1{}`
done
这个脚本往redis服务器加入1000个key,其中Key值10个字节,value值一个字节

10.0.0.23:6379\PYGZgt{} flushall //清空redis
10.0.0.23:6379\PYGZgt{} info
\PYGZsh{} Memory
used\PYGZus{}memory:821552
used\PYGZus{}memory\PYGZus{}human:803.70K
看出来,redis初始状态就占用803.70k,所以设置1m最大内存,实际可用应该是200k左右
运行脚本插入1000个键值

\PYGZgt{}方式1
./addkey.sh \textbar{} nc 10.0.0.23 6379

\PYGZgt{}方式2
./addkey.sh \textbar{} ../redis\PYGZhy{}3.2.6/src/redis\PYGZhy{}cli \PYGZhy{}h 10.0.0.23 \PYGZhy{}a prettydogKnockTheDoor \PYGZhy{}\PYGZhy{}pipe
All data transferred. Waiting for the last reply...
Last reply received from server.
errors: 0, replies: 100001


10.0.0.23:6379\PYGZgt{} info
\PYGZsh{} Memory
used\PYGZus{}memory:871184
used\PYGZus{}memory\PYGZus{}human:850.77K

占用内存47k,所以11字节的key+value大概占用47字节的内存空间
修改一下脚本加入4000key
\PYGZhy{}OOM command not allowed when used memory \PYGZgt{} \PYGZsq{}maxmemory\PYGZsq{}.
10.0.0.23:6379\PYGZgt{} dbsize
(integer) 3109
只加入了3109键值
used\PYGZus{}memory:980120
used\PYGZus{}memory\PYGZus{}human:957.15K
\end{sphinxVerbatim}

\item {} 
大并发测试

\begin{sphinxVerbatim}[commandchars=\\\{\}]
50个客户端发送100000请求每个请求3个字节,看起来redis很给力
root@ubuntu1204base:/home/cxl/redis/redis\PYGZhy{}3.2.6\PYGZsh{} src/redis\PYGZhy{}benchmark \PYGZhy{}h 10.0.0.23 \PYGZhy{}q (\PYGZlt{}span style=\PYGZsq{}color:red;font\PYGZhy{}weight:bold\PYGZsq{}\PYGZgt{}本机测试\PYGZlt{}/span\PYGZgt{})
PING\PYGZus{}INLINE: 75131.48 requests per second
PING\PYGZus{}BULK: 74074.07 requests per second
SET: 76277.65 requests per second
GET: 74349.44 requests per second
INCR: 75815.01 requests per second
LPUSH: 74074.07 requests per second
RPUSH: 73421.44 requests per second
LPOP: 74794.31 requests per second
RPOP: 74571.22 requests per second
SADD: 74906.37 requests per second
SPOP: 74794.31 requests per second
LPUSH (needed to benchmark LRANGE): 73367.57 requests per second
LRANGE\PYGZus{}100 (first 100 elements): 72621.64 requests per second
LRANGE\PYGZus{}300 (first 300 elements): 74515.65 requests per second
LRANGE\PYGZus{}500 (first 450 elements): 74019.25 requests per second
LRANGE\PYGZus{}600 (first 600 elements): 76277.65 requests per second
MSET (10 keys): 73746.31 requests per second

\PYGZlt{}span style=\PYGZsq{}color:red;font\PYGZhy{}weight:bold\PYGZsq{}\PYGZgt{}内网跨服务器测试\PYGZlt{}/span\PYGZgt{}
PING\PYGZus{}INLINE: 13596.19 requests per second
PING\PYGZus{}BULK: 13696.75 requests per second
SET: 13676.15 requests per second
GET: 13738.15 requests per second
INCR: 13738.15 requests per second
LPUSH: 13664.94 requests per second
RPUSH: 13585.11 requests per second
LPOP: 13730.61 requests per second
RPOP: 13719.30 requests per second
SADD: 13702.38 requests per second
SPOP: 13691.13 requests per second
LPUSH (needed to benchmark LRANGE): 13655.61 requests per second
LRANGE\PYGZus{}100 (first 100 elements): 13650.01 requests per second
LRANGE\PYGZus{}300 (first 300 elements): 13648.15 requests per second
LRANGE\PYGZus{}500 (first 450 elements): 13550.14 requests per second
LRANGE\PYGZus{}600 (first 600 elements): 13674.28 requests per second
MSET (10 keys): 13220.52 requests per second
\end{sphinxVerbatim}

\end{itemize}

\end{itemize}

\begin{sphinxadmonition}{note}{Note:}
结论:本机访问是内网跨服务器访问各类请求每秒执行数量五倍多
\end{sphinxadmonition}
\begin{itemize}
\item {} 
监控性能

\begin{sphinxVerbatim}[commandchars=\\\{\}]
\PYG{o}{\PYGZgt{}\PYGZgt{}}\PYG{n}{top监控}
 \PYG{n}{root}\PYG{n+nd}{@ubuntu1204base}\PYG{p}{:}\PYG{o}{/}\PYG{n}{home}\PYG{o}{/}\PYG{n}{cxl}\PYG{o}{/}\PYG{n}{redis}\PYG{o}{/}\PYG{n}{redis}\PYG{o}{\PYGZhy{}}\PYG{l+m+mf}{3.2}\PYG{o}{.}\PYG{l+m+mi}{6}\PYG{c+c1}{\PYGZsh{}  top \PYGZhy{}b \PYGZhy{}p 16417 \PYGZhy{}n 2\textbar{}egrep \PYGZdq{}16417\textbar{}PID\PYGZdq{}\textbar{}tail \PYGZhy{}2}
 \PYG{n}{PID} \PYG{n}{USER}      \PYG{n}{PR}  \PYG{n}{NI}  \PYG{n}{VIRT}  \PYG{n}{RES}  \PYG{n}{SHR} \PYG{n}{S} \PYG{o}{\PYGZpc{}}\PYG{n}{CPU} \PYG{o}{\PYGZpc{}}\PYG{n}{MEM}    \PYG{n}{TIME}\PYG{o}{+}  \PYG{n}{COMMAND}
 \PYG{l+m+mi}{16417} \PYG{n}{root}      \PYG{l+m+mi}{20}   \PYG{l+m+mi}{0} \PYG{l+m+mi}{42268} \PYG{l+m+mi}{8980} \PYG{l+m+mi}{1240} \PYG{n}{S}  \PYG{l+m+mf}{0.0}  \PYG{l+m+mf}{1.8}   \PYG{l+m+mi}{0}\PYG{p}{:}\PYG{l+m+mf}{02.70} \PYG{n}{redis}\PYG{o}{\PYGZhy{}}\PYG{n}{server}

\PYG{o}{\PYGZgt{}\PYGZgt{}}\PYG{n}{redis}\PYG{o}{\PYGZhy{}}\PYG{n}{cli} \PYG{n}{info监控}

 \PYG{n}{root}\PYG{n+nd}{@ubuntu1204base}\PYG{p}{:}\PYG{o}{/}\PYG{n}{home}\PYG{o}{/}\PYG{n}{cxl}\PYG{o}{/}\PYG{n}{redis}\PYG{o}{/}\PYG{n}{redis}\PYG{o}{\PYGZhy{}}\PYG{l+m+mf}{3.2}\PYG{o}{.}\PYG{l+m+mi}{6}\PYG{c+c1}{\PYGZsh{} src/redis\PYGZhy{}cli \PYGZhy{}h 10.0.0.23 \PYGZhy{}a prettydogKnockTheDoor info\textbar{}grep used\PYGZus{}memory}
 \PYG{n}{used\PYGZus{}memory}\PYG{p}{:}\PYG{l+m+mi}{7470144}
 \PYG{n}{used\PYGZus{}memory\PYGZus{}human}\PYG{p}{:}\PYG{l+m+mf}{7.12}\PYG{n}{M}
 \PYG{n}{used\PYGZus{}memory\PYGZus{}rss}\PYG{p}{:}\PYG{l+m+mi}{9023488}
 \PYG{n}{used\PYGZus{}memory\PYGZus{}rss\PYGZus{}human}\PYG{p}{:}\PYG{l+m+mf}{8.61}\PYG{n}{M}
 \PYG{n}{used\PYGZus{}memory\PYGZus{}peak}\PYG{p}{:}\PYG{l+m+mi}{7470144}
 \PYG{n}{used\PYGZus{}memory\PYGZus{}peak\PYGZus{}human}\PYG{p}{:}\PYG{l+m+mf}{7.12}\PYG{n}{M}
 \PYG{n}{used\PYGZus{}memory\PYGZus{}lua}\PYG{p}{:}\PYG{l+m+mi}{37888}
 \PYG{n}{used\PYGZus{}memory\PYGZus{}lua\PYGZus{}human}\PYG{p}{:}\PYG{l+m+mf}{37.00}\PYG{n}{K}
\end{sphinxVerbatim}

\end{itemize}


\subsubsection{代码示例}
\label{\detokenize{program/redis:id4}}\begin{itemize}
\item {} 
predis

\begin{sphinxVerbatim}[commandchars=\\\{\}]
好处是不需要安装php扩展,直接require就行了
git clone git://github.com/nrk/predis.git
wget https://github.com/nrk/predis/archive/v1.1.1.tar.gz
拷贝到Thinkphp vendor目录下

示例代码
\PYGZlt{}pre\PYGZgt{}
       require\PYGZus{}once VENDOR\PYGZus{}PATH.\PYGZdq{}predis/autoload.php\PYGZdq{};
       try \PYGZob{}
           \PYGZdl{}redis = new Predis\PYGZbs{}Client(\PYGZdq{}redis://127.0.0.1:6379/\PYGZdq{});
           \PYGZdl{}redis\PYGZhy{}\PYGZgt{}set(\PYGZsq{}library\PYGZsq{}, \PYGZsq{}predis1\PYGZsq{});
           \PYGZdl{}retval = \PYGZdl{}redis\PYGZhy{}\PYGZgt{}get(\PYGZsq{}library1\PYGZsq{});
           var\PYGZus{}export(\PYGZdl{}retval);
       \PYGZcb{} catch (Exception \PYGZdl{}e) \PYGZob{}
           var\PYGZus{}dump(\PYGZdl{}e\PYGZhy{}\PYGZgt{}getMessage());
       \PYGZcb{}

\PYGZlt{}/pre\PYGZgt{}

predis函数集:http://www.open\PYGZhy{}open.com/lib/view/open1355830836135.html
\end{sphinxVerbatim}

\end{itemize}


\subsection{usage}
\label{\detokenize{program/redis:usage}}\begin{itemize}
\item {} 
key

\begin{sphinxVerbatim}[commandchars=\\\{\}]
1、help set
SET key value [EX seconds] [PX milliseconds] [NX\textbar{}XX]
summary: Set the string value of a key
since: 1.0.0
group: string

set test 10
set test 10 EX 60 60秒过期
\end{sphinxVerbatim}

\item {} \begin{description}
\item[{list}] \leavevmode\begin{enumerate}
\item {} \begin{description}
\item[{查询list}] \leavevmode
lrange queues:importQueue 0 -1

\end{description}

\item {} \begin{description}
\item[{从队列头pop元素,在队列里删除该元素}] \leavevmode
lpop queues:importQueue

\end{description}

\item {} \begin{description}
\item[{清空队列}] \leavevmode
ltrim queues:importQueue 1 0

\end{description}

\end{enumerate}

\end{description}

\item {} \begin{description}
\item[{set}] \leavevmode\begin{itemize}
\item {} 
zset(sorted set)操作相关

\begin{sphinxVerbatim}[commandchars=\\\{\}]
\PYG{l+m+mf}{1.} \PYG{n}{查询}
    \PYG{n}{zrange} \PYG{n}{queues}\PYG{p}{:}\PYG{n}{importQueue}\PYG{p}{:}\PYG{n}{reserved} \PYG{l+m+mi}{0} \PYG{o}{\PYGZhy{}}\PYG{l+m+mi}{1}
\PYG{l+m+mf}{2.} \PYG{n}{删除所有元素}
    \PYG{n}{zrem} \PYG{n}{queues}\PYG{p}{:}\PYG{n}{importQueue}\PYG{p}{:}\PYG{n}{reserved} \PYG{l+m+mi}{0} \PYG{o}{\PYGZhy{}}\PYG{l+m+mi}{1}
\end{sphinxVerbatim}

\item {} 
hash

\begin{sphinxVerbatim}[commandchars=\\\{\}]
\PYG{l+m+mf}{1.} \PYG{n}{查看hash中的值和value}
    \PYG{n}{hgetall} \PYG{n}{job\PYGZus{}result}
    \PYG{l+m+mi}{1}\PYG{p}{)} \PYG{l+s+s2}{\PYGZdq{}}\PYG{l+s+s2}{importQueue\PYGZus{}p0yzx4AQEcVefgUnvZ9nLLiUprwP6ff5}\PYG{l+s+s2}{\PYGZdq{}}
    \PYG{l+m+mi}{2}\PYG{p}{)} \PYG{l+s+s2}{\PYGZdq{}}\PYG{l+s+s2}{\PYGZob{}}\PYG{l+s+se}{\PYGZbs{}\PYGZdq{}}\PYG{l+s+s2}{job\PYGZus{}status}\PYG{l+s+se}{\PYGZbs{}\PYGZdq{}}\PYG{l+s+s2}{:1\PYGZcb{}}\PYG{l+s+s2}{\PYGZdq{}}
\PYG{l+m+mf}{2.} \PYG{n}{删除值}
    \PYG{n}{hDel} \PYG{n}{job\PYGZus{}result} \PYG{n}{importQueue\PYGZus{}p0yzx4AQEcVefgUnvZ9nLLiUprwP6ff5}
    \PYG{p}{(}\PYG{n}{integer}\PYG{p}{)} \PYG{l+m+mi}{1}
\end{sphinxVerbatim}

\end{itemize}

\end{description}

\end{itemize}


\section{nodejs}
\label{\detokenize{program/nodejs::doc}}\label{\detokenize{program/nodejs:nodejs}}

\subsection{example}
\label{\detokenize{program/nodejs:example}}

\subsubsection{切换到淘宝镜像}
\label{\detokenize{program/nodejs:id1}}
npm install -g cnpm \textendash{}registry=https//registry.npm.taobao.org


\chapter{PROJECT}
\label{\detokenize{project/index:project}}\label{\detokenize{project/index::doc}}

\section{spi}
\label{\detokenize{project/spi:spi}}\label{\detokenize{project/spi::doc}}\begin{quote}

spi project
\end{quote}


\chapter{VERSION CONTROL}
\label{\detokenize{versionCtrl/index:version-control}}\label{\detokenize{versionCtrl/index::doc}}

\section{git}
\label{\detokenize{versionCtrl/git:git}}\label{\detokenize{versionCtrl/git::doc}}

\subsection{refer}
\label{\detokenize{versionCtrl/git:refer}}

\subsection{command example}
\label{\detokenize{versionCtrl/git:command-example}}\begin{itemize}
\item {} 
使用远程覆盖本地

\begin{sphinxVerbatim}[commandchars=\\\{\}]
\PYG{n}{git} \PYG{n}{fetch} \PYG{o}{\PYGZhy{}}\PYG{o}{\PYGZhy{}}\PYG{n+nb}{all}
\PYG{n}{git} \PYG{n}{reset} \PYG{o}{\PYGZhy{}}\PYG{o}{\PYGZhy{}}\PYG{n}{hard} \PYG{n}{origin}\PYG{o}{/}\PYG{n}{master}
\end{sphinxVerbatim}

\item {} 
错误:fatal: Pathspec ‘shell-baselib/cxl\_cli’ is in submodule ‘shell-baselib

\begin{sphinxVerbatim}[commandchars=\\\{\}]
\PYG{n}{git} \PYG{n}{rm} \PYG{o}{\PYGZhy{}}\PYG{o}{\PYGZhy{}}\PYG{n}{cached} \PYG{n}{shell}\PYG{o}{\PYGZhy{}}\PYG{n}{baselib}\PYG{o}{/}
\PYG{n}{git} \PYG{n}{add} \PYG{n}{shell}\PYG{o}{\PYGZhy{}}\PYG{n}{baselib}\PYG{o}{/}
\end{sphinxVerbatim}

\item {} 
merge其他分支(或远程分支),有冲突是使用远程覆盖本地

\begin{sphinxVerbatim}[commandchars=\\\{\}]
\PYG{n}{git} \PYG{n}{svn} \PYG{n}{clone} \PYG{n}{svn}\PYG{p}{:}\PYG{o}{/}\PYG{o}{/}\PYG{l+m+mf}{192.168}\PYG{o}{.}\PYG{l+m+mf}{1.66}\PYG{o}{/}\PYG{n}{namtso}\PYG{o}{/}\PYG{n}{branch}\PYG{o}{/}\PYG{n}{web\PYGZus{}code}\PYG{o}{/}\PYG{n}{svntest}\PYG{o}{/}\PYG{n}{front} \PYG{o}{\PYGZhy{}}\PYG{n}{r31772}\PYG{p}{:}\PYG{n}{HEAD}
\PYG{n}{git} \PYG{n}{remote} \PYG{n}{add} \PYG{n}{origin} \PYG{n}{git}\PYG{n+nd}{@gitlab28}\PYG{p}{:}\PYG{n}{chenxuelin}\PYG{o}{/}\PYG{n}{spifront}
\PYG{n}{git} \PYG{n}{fetch} \PYG{n}{origin} \PYG{n}{master}
\PYG{n}{git} \PYG{n}{merge} \PYG{o}{\PYGZhy{}}\PYG{n}{m} \PYG{l+s+s2}{\PYGZdq{}}\PYG{l+s+s2}{merge from git}\PYG{l+s+s2}{\PYGZdq{}} \PYG{o}{\PYGZhy{}}\PYG{n}{X} \PYG{n}{theirs} \PYG{n}{origin}\PYG{o}{/}\PYG{n}{master}
\PYG{n}{echo} \PYG{n}{xuelin}\PYG{o}{\textbar{}}\PYG{n}{git} \PYG{n}{svn} \PYG{n}{clone} \PYG{l+s+s2}{\PYGZdq{}}\PYG{l+s+s2}{svn://192.168.1.66/EmicallDev/ApplicationPlatform/sandbox/WebApplication/spi\PYGZhy{}front}\PYG{l+s+s2}{\PYGZdq{}} \PYG{o}{\PYGZhy{}}\PYG{n}{r3995}\PYG{p}{:}\PYG{n}{HEAD} \PYG{o}{\PYGZhy{}}\PYG{o}{\PYGZhy{}}\PYG{n}{username} \PYG{n}{chenxl}
\PYG{n}{git} \PYG{n}{remote} \PYG{n}{add} \PYG{n}{origin} \PYG{n}{git}\PYG{n+nd}{@gitlab28}\PYG{p}{:}\PYG{n}{websrc}\PYG{o}{/}\PYG{n}{vuefront}

\PYG{n}{git} \PYG{n}{svn} \PYG{n}{clone} \PYG{n}{svn}\PYG{p}{:}\PYG{o}{/}\PYG{o}{/}\PYG{l+m+mf}{192.168}\PYG{o}{.}\PYG{l+m+mf}{1.66}\PYG{o}{/}\PYG{n}{namtso}\PYG{o}{/}\PYG{n}{branch}\PYG{o}{/}\PYG{n}{web\PYGZus{}code}\PYG{o}{/}\PYG{n}{svntest}\PYG{o}{/}\PYG{n}{front} \PYG{o}{\PYGZhy{}}\PYG{n}{r31772}\PYG{p}{:}\PYG{n}{HEAD}
\PYG{n}{git} \PYG{n}{remote} \PYG{n}{add} \PYG{n}{origin} \PYG{n}{git}\PYG{n+nd}{@gitlab28}\PYG{p}{:}\PYG{n}{chenxuelin}\PYG{o}{/}\PYG{n}{spifront}
\PYG{n}{git} \PYG{n}{fetch} \PYG{n}{origin} \PYG{n}{master}
\PYG{n}{git} \PYG{n}{reset} \PYG{o}{\PYGZhy{}}\PYG{o}{\PYGZhy{}}\PYG{n}{hard} \PYG{n}{origin}\PYG{o}{/}\PYG{n}{master}

\PYG{n}{git} \PYG{n}{merge} \PYG{o}{\PYGZhy{}}\PYG{n}{X} \PYG{n}{theirs} \PYG{o}{\PYGZhy{}}\PYG{n}{m} \PYG{l+s+s2}{\PYGZdq{}}\PYG{l+s+s2}{merge from git when confict use theirs}\PYG{l+s+s2}{\PYGZdq{}} \PYG{n}{origin} \PYG{n}{master}
\PYG{n}{sed} \PYG{o}{\PYGZhy{}}\PYG{n}{i} \PYG{l+s+s1}{\PYGZsq{}}\PYG{l+s+s1}{/\PYGZca{}dist}\PYG{l+s+s1}{\PYGZbs{}}\PYG{l+s+s1}{//d}\PYG{l+s+s1}{\PYGZsq{}} \PYG{o}{.}\PYG{n}{gitignore}
\PYG{n}{npm} \PYG{n}{config} \PYG{n+nb}{set} \PYG{n}{registry} \PYG{l+s+s2}{\PYGZdq{}}\PYG{l+s+s2}{ https://registry.npm.taobao.org }\PYG{l+s+s2}{\PYGZdq{}}
\PYG{n}{svn} \PYG{n}{info} \PYG{n}{svn}\PYG{p}{:}\PYG{o}{/}\PYG{o}{/}\PYG{l+m+mf}{192.168}\PYG{o}{.}\PYG{l+m+mf}{1.66}\PYG{o}{/}\PYG{n}{EmicallDev}\PYG{o}{/}\PYG{n}{ApplicationPlatform}\PYG{o}{/}\PYG{n}{sandbox}\PYG{o}{/}\PYG{n}{WebApplication}\PYG{o}{/}\PYG{n}{spi}\PYG{o}{\PYGZhy{}}\PYG{n}{front} \PYG{o}{\textbar{}} \PYG{n}{awk} \PYG{l+s+s1}{\PYGZsq{}}\PYG{l+s+s1}{(\PYGZdl{}1==}\PYG{l+s+s1}{\PYGZdq{}}\PYG{l+s+s1}{Last}\PYG{l+s+s1}{\PYGZdq{}}\PYG{l+s+s1}{)}\PYG{l+s+s1}{\PYGZob{}}\PYG{l+s+s1}{if(\PYGZdl{}3==}\PYG{l+s+s1}{\PYGZdq{}}\PYG{l+s+s1}{Rev:}\PYG{l+s+s1}{\PYGZdq{}}\PYG{l+s+s1}{)}\PYG{l+s+s1}{\PYGZob{}}\PYG{l+s+s1}{print \PYGZdl{}4\PYGZcb{}\PYGZcb{}}\PYG{l+s+s1}{\PYGZsq{}}
\end{sphinxVerbatim}

\item {} 
开发使用git,打版本使用svn

\begin{sphinxVerbatim}[commandchars=\\\{\}]
\PYG{n}{echo} \PYG{n}{xuelin}\PYG{o}{\textbar{}}\PYG{n}{git} \PYG{n}{svn} \PYG{n}{clone} \PYG{l+s+s2}{\PYGZdq{}}\PYG{l+s+s2}{\PYGZdl{}svnfront}\PYG{l+s+s2}{\PYGZdq{}} \PYG{o}{\PYGZhy{}}\PYG{l+s+sa}{r}\PYG{l+s+s2}{\PYGZdq{}}\PYG{l+s+s2}{\PYGZdl{}svnLastVersion}\PYG{l+s+s2}{\PYGZdq{}}\PYG{p}{:}\PYG{n}{HEAD} \PYG{o}{\PYGZhy{}}\PYG{o}{\PYGZhy{}}\PYG{n}{username} \PYG{n}{chenxl}
\PYG{n}{git} \PYG{n}{remote} \PYG{n}{add} \PYG{n}{origin} \PYG{l+s+s2}{\PYGZdq{}}\PYG{l+s+s2}{\PYGZdl{}gitfront}\PYG{l+s+s2}{\PYGZdq{}}
\PYG{n}{git} \PYG{n}{fetch} \PYG{n}{origin} \PYG{n}{master}
\PYG{n}{git} \PYG{n}{merge} \PYG{o}{\PYGZhy{}}\PYG{n}{m} \PYG{l+s+s2}{\PYGZdq{}}\PYG{l+s+s2}{merge from git}\PYG{l+s+s2}{\PYGZdq{}} \PYG{o}{\PYGZhy{}}\PYG{n}{X} \PYG{n}{theirs} \PYG{n}{origin}\PYG{o}{/}\PYG{n}{master}
\PYG{n}{git} \PYG{n}{svn} \PYG{n}{dcommit}
\end{sphinxVerbatim}

\item {} 
配置ssh方式连接github和本地gitlab

\begin{sphinxVerbatim}[commandchars=\\\{\}]
\PYGZsh{} cd \PYGZti{}/.ssh
\PYGZsh{} ssh\PYGZhy{}keygen \PYGZhy{}t rsa \PYGZhy{}C \PYGZdq{}303566@qq.com\PYGZdq{} \PYGZhy{}f \PYGZti{}/.ssh/github
\PYGZsh{} 在目录下有两个文件 github和github.pub,cat github.pub拷贝内容
\PYGZsh{} 登录github》》your profile》》ssh keys》》粘贴内容到sshkey
\PYGZsh{} ssh\PYGZhy{}keygen \PYGZhy{}t rsa \PYGZhy{}C \PYGZdq{}chenxuelin@emicnet.com\PYGZdq{} \PYGZhy{}f \PYGZti{}/.ssh/gitlib\PYGZhy{}cxl
\PYGZsh{} 在目录下有两个文件 gitlib\PYGZhy{}cxl和gitlib\PYGZhy{}cxl.pub,cat gitlib\PYGZhy{}cxl.pub拷贝内容
\PYGZsh{} 登录gitlib》》your profile》》ssh keys》》粘贴内容到sshkey
\PYGZsh{} 创建\PYGZti{}/.ssh/config
\PYGZsh{}303566\PYGZhy{}github
  host github
  user git
  hostname github.com
  port 22
  identityfile \PYGZti{}/.ssh/github
  \PYGZsh{}chenxuelin\PYGZhy{}gitlib28
  host gitlib28
  user git
  hostname 10.0.0.28
  port 22
  identityfile \PYGZti{}/.ssh/gitlib\PYGZhy{}cxl
\PYGZsh{} 测试
\PYGZsh{}\PYGZsh{} root@1604developer:\PYGZti{}/.ssh\PYGZsh{} ssh \PYGZhy{}T gitlib28
    Welcome to GitLab, chenxuelin!
\PYGZsh{}\PYGZsh{} root@1604developer:\PYGZti{}/.ssh\PYGZsh{} ssh \PYGZhy{}T github
    Hi xuelinchen! You\PYGZsq{}ve successfully authenticated, but GitHub does not provide shell access.
\PYGZsh{} 在github创建项目
   git clone git@github:xuelinchen/schedule.git
   \PYGZsh{} 增加remote仓库
   git remote add gitlib28 git@gitlib28:chenxuelin/schedule.git
   \PYGZsh{} 提交结果到远程仓库
   git push gitlib28
\end{sphinxVerbatim}

\end{itemize}


\subsection{git2svn}
\label{\detokenize{versionCtrl/git:git2svn}}
\# clone git project

\begin{sphinxVerbatim}[commandchars=\\\{\}]
\PYG{n}{git} \PYG{n}{clone} \PYG{n}{git}\PYG{n+nd}{@gitlib28}\PYG{p}{:}\PYG{n}{chenxuelin}\PYG{o}{/}\PYG{n}{schedule}\PYG{o}{.}\PYG{n}{git}
\end{sphinxVerbatim}

\# check out svn project with same directory of git project

\begin{sphinxVerbatim}[commandchars=\\\{\}]
\PYG{n}{svn} \PYG{n}{co} \PYG{n}{svn}\PYG{p}{:}\PYG{o}{/}\PYG{o}{/}\PYG{l+m+mf}{192.168}\PYG{o}{.}\PYG{l+m+mf}{1.66}\PYG{o}{/}\PYG{n}{namtso}\PYG{o}{/}\PYG{n}{branch}\PYG{o}{/}\PYG{n}{web\PYGZus{}code}\PYG{o}{/}\PYG{n}{svntest} \PYG{n}{schedule} \PYG{o}{\PYGZhy{}}\PYG{o}{\PYGZhy{}}\PYG{n}{username} \PYG{n}{chenxl} \PYG{o}{\PYGZhy{}}\PYG{o}{\PYGZhy{}}\PYG{n}{password} \PYG{n}{xuelin}
\end{sphinxVerbatim}

\# add ignore of svn,attention that has a enter after git and has dot in the end

\begin{sphinxVerbatim}[commandchars=\\\{\}]
\PYG{n}{svn} \PYG{n}{propset} \PYG{n}{svn}\PYG{p}{:}\PYG{n}{ignore} \PYG{l+s+s2}{\PYGZdq{}}\PYG{l+s+s2}{.git}
\PYG{o}{.}\PYG{n}{gitignore}\PYG{l+s+s2}{\PYGZdq{}}\PYG{l+s+s2}{ .}
\PYG{n}{svn} \PYG{n}{status} \PYG{n}{查看状态}
\end{sphinxVerbatim}

\# commit to svn

\begin{sphinxVerbatim}[commandchars=\\\{\}]
\PYG{n}{svn} \PYG{n}{add} \PYG{o}{*}
\PYG{n}{svn} \PYG{n}{commit} \PYG{o}{\PYGZhy{}}\PYG{n}{m} \PYG{l+s+s2}{\PYGZdq{}}\PYG{l+s+s2}{result from git}\PYG{l+s+s2}{\PYGZdq{}}  \PYG{o}{\PYGZhy{}}\PYG{o}{\PYGZhy{}}\PYG{n}{username} \PYG{n}{chenxl} \PYG{o}{\PYGZhy{}}\PYG{o}{\PYGZhy{}}\PYG{n}{password} \PYG{n}{xuelin}
\end{sphinxVerbatim}

\# commit to git

\begin{sphinxVerbatim}[commandchars=\\\{\}]
\PYG{n}{git} \PYG{n}{add} \PYG{o}{*}
\PYG{n}{git} \PYG{n}{commit} \PYG{o}{\PYGZhy{}}\PYG{n}{m} \PYG{l+s+s2}{\PYGZdq{}}\PYG{l+s+s2}{add .svn path to git res}\PYG{l+s+s2}{\PYGZdq{}}
\PYG{n}{git} \PYG{n}{push}
\end{sphinxVerbatim}

\# add test file in git and check in to svn

\begin{sphinxVerbatim}[commandchars=\\\{\}]
\PYG{n}{vi} \PYG{n}{test}\PYG{o}{.}\PYG{n}{md}
\PYG{n}{git} \PYG{n}{add} \PYG{n}{test}\PYG{o}{.}\PYG{n}{md}
\PYG{n}{git} \PYG{n}{commit} \PYG{o}{\PYGZhy{}}\PYG{n}{m} \PYG{l+s+s1}{\PYGZsq{}}\PYG{l+s+s1}{add test.md}\PYG{l+s+s1}{\PYGZsq{}}
\PYG{n}{svn} \PYG{n}{add} \PYG{n}{test}\PYG{o}{.}\PYG{n}{md}
\PYG{n}{svn} \PYG{n}{commit} \PYG{o}{\PYGZhy{}}\PYG{n}{m} \PYG{l+s+s1}{\PYGZsq{}}\PYG{l+s+s1}{check in file from git}\PYG{l+s+s1}{\PYGZsq{}}  \PYG{o}{\PYGZhy{}}\PYG{o}{\PYGZhy{}}\PYG{n}{username} \PYG{n}{chenxl} \PYG{o}{\PYGZhy{}}\PYG{o}{\PYGZhy{}}\PYG{n}{password} \PYG{n}{xuelin}
\PYG{n}{git} \PYG{n}{commit} \PYG{o}{\PYGZhy{}}\PYG{n}{m} \PYG{l+s+s1}{\PYGZsq{}}\PYG{l+s+s1}{delete test.md}\PYG{l+s+s1}{\PYGZsq{}}
\PYG{n}{git} \PYG{n}{pull}
\PYG{n}{svn} \PYG{n}{add} \PYG{o}{*} \PYG{o}{\PYGZhy{}}\PYG{o}{\PYGZhy{}}\PYG{n}{force}
\PYG{n}{svn} \PYG{n}{commit} \PYG{o}{\PYGZhy{}}\PYG{n}{m} \PYG{l+s+s1}{\PYGZsq{}}\PYG{l+s+s1}{test add file}\PYG{l+s+s1}{\PYGZsq{}} \PYG{o}{\PYGZhy{}}\PYG{o}{\PYGZhy{}}\PYG{n}{username} \PYG{n}{chenxl} \PYG{o}{\PYGZhy{}}\PYG{o}{\PYGZhy{}}\PYG{n}{password} \PYG{n}{xuelin}
\PYG{n}{删除文件无法自动提交到svn中}
\end{sphinxVerbatim}

\# git-svn

\begin{sphinxVerbatim}[commandchars=\\\{\}]
\PYG{n}{echo} \PYG{n}{xuelin}\PYG{o}{\textbar{}}\PYG{n}{git} \PYG{n}{svn} \PYG{n}{clone} \PYG{n}{svn}\PYG{p}{:}\PYG{o}{/}\PYG{o}{/}\PYG{l+m+mf}{192.168}\PYG{o}{.}\PYG{l+m+mf}{1.66}\PYG{o}{/}\PYG{n}{namtso}\PYG{o}{/}\PYG{n}{branch}\PYG{o}{/}\PYG{n}{web\PYGZus{}code}\PYG{o}{/}\PYG{n}{tp5}\PYG{o}{\PYGZhy{}}\PYG{n}{project} \PYG{o}{\PYGZhy{}}\PYG{n}{r29318}\PYG{p}{:}\PYG{n}{HEAD} \PYG{o}{\PYGZhy{}}\PYG{o}{\PYGZhy{}}\PYG{n}{username} \PYG{n}{chenxl}
\PYG{n}{git} \PYG{n}{branch}
\PYG{n}{git} \PYG{n}{remote} \PYG{n}{add} \PYG{n}{origin} \PYG{n}{git}\PYG{n+nd}{@gitlib28}\PYG{p}{:}\PYG{n}{chenxuelin}\PYG{o}{/}\PYG{n}{schedule}\PYG{o}{.}\PYG{n}{git}
\PYG{n}{git} \PYG{n}{pull} \PYG{n}{origin} \PYG{n}{master} \PYG{n}{ctrl}\PYG{o}{+}\PYG{n}{x提交}
\PYG{n}{git} \PYG{n}{branch} \PYG{o}{\PYGZhy{}}\PYG{o}{\PYGZhy{}}\PYG{n+nb}{set}\PYG{o}{\PYGZhy{}}\PYG{n}{upstream}\PYG{o}{\PYGZhy{}}\PYG{n}{to}\PYG{o}{=}\PYG{n}{origin}\PYG{o}{/}\PYG{n}{master}
\PYG{n}{强制刷新本地版本为远程仓库}
\PYG{n}{git} \PYG{n}{fetch} \PYG{o}{\PYGZhy{}}\PYG{o}{\PYGZhy{}}\PYG{n+nb}{all}
\PYG{n}{git} \PYG{n}{reset} \PYG{o}{\PYGZhy{}}\PYG{o}{\PYGZhy{}}\PYG{n}{hard} \PYG{n}{origin}\PYG{o}{/}\PYG{n}{master}
\end{sphinxVerbatim}

\# Append svn:ignore settings to the default Git exclude file

\begin{sphinxVerbatim}[commandchars=\\\{\}]
\PYG{n}{git} \PYG{n}{svn} \PYG{n}{show}\PYG{o}{\PYGZhy{}}\PYG{n}{ignore} \PYG{o}{\PYGZgt{}\PYGZgt{}} \PYG{o}{.}\PYG{n}{git}\PYG{o}{/}\PYG{n}{info}\PYG{o}{/}\PYG{n}{exclude}
\PYG{n}{git} \PYG{n}{svn} \PYG{n}{dcommit}
\end{sphinxVerbatim}

\# from git commit to svn

\begin{sphinxVerbatim}[commandchars=\\\{\}]
1、create new clone project
    git clone git@gitlib28:chenxuelin/schedule
    cd schedule
2、create new svn clone project
    echo xuelin\textbar{}git svn clone svn://192.168.1.66/namtso/branch/web\PYGZus{}code/tp5 \PYGZhy{}r29318:HEAD \PYGZhy{}\PYGZhy{}username chenxl
    cd tp5
    git remote add origin git@gitlib28:chenxuelin/schedule.git
\end{sphinxVerbatim}

\# git version commit to svn

\begin{sphinxVerbatim}[commandchars=\\\{\}]
\PYG{n}{http}\PYG{p}{:}\PYG{o}{/}\PYG{o}{/}\PYG{n}{blog}\PYG{o}{.}\PYG{n}{csdn}\PYG{o}{.}\PYG{n}{net}\PYG{o}{/}\PYG{n}{zhangskd}\PYG{o}{/}\PYG{n}{article}\PYG{o}{/}\PYG{n}{details}\PYG{o}{/}\PYG{l+m+mi}{43452627}
\PYG{l+m+mf}{1.} \PYG{n}{git} \PYG{n}{clone} \PYG{n}{git}\PYG{n+nd}{@gitlib28}\PYG{p}{:}\PYG{n}{chenxuelin}\PYG{o}{/}\PYG{n}{schedule}
\PYG{l+m+mf}{2.} \PYG{n}{git} \PYG{n}{svn} \PYG{n}{init} \PYG{n}{svn}\PYG{p}{:}\PYG{o}{/}\PYG{o}{/}\PYG{l+m+mf}{192.168}\PYG{o}{.}\PYG{l+m+mf}{1.66}\PYG{o}{/}\PYG{n}{namtso}\PYG{o}{/}\PYG{n}{branch}\PYG{o}{/}\PYG{n}{web\PYGZus{}code}\PYG{o}{/}\PYG{n}{tp5}
\PYG{l+m+mf}{3.} \PYG{n}{git} \PYG{n}{svn} \PYG{n}{fetch}
\PYG{l+m+mf}{4.} \PYG{n}{git} \PYG{n}{show}\PYG{o}{\PYGZhy{}}\PYG{n}{ref} \PYG{n}{svn} \PYG{o}{\textbar{}} \PYG{n}{tail} \PYG{o}{\PYGZhy{}}\PYG{n}{n} \PYG{l+m+mi}{1}
\PYG{n}{f7e97acecbb8098757d9dc451b3c76eb59c4da9d} \PYG{n}{refs}\PYG{o}{/}\PYG{n}{remotes}\PYG{o}{/}\PYG{n}{origin}\PYG{o}{/}\PYG{n}{master}
\PYG{l+m+mf}{5.} \PYG{n}{git} \PYG{n}{log} \PYG{o}{\PYGZhy{}}\PYG{o}{\PYGZhy{}}\PYG{n}{pretty}\PYG{o}{=}\PYG{n}{oneline} \PYG{n}{master} \PYG{o}{\textbar{}} \PYG{n}{tail} \PYG{o}{\PYGZhy{}}\PYG{n}{n} \PYG{l+m+mi}{1}
    \PYG{l+m+mi}{8041}\PYG{n}{fab27f1ff38f980b9ea00fd335c50005af8c} \PYG{n}{nf}\PYG{p}{:}  \PYG{n}{create} \PYG{n}{svntest} \PYG{n}{project}
\PYG{l+m+mf}{6.} \PYG{n}{echo} \PYG{l+s+s2}{\PYGZdq{}}\PYG{l+s+s2}{8041fab27f1ff38f980b9ea00fd335c50005af8c 5194b04324e1fa8bcec215ec053c5e707d2e4b83}\PYG{l+s+s2}{\PYGZdq{}} \PYG{o}{\PYGZgt{}\PYGZgt{}} \PYG{o}{.}\PYG{n}{git}\PYG{o}{/}\PYG{n}{info}\PYG{o}{/}\PYG{n}{grafts}
    \PYG{n}{试验失败}
\end{sphinxVerbatim}

\# another

\begin{sphinxVerbatim}[commandchars=\\\{\}]
1、echo xuelin\textbar{}git svn clone svn://192.168.1.66/namtso/branch/web\PYGZus{}code/tp5 \PYGZhy{}r29318:HEAD \PYGZhy{}\PYGZhy{}username chenxl
2、git remote add origin git@gitlib28:chenxuelin/schedule.git
3、git branch \PYGZhy{}\PYGZhy{}set\PYGZhy{}upstream\PYGZhy{}to=origin/master master
4、git pull origin
5、git svn dcommit
6、git checkout \PYGZhy{}b newfeature
7、git checkout master
8、git merge newfeature
9、git push origin
10、git svn dcommit
\end{sphinxVerbatim}

\# schedule project

\begin{sphinxVerbatim}[commandchars=\\\{\}]
1. install git\PYGZhy{}svn
   apt install git\PYGZhy{}svn
2. clone svn project,that is empty project
   echo xuelin\textbar{}git svn clone svn://192.168.1.66/namtso/branch/web\PYGZus{}code/schedule \PYGZhy{}r29318:HEAD \PYGZhy{}\PYGZhy{}username chenxl
3. update from svn
   git svn rebase
4. find all branch
   git branch \PYGZhy{}a
5. add remote git repository
   git remote add origin git@gitlab28:chenxuelin/schedule.git
6. set master branch upstream
   git branch \PYGZhy{}\PYGZhy{}set\PYGZhy{}upstream\PYGZhy{}to=origin/master master
7. pull data from remote git
   git pull origin ==》ctrl+x
8. submit to svn
   git svn dcommit
9. develope in branch
   git checkout \PYGZhy{}b \PYGZdq{}cxl\PYGZdq{}
   git rm test.md
   git commit \PYGZhy{}m \PYGZdq{}made some change\PYGZdq{}
10.return to master for push to remote of git\PYGZam{}svn
   git checkout master
   git merge cxl
   git push origin master
   git svn dcommit
11.add ignore from git to svn
   git svn show\PYGZhy{}ignore \PYGZgt{}\PYGZgt{} .git/info/exclude
\end{sphinxVerbatim}

\# tp5 project

\begin{sphinxVerbatim}[commandchars=\\\{\}]
\PYG{l+m+mf}{1.} \PYG{n}{clone} \PYG{n}{svn} \PYG{n}{project}\PYG{p}{,}\PYG{n}{that} \PYG{o+ow}{is} \PYG{n}{empty} \PYG{n}{project}
   \PYG{n}{echo} \PYG{n}{xuelin}\PYG{o}{\textbar{}}\PYG{n}{git} \PYG{n}{svn} \PYG{n}{clone} \PYG{n}{svn}\PYG{p}{:}\PYG{o}{/}\PYG{o}{/}\PYG{l+m+mf}{192.168}\PYG{o}{.}\PYG{l+m+mf}{1.66}\PYG{o}{/}\PYG{n}{namtso}\PYG{o}{/}\PYG{n}{branch}\PYG{o}{/}\PYG{n}{web\PYGZus{}code}\PYG{o}{/}\PYG{n}{spi} \PYG{o}{\PYGZhy{}}\PYG{n}{r29318}\PYG{p}{:}\PYG{n}{HEAD} \PYG{o}{\PYGZhy{}}\PYG{o}{\PYGZhy{}}\PYG{n}{username} \PYG{n}{chenxl}
\PYG{l+m+mf}{2.} \PYG{n}{cd} \PYG{n}{spi}
\PYG{l+m+mf}{3.} \PYG{n}{add} \PYG{n}{remote} \PYG{n}{git} \PYG{n}{repository}
   \PYG{n}{git} \PYG{n}{remote} \PYG{n}{add} \PYG{n}{origin} \PYG{n}{git}\PYG{n+nd}{@gitlab28}\PYG{p}{:}\PYG{n}{websrc}\PYG{o}{/}\PYG{n}{spi}\PYG{o}{.}\PYG{n}{git}
\PYG{l+m+mf}{4.} \PYG{n}{make} \PYG{n}{some} \PYG{n}{change}
\PYG{l+m+mf}{5.} \PYG{n}{git} \PYG{n}{add} \PYG{o}{.}
\PYG{l+m+mf}{6.} \PYG{n}{git} \PYG{n}{commit} \PYG{o}{\PYGZhy{}}\PYG{n}{m} \PYG{l+s+s1}{\PYGZsq{}}\PYG{l+s+s1}{add ignore}\PYG{l+s+s1}{\PYGZsq{}}
\PYG{l+m+mf}{7.} \PYG{n}{submit} \PYG{n}{to} \PYG{n}{remote} \PYG{n}{git}
        \PYG{n}{git} \PYG{n}{push} \PYG{o}{\PYGZhy{}}\PYG{n}{u} \PYG{n}{origin} \PYG{n}{master}
\PYG{l+m+mf}{8.} \PYG{n}{submit} \PYG{n}{to} \PYG{n}{svn}
   \PYG{n}{git} \PYG{n}{svn} \PYG{n}{dcommit}
\end{sphinxVerbatim}

\# spi project

\begin{sphinxVerbatim}[commandchars=\\\{\}]
1. clone svn project spi\PYGZhy{}front
        echo xuelin\textbar{}git svn clone svn://192.168.1.66/EmicallDev/ApplicationPlatform/sandbox/WebApplication/spi\PYGZhy{}front \PYGZhy{}r3000:HEAD \PYGZhy{}\PYGZhy{}username chenxl
2. clone svn project spi\PYGZhy{}php
        echo xuelin\textbar{}git svn clone svn://192.168.1.66/EmicallDev/ApplicationPlatform/sandbox/WebApplication/spi\PYGZhy{}php \PYGZhy{}r3000:HEAD \PYGZhy{}\PYGZhy{}username chenxl
3. cd spi\PYGZhy{}php
4. add remote git repository
   git remote add origin git@gitlab28:websrc/spi.git
5. view remote info
   git remote \PYGZhy{}v
6. pull data from remote git
        git pull origin
7. git branch \PYGZhy{}a
   * master
        remotes/git\PYGZhy{}svn
        remotes/origin/master
8. merge origin/master to local branch master
        git merge remotes/origin/master ==》ctrl+x
9. add ignore from git to svn
   git svn show\PYGZhy{}ignore \PYGZgt{}\PYGZgt{} .git/info/exclude
10.create branch of \PYGZdq{}bin\PYGZdq{}
   git checkout \PYGZhy{}b \PYGZdq{}bin\PYGZdq{}
11.commit data to svn
        git svn dcommit
12.cd spi\PYGZhy{}front
13.add remote git repository
        git remote add origin git@gitlab28:websrc/vuefront.git
14.pull data from remote git
        git pull origin master =\PYGZgt{} ctrl+x
15.add ignore from git to svn
        git svn show\PYGZhy{}ignore \PYGZgt{}\PYGZgt{} .git/info/exclude
16.commit data to svn
        git svn dcommit
\end{sphinxVerbatim}


\section{svn}
\label{\detokenize{versionCtrl/svn:svn}}\label{\detokenize{versionCtrl/svn::doc}}

\subsection{refer}
\label{\detokenize{versionCtrl/svn:refer}}

\subsection{install}
\label{\detokenize{versionCtrl/svn:install}}\begin{itemize}
\item {} 
安装1.8

\begin{sphinxVerbatim}[commandchars=\\\{\}]
\PYG{n}{sudo} \PYG{n}{sh} \PYG{o}{\PYGZhy{}}\PYG{n}{c} \PYG{l+s+s1}{\PYGZsq{}}\PYG{l+s+s1}{echo }\PYG{l+s+s1}{\PYGZdq{}}\PYG{l+s+s1}{deb http://opensource.wandisco.com/ubuntu precise svn18}\PYG{l+s+s1}{\PYGZdq{}}\PYG{l+s+s1}{ \PYGZgt{}\PYGZgt{} /etc/apt/sources.list.d/subversion18.list}\PYG{l+s+s1}{\PYGZsq{}}
\PYG{n}{sudo} \PYG{n}{wget} \PYG{o}{\PYGZhy{}}\PYG{n}{q} \PYG{n}{http}\PYG{p}{:}\PYG{o}{/}\PYG{o}{/}\PYG{n}{opensource}\PYG{o}{.}\PYG{n}{wandisco}\PYG{o}{.}\PYG{n}{com}\PYG{o}{/}\PYG{n}{wandisco}\PYG{o}{\PYGZhy{}}\PYG{n}{debian}\PYG{o}{.}\PYG{n}{gpg} \PYG{o}{\PYGZhy{}}\PYG{n}{O}\PYG{o}{\PYGZhy{}} \PYG{o}{\textbar{}} \PYG{n}{sudo} \PYG{n}{apt}\PYG{o}{\PYGZhy{}}\PYG{n}{key} \PYG{n}{add} \PYG{o}{\PYGZhy{}}
\PYG{n}{sudo} \PYG{n}{apt}\PYG{o}{\PYGZhy{}}\PYG{n}{get} \PYG{n}{update}
\PYG{n}{sudo} \PYG{n}{apt}\PYG{o}{\PYGZhy{}}\PYG{n}{cache} \PYG{n}{show} \PYG{n}{subversion} \PYG{o}{\textbar{}} \PYG{n}{grep} \PYG{l+s+s1}{\PYGZsq{}}\PYG{l+s+s1}{\PYGZca{}Version:}\PYG{l+s+s1}{\PYGZsq{}}
\PYG{n}{sudo} \PYG{n}{apt}\PYG{o}{\PYGZhy{}}\PYG{n}{get} \PYG{n}{install} \PYG{n}{subversion}
\end{sphinxVerbatim}

\end{itemize}


\subsection{usage}
\label{\detokenize{versionCtrl/svn:usage}}\begin{itemize}
\item {} 
创建项目目录

\begin{sphinxVerbatim}[commandchars=\\\{\}]
\PYG{n}{mkdir} \PYG{n}{svn}
\end{sphinxVerbatim}

\item {} 
创建仓库

\begin{sphinxVerbatim}[commandchars=\\\{\}]
\PYG{n}{svnadmin} \PYG{n}{create} \PYG{o}{/}\PYG{n}{home}\PYG{o}{/}\PYG{n}{svn}\PYG{o}{/}\PYG{n}{project}
\PYG{n}{chmod} \PYG{l+m+mi}{777} \PYG{o}{\PYGZhy{}}\PYG{n}{R} \PYG{o}{/}\PYG{n}{home}\PYG{o}{/}\PYG{n}{svn}\PYG{o}{/}\PYG{n}{project}
\end{sphinxVerbatim}

\item {} 
查看svn服务器是否支持 cyrus sasl 方式认证

\begin{sphinxVerbatim}[commandchars=\\\{\}]
\PYG{n}{svnserve} \PYG{o}{\PYGZhy{}}\PYG{o}{\PYGZhy{}}\PYG{n}{version}
\PYG{n}{是否存在} \PYG{n}{Cyrus} \PYG{n}{SASL} \PYG{n}{authentication} \PYG{o+ow}{is} \PYG{n}{available}\PYG{o}{.}
\end{sphinxVerbatim}

\item {} 
配置权限

\begin{sphinxVerbatim}[commandchars=\\\{\}]
1、vi /home/svn/project/conf/svnserve.conf
打开 auth\PYGZhy{}access = write 授权用户可写
password\PYGZhy{}db = passwd
2、vi /home/svn/project/conf/passwd
添加用户
[users]
\PYGZsh{} harry = harryssecret
\PYGZsh{} sally = sallyssecret
cxl = xuelin
wqx = qinxue
3、vi /home/svn/project/conf/authz
admin = cxl
@admin = rw
* =
admin=cxl,cxl用户属于admin权限组,@admin=rw,admin权限组可以read,其他用户不能读取
\end{sphinxVerbatim}

\item {} 
守护进程方式启动svn

\begin{sphinxVerbatim}[commandchars=\\\{\}]
\PYG{n}{svnserve} \PYG{o}{\PYGZhy{}}\PYG{n}{d} \PYG{o}{\PYGZhy{}}\PYG{n}{r} \PYG{o}{/}\PYG{n}{home}\PYG{o}{/}\PYG{n}{svn}
\end{sphinxVerbatim}

\item {} 
svnsync 同步测试
\begin{quote}

1、在源地址创建项目svn://192.168.1.66/namtso/branch/web\_code/svnsynctest
2、vi /home/svn/svnsynctest/hooks/pre\_revprop-change.bat
添加一行exit 0
3、配置权限,同上
4、初始化:
svnsync init  \textendash{}source-username chenxl \textendash{}source-password xuelin \textendash{}sync-username chenxl  \textendash{}sync-password xuelin svn://10.0.0.21/svnsynctest svn://192.168.1.66/namtso/branch/web\_code/svnsynctest
5、同步版本
svnsync sync \textendash{}source-username chenxl \textendash{}source-password xuelin \textendash{}sync-username chenxl  \textendash{}sync-password xuelin svn://10.0.0.21/svnsynctest
\end{quote}

\end{itemize}



\renewcommand{\indexname}{Index}
\printindex
\end{document}